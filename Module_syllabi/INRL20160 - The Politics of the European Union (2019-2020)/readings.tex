%!TEX root = INRL20160_Syllabus_1920.tex

\section*{Course reading}

\subsection*{Required readings:}

The following texts shall be used extensively throughout the course, so it is recommended that they are purchased:

\begin{itemize}
	\item Cini, M. and Borragán, N.P.S. 2019. European Union Politics. Oxford University Press.
	\item Hix, S. and H\o yland, B. 2011. The political system of the European Union. Palgrave Macmillan. 3rd Ed.
\end{itemize}

\subsection*{Recommended Readings}

The following books provide a general overview of the topics that will be covered in this module. 

\begin{itemize}
	\item Wallace, H., Pollack, M.A. and Young, A.R. eds. 2015. Policy-making in the European Union. Oxford University Press, USA.
	\item Bomberg, E., Peterson, J. and Corbett, R. 2012. The European Union: how does it work?. Oxford University Press.
	\item Bache, I., Bulmer, S., George, S. and Parker, O. 2014. Politics in the European Union. Oxford University Press, USA.
	\item Hix, S. 2013. What's Wrong with the Europe Union and How to Fix it. John Wiley \& Sons.
	\item Majone, G. 2009. Dilemmas of European integration: the ambiguities and pitfalls of integration by stealth. Oxford: Oxford University Press.
	\item Peterson, J. and Shackleton, M. 2012. The institutions of the European Union. Oxford University Press.
	\item Thomson, R. 2011. Resolving controversy in the European Union: legislative decision-making before and after enlargement. Cambridge University Press.
	\item Wiener, A. and Diez, T. 2009. European integration theory. Oxford: Oxford University Press.
\end{itemize}


In addition to these readings, students should keep up to date on current European affairs by reading daily newspapers, or one of the many websites devoted to EU politics. This reading is essential as it will allow you to keep up to date with current affairs in the EU and identify potential blog post topics. These websites include the following:

\begin{itemize}
	\item \url{http://www.euobserver.com}
	\item \url{http://europa.eu.int}
	\item \url{http://www.eupolitics.com}
	\item \url{http://www.ft.com}
	\item \url{http://blogs.lse.ac.uk/europpblog/}
	\item \url{http://blogs.lse.ac.uk/brexit/}
\end{itemize}

\section*{Brightspace}

Please make sure you have access to the module in Brightspace as soon as possible. It is the student's responsibility to make sure that they are signed up to the module correctly and they know how to submit coursework through the appropriate Brightspace assignment tab. If you have any issues with Brightspace contact the Brightspace support people to resolve the issue.

Furthermore, module materials such as this syllabus and announcements made outside lectures shall be on Brightspace. As such, Brightspace is an important communication tool for the module.

\subsection*{Detailed course programme}

\subsubsection*{Week 1}

%!TEX root = INRL20160_Syllabus_1920.tex

\subsubsection*{Introduction and the history of European integration}

Study questions

\begin{itemize}
	\item What are the major challenges, threats, and opportunities facing the EU today? How do you think these issues will affect the EU and shape its future?
	\item What are the main milestones that the integration process in Europe went through? 
	\item How has the European integration project developed over time and how has the policy remit of the EU evolved?
\end{itemize}

\noindent Key reading

\begin{itemize}
	\item Hix, S. and H\o yland, B. 2011. The political system of the European Union. Palgrave Macmillan. Chapter 1.
	\item Cini, M. and Borragán, N.P.S. 2019. European Union Politics. Oxford University Press. Chapter 1-3.
\end{itemize}

\noindent Further reading

\begin{itemize}
	\item Urwin, D.W. 2014. The community of Europe: A history of European integration since 1945. Routledge.
	\item Dinan, D. 2010. Ever Closer Union: An Introduction to European Integration. London: Palgrave Macmillan. Chapters 1-6.
	\item Marks, G. 2011. Europe and Its Empires: From Rome to the European Union. Journal of Common Market Studies 50(1), pp.1-20.
	\item Nugent, N. 2010. The Government and Politics of the European Union. Basingstoke: Palgrave Macmillan, Part 1: Chapters 1-7.
	\item Bache, I. and George, S. 2011. Politics in the European Union. Oxford: Oxford University Press, Chapters 5-11.
	\item Diebold, W. 1959. The Schuman Plan. New York: Praeger.
	\item Dougan, M. 2008. The Treaty of Lisbon 2007: Winning Minds not Hearts. Common Market Law Review 45, pp.617–703.
	\item Gray, M. and Stubb, A. 2001. Keynote Article: The Treaty of Nice–Negotiating a Poisoned Chalice?. JCMS: Journal of Common Market Studies, 39(s1), pp.5-23.
	\item Magnette, P. and Nicolaidis, K. 2004. The European Convention: bargaining in the shadow of rhetoric. West European Politics, 27(3), pp.381-404.
	\item Majone, G. 2006. The Common Sense of European Integration. Journal of European Public Policy 13(5), pp.607-626.
	\item Piris, J-C. 2010. The Lisbon Treaty: a legal and political analysis. Cambridge: Cambridge University Press.
\end{itemize}

\subsubsection*{Week 2 and Week 3}

%!TEX root = INRL20160_Syllabus_1920.tex

\subsubsection*{Theories of European integration}

Study questions

\begin{itemize}
	\item Describe the main logic behind neofunctionalism as a theory of European integration. Can this theory successfully explain how we got to where we are now in terms of EU integration?
	\item Describe the main logic behind intergovernmentalism. Does this theory do a better job of explaining the European integration project than neofunctionalism?
	\item What are the strengths and weaknesses of both theories of integration?
	\item What are the three different types of institutionalism? How do they differ from one another?
	\item How do institutionalist theories differ from the `grand' theories of integration that went before?
	\item How successful are institutionalist theories at explaining EU politics?
\end{itemize}

\noindent Key reading

\begin{itemize}
	\item Cini, M. and Pérez-Solórzano Borragán, N. (eds.) 2019. European Union Politics, Oxford: Oxford University Press. Chapters 54-5
	\item Moravcsik, A. 1993. Preferences and Power in the European Community. Journal of Common Market Studies 31(4), pp.473-524.
	\item Aspinwall, M. and Schneider, G. 2000. Same Menu, Separate Tables: The Institutionalist Turn in Political Science and the Study of European Integration. European Journal of Political Research 38, pp.1-36.
\end{itemize}


\noindent Further reading

\begin{itemize}
	\item Hoffmann, S. 1966. Obstinate or obsolete? The fate of the nation-state and the case of Western Europe. Daedalus, pp.862-915.
	\item Haas, E.B. 1976. Turbulent Fields and the Theory of Regional Integration. International Organization 30, pp.173-212.
	\item Monnet, J. 1978. Memoirs. New York: Doubleday.
	\item Bache, I. and George, S. 2011. Politics in the European Union. Oxford: Oxford University Press, Chapter 1.
	\item Haas, E.B. 1958. The Uniting of Europe. Stanford: Stanford University Press.
	\item Haas, E.B. 1961. International Integration: The European and Universal Process. International Organization 15(3), pp.366-92.
	\item Haas, E.B. 1975. The Obsolescence of Regional Integration Theory. Berkeley: Institute of International Studies.
	\item Hooghe, L and Marks G. 2001. Multi-Level Governance and European Integration. Oxford: Rowman and Littlefield.
	\item Jupille, J. and Caporaso, J.A. 1999. Institutionalism and the European Union: Beyond Comparative Politics and International Relations. Annual Review Political Science 2, pp.429-44.
	\item Moravcsik, A. 1998. The Choice For Europe: Social Purpose And State Power From Messina To Maastricht. Cornell University Press, Chapter 1, 7.
	\item Nugent, N. 2010. The Government and Politics of the European Union. London: Palgrave Macmillan, Chapter 23.
	\item Pollack, M.A. 2003. The Engines of European Integration: Delegation, Agency and Agenda Setting in the EU. Oxford: Oxford University Press.
	\item Pollock, M.A. 2005. Theorizing the EU: International Organization, Domestic Polity, or Experiment in New Governance? Annual Review Political Science. 8, pp.357–98.
	\item Rosamond, B. 2000. Theories of European Integration London: MacMillan, Chapters 2 and 3. 
	\item Sandholtz, W. And Sweet Stone, A. 1998. European Integration and Supranational Governance. Oxford: Oxford University Press.
	\item Weiner, A. and Diez, T. 2009. European Integration Theory. Oxford: Oxford University Press (2nd Edition).
	\item Hooghe, L. and Marks, G. 2009. A Postfunctionalist Theory of European Integration: From Permissive Consensus to Constraining Dissensus. British Journal of Political Science 39(1), pp.1-23.
	\item Pierson P. 1996., The path to European Integration: An Historical Institutionalist Perspective. Comparative Political Studies. 29(2), pp.123-63.
	\item Tsebelis, G. and Garrett, G. 2001., The Institutional Foundations of Intergovernmentalism and Supranationalism. International Organization 55(2), pp.357-90.
	\item Bickerton, C.J., Hodson, D. and Puetter, U. 2014. The new intergovernmentalism: European integration in the post-Maastricht era. Journal of Common Market Studies 53(4), pp.703–22.
	\item Schimmelfennig, F. 2015a. What’s the news in `new intergovernmentalism'? A critique of Bickerton, Hodson and Puetter. JCMS: Journal of Common Market Studies 53(4), pp.723–30.
	\item Schimmelfennig, F. 2015. Liberal intergovernmentalism and the euro area crisis. Journal of European Public Policy, 22(2), pp.177-195.
	\item Puetter, U. 2016. The centrality of consensus and deliberation in contemporary EU politics and the new intergovernmentalism. Journal of European Integration, 38(5), pp.601-615.
	\item Puetter, U. and Fabbrini, S. 2016. Catalysts of integration–the role of core intergovernmental forums in EU politics. Journal of European Integration, 38(5), pp.633-642.
	\item Kleine, M., and Pollack, M. 2018. Liberal Intergovernmentalism and Its Critics. JCMS: Journal of Common Market Studies, 56, pp.1493–1509.
	\item Moravcsik, A. 2018. Preferences, Power and Institutions in 21st-century Europe. JCMS: Journal of Common Market Studies, 56, pp.1648-1674.
	\item Meunier, S., and Vachudova, M. A. 2018. Liberal Intergovernmentalism, Illiberalism and the Potential Superpower of the European Union. JCMS: Journal of Common Market Studies, 56, pp.1631–1647.
	\item Hix, S. 2018. When Optimism Fails: Liberal Intergovernmentalism and Citizen Representation. JCMS: Journal of Common Market Studies, 56, pp.1595–1613.
	\item Schimmelfennig, F. 2018. Liberal Intergovernmentalism and the Crises of the European Union. JCMS: Journal of Common Market Studies, 56, pp.1578–1594.
	\item Schimmelfennig, F. 2018. European integration (theory) in times of crisis. A comparison of the Euro and Schengen crises, Journal of European Public Policy, 257), pp.969-989
	\item Schmidt, V. A. 2018. Rethinking EU Governance: From Old to New Approaches to Who Steers Integration. JCMS: Journal of Common Market Studies, 56, pp.1544–1561.
	\item Naurin, D. 2018. Liberal Intergovernmentalism in the Councils of the EU: A Baseline Theory?. JCMS: Journal of Common Market Studies, 56, pp.1526–1543.
	\item McNamara, K. R. 2018. Authority Under Construction: The European Union in Comparative Political Perspective. JCMS: Journal of Common Market Studies, 56, pp.1510–1525. 
	\item Jones, E. 2018. Towards a theory of disintegration, Journal of European Public Policy, 25(3), pp.440-451.
\end{itemize}


%\subsubsection*{Week 3}

%%!TEX root = INRL20160_Syllabus_1920.tex

\subsubsection*{Theories of European integration}

Study questions

\begin{itemize}
	\item Describe the main logic behind neofunctionalism as a theory of European integration. Can this theory successfully explain how we got to where we are now in terms of EU integration?
	\item Describe the main logic behind intergovernmentalism. Does this theory do a better job of explaining the European integration project than neofunctionalism?
	\item What are the strengths and weaknesses of both theories of integration?
	\item What are the three different types of institutionalism? How do they differ from one another?
	\item How do institutionalist theories differ from the `grand' theories of integration that went before?
	\item How successful are institutionalist theories at explaining EU politics?
\end{itemize}

\noindent Key reading

\begin{itemize}
	\item Cini, M. and Pérez-Solórzano Borragán, N. (eds.) 2019. European Union Politics, Oxford: Oxford University Press. Chapters 54-5
	\item Moravcsik, A. 1993. Preferences and Power in the European Community. Journal of Common Market Studies 31(4), pp.473-524.
	\item Aspinwall, M. and Schneider, G. 2000. Same Menu, Separate Tables: The Institutionalist Turn in Political Science and the Study of European Integration. European Journal of Political Research 38, pp.1-36.
\end{itemize}


\noindent Further reading

\begin{itemize}
	\item Hoffmann, S. 1966. Obstinate or obsolete? The fate of the nation-state and the case of Western Europe. Daedalus, pp.862-915.
	\item Haas, E.B. 1976. Turbulent Fields and the Theory of Regional Integration. International Organization 30, pp.173-212.
	\item Monnet, J. 1978. Memoirs. New York: Doubleday.
	\item Bache, I. and George, S. 2011. Politics in the European Union. Oxford: Oxford University Press, Chapter 1.
	\item Haas, E.B. 1958. The Uniting of Europe. Stanford: Stanford University Press.
	\item Haas, E.B. 1961. International Integration: The European and Universal Process. International Organization 15(3), pp.366-92.
	\item Haas, E.B. 1975. The Obsolescence of Regional Integration Theory. Berkeley: Institute of International Studies.
	\item Hooghe, L and Marks G. 2001. Multi-Level Governance and European Integration. Oxford: Rowman and Littlefield.
	\item Jupille, J. and Caporaso, J.A. 1999. Institutionalism and the European Union: Beyond Comparative Politics and International Relations. Annual Review Political Science 2, pp.429-44.
	\item Moravcsik, A. 1998. The Choice For Europe: Social Purpose And State Power From Messina To Maastricht. Cornell University Press, Chapter 1, 7.
	\item Nugent, N. 2010. The Government and Politics of the European Union. London: Palgrave Macmillan, Chapter 23.
	\item Pollack, M.A. 2003. The Engines of European Integration: Delegation, Agency and Agenda Setting in the EU. Oxford: Oxford University Press.
	\item Pollock, M.A. 2005. Theorizing the EU: International Organization, Domestic Polity, or Experiment in New Governance? Annual Review Political Science. 8, pp.357–98.
	\item Rosamond, B. 2000. Theories of European Integration London: MacMillan, Chapters 2 and 3. 
	\item Sandholtz, W. And Sweet Stone, A. 1998. European Integration and Supranational Governance. Oxford: Oxford University Press.
	\item Weiner, A. and Diez, T. 2009. European Integration Theory. Oxford: Oxford University Press (2nd Edition).
	\item Hooghe, L. and Marks, G. 2009. A Postfunctionalist Theory of European Integration: From Permissive Consensus to Constraining Dissensus. British Journal of Political Science 39(1), pp.1-23.
	\item Pierson P. 1996., The path to European Integration: An Historical Institutionalist Perspective. Comparative Political Studies. 29(2), pp.123-63.
	\item Tsebelis, G. and Garrett, G. 2001., The Institutional Foundations of Intergovernmentalism and Supranationalism. International Organization 55(2), pp.357-90.
	\item Bickerton, C.J., Hodson, D. and Puetter, U. 2014. The new intergovernmentalism: European integration in the post-Maastricht era. Journal of Common Market Studies 53(4), pp.703–22.
	\item Schimmelfennig, F. 2015a. What’s the news in `new intergovernmentalism'? A critique of Bickerton, Hodson and Puetter. JCMS: Journal of Common Market Studies 53(4), pp.723–30.
	\item Schimmelfennig, F. 2015. Liberal intergovernmentalism and the euro area crisis. Journal of European Public Policy, 22(2), pp.177-195.
	\item Puetter, U. 2016. The centrality of consensus and deliberation in contemporary EU politics and the new intergovernmentalism. Journal of European Integration, 38(5), pp.601-615.
	\item Puetter, U. and Fabbrini, S. 2016. Catalysts of integration–the role of core intergovernmental forums in EU politics. Journal of European Integration, 38(5), pp.633-642.
	\item Kleine, M., and Pollack, M. 2018. Liberal Intergovernmentalism and Its Critics. JCMS: Journal of Common Market Studies, 56, pp.1493–1509.
	\item Moravcsik, A. 2018. Preferences, Power and Institutions in 21st-century Europe. JCMS: Journal of Common Market Studies, 56, pp.1648-1674.
	\item Meunier, S., and Vachudova, M. A. 2018. Liberal Intergovernmentalism, Illiberalism and the Potential Superpower of the European Union. JCMS: Journal of Common Market Studies, 56, pp.1631–1647.
	\item Hix, S. 2018. When Optimism Fails: Liberal Intergovernmentalism and Citizen Representation. JCMS: Journal of Common Market Studies, 56, pp.1595–1613.
	\item Schimmelfennig, F. 2018. Liberal Intergovernmentalism and the Crises of the European Union. JCMS: Journal of Common Market Studies, 56, pp.1578–1594.
	\item Schimmelfennig, F. 2018. European integration (theory) in times of crisis. A comparison of the Euro and Schengen crises, Journal of European Public Policy, 257), pp.969-989
	\item Schmidt, V. A. 2018. Rethinking EU Governance: From Old to New Approaches to Who Steers Integration. JCMS: Journal of Common Market Studies, 56, pp.1544–1561.
	\item Naurin, D. 2018. Liberal Intergovernmentalism in the Councils of the EU: A Baseline Theory?. JCMS: Journal of Common Market Studies, 56, pp.1526–1543.
	\item McNamara, K. R. 2018. Authority Under Construction: The European Union in Comparative Political Perspective. JCMS: Journal of Common Market Studies, 56, pp.1510–1525. 
	\item Jones, E. 2018. Towards a theory of disintegration, Journal of European Public Policy, 25(3), pp.440-451.
\end{itemize}


\subsubsection*{Week 4}

\subsubsection*{Executive politics and writing workshops}

Study questions

\begin{itemize}
	\item How is the European Council structured? What strengths and weaknesses arise from this structure in terms of its ability to influence the direction of the EU?
	\item How is the European Commission structured? How does it fit into the institutional structure of the EU? 
	\item Does the institutional structure of the Commission help or hinder its ability to make policy?
	\item What roles do the European Council and Commission play in the political system of the EU? How do their roles differ?
	\item Explain principle-agent theory. Does this theory successfully describe the sources of Commission power and influence in EU politics?
\end{itemize}

\noindent Key reading

\begin{itemize}
	\item Cini, M. and Borragán, N.P.S. 2019. European Union Politics. Oxford University Press. Chapters 10, Chapter 11 (part about European Council)
	\item Hix, S. and H\o yland, B. 2011. The political system of the European Union. Palgrave Macmillan. Chapter 2. 
	\item Pollack, M A. 1997. Delegation, Agency, and Agenda Setting in the European Community. International Organization 51(1), pp.99–134.
\end{itemize}

\noindent Further reading

\begin{itemize}
	\item Rauh, C. 2019. EU politicization and policy initiatives of the European Commission: the case of consumer policy, Journal of European Public Policy, 26(3), pp.344-365
	\item Delreux, T. and Adriaensen, J. eds., 2017. The Principal Agent Model and the European Union. Basingstoke: Palgrave Macmillan.
	\item Delreux, T. and Adriaensen, J. 2017. Twenty years of principal-agent research in EU politics: how to cope with complexity?. European Political Science, pp.1-18.
	\item Concei\c{c}\~{a}o-Heldt, E. 2010. Who controls whom? Dynamics of power delegation and agency losses in EU trade politics. JCMS: Journal of Common Market Studies, 48(4), pp.1107-1126.
	\item Nugent, N. and Rhinard, M. 2016. Is the European Commission Really in Decline?. JCMS: Journal of Common Market Studies. 54(5), pp.1199-1215.
	\item Brown, S.A. 2016. The Commission and the Crisis: Chief Loser or Unexpected Winner?. In The European Commission and Europe's Democratic Process, pp.69-78. Palgrave Macmillan UK.
	\item Bauer, M.W. and Becker S. 2014. The unexpected winner of the crisis: the European Commission’s strengthened role in economic governance. Journal of European Integration. 36(3), pp.213–29.
	\item da Concei\c{c}\~{a}o-Heldt, E. 2016. Why the European Commission is not the ``unexpected winner" of the Euro crisis: A comment on Bauer and Becker. Journal of European Integration, 38(1), pp.95-100.
	\item Bauer, M.W. and Becker, S. 2016. Absolute Gains Are Still Gains: Why the European Commission Is a Winner of the Crisis, and Unexpectedly So. A Rejoinder to Eugénia da Concei\c{c}\~{a}o-Heldt. Journal of European Integration, 38(1), pp.101-106.
	\item Fabbrini, S. and Puetter, U. 2016. Integration without supranationalisation: studying the lead roles of the European Council and the Council in post-Lisbon EU politics. Journal of European Integration, 38(5), pp.481-495.
	\item Bailer, S. 2014. An Agent Dependent on the EU Member States? The Determinants of the European Commission’s Legislative Success in the European Union, Journal of European Integration, 36(1), pp.37-53.
	\item Egeberg, M. 2006. Executive Politics as Usual: Role Behaviour and Conflict Dimensions in the College of European Commissioners. Journal of European Public Policy 13(1), pp.1-15.
	\item Franchino, F. 2009. Experience and the Distribution of Portfolio Payoffs in the European Commission. European Journal of Political Research 48(1), pp.1–30.
	\item Franchino, F. 1999. Delegating Powers in the European Community. British Journal of Political Science 34(2), pp.269–93.
	\item Dinan, D. 2010. Ever Closer Union: An Introduction to European Integration. London: Palgrave Macmillan, Chapters 7-8.
	\item Bunse, S. 2009. Small States and EU Governance: Leadership through the Council Presidency. Basingstoke: Palgrave Macmillan.
	\item Cini, M. 1996. The European Commission: Leadership, Organisation and Culture in the EU Administration. Manchester: Manchester University Press.
	\item Franchino, F. 2009. Experience and the distribution of portfolio payoffs in the European Commission. European Journal of Political Research 48(1), pp.1-30.
	\item H{\"a}ge, F.M. 2008. Who Decides in the Council of the European Union? Journal of Common Market Studies 46(3), pp.533-58.
	\item H{\"a}ge, F.M. 2016. Political attention in the Council of the European Union: A new dataset of working party meetings, 1995–2014. European Union Politics 17(4), pp.683–703
	\item Hayes-Renshaw, F. and Wallace H. 2006. The Council of Ministers. Basingstoke: Palgrave Macmillan.
	\item Hooghe, L. 1999. Images of Europe: Orientations to European Integration Among Senior Officials of the Commission. British Journal of Political Science 29:345–367.
	\item Hooghe, L. 2002. The European Commission and the Integration of Europe. New York: Cambridge University Press. 
	\item Hooghe, L. 2005. Several Roads Lead to International Norms, But Few via International Socialization: A Case Study of the European Commission. International Organization 59: 861-898.
	\item Rasmussen, A. 2007. Challenging the Commission’s Right of Initiative? Conditions for Institutional Change and Stability. West European Politics 30(2), pp.244-64.
	\item Tallberg, J. 2008. Bargaining Power in the European Council. Journal of Common Market Studies 46(3), pp.685-708.
	\item Thomson, R. 2008. National Actors in International Organizations: The Case of the European Commission. Comparative Political Studies 41: 169-92.
	\item Tsebelis, G. and Garrett G. 2000. Legislative Politics in the EU. European Union Politics 1, pp.9–36.
	\item Wallace, H., Pollock, M, and Young, A.R. 2010. Policy-Making in the European Union. Oxford: Oxford University Press, Chapter 4.
	\item Wonka, A. 2007. Technocratic and Independent? The Appointment of European Commissioners and its Policy Implications. Journal of European Public Policy 14(2), pp.169-89.
	\item Slapin, J.B. 2006. Who Is Powerful?: Examining Preferences and Testing Sources of Bargaining Strength at European Intergovernmental Conferences. European Union Politics 7 (1), pp.51–76.
	\item Hug, S, and K{\"o}nig, T. 2002. In View of Ratification: Governmental Preferences and Domestic Constraints at the Amsterdam Intergovernmental Conference. International Organization 56 (02), pp.447–76.
\end{itemize}

\subsubsection*{Week 5}

\subsubsection*{Legislative politics}

Study questions

\begin{itemize}
	\item Describe the different legislative procedures used in the EU. How do they differ from one another? 
	\item Describe the internal structure of the Council of Ministers. How are legislative powers and influence distributed between 1) the member states in terms of voting power? and 2) between the different levels of negotiation within the Council?
	\item Describe the internal structure of the European Parliament. What are the main dimensions of political conflict in the EP? How stable are coalitions between the EP party groups?
\end{itemize}

\noindent Key reading

\begin{itemize}
	\item Cini, M. and Borragán, N.P.S. 2019. European Union Politics. Oxford University Press. Chapter 11 (part about Council of Ministers), Chapter 12, 16
	\item Hix, S. and	Hoyland B. 2011. The political system of the European Union. Palgrave, Chapter 3. 
	\item Hix, S., Noury, A. and Roland G. 2009. Voting Patterns and Alliance Formation in the European Parliament. Philosophical Transactions of the Royal Society B, 364: 821-831.
	\item Cross, J.P. and Hermansson, H. 2017. Legislative amendments and informal politics in the European Union: A text reuse approach. European Union Politics, 18(4), pp.581-602.
\end{itemize}

\noindent Further reading

\begin{itemize}
	\item Wratil, C. 2018. Modes of government responsiveness in the European Union: Evidence from Council negotiation positions. European Union Politics, 19(1), pp.52–74.
	\item Thomson, R. and Hosli, M. 2006. Who Has Power in the EU? Journal of Common Market Studies 44(2), pp.391-417.
	\item Huhe, N., Naurin, D., and Thomson, R. 2018. The evolution of political networks: Evidence from the Council of the European Union. European Union Politics, 19(1), pp.25–51.
	\item Rasmussen, A. and Reh, C. 2013. The consequences of concluding codecision early: trilogues and intra-institutional bargaining success. Journal of European Public Policy, 20(7), pp.1006-1024.
	\item Rasmussen, A. 2011. Early conclusion in bicameral bargaining: Evidence from the co-decision legislative procedure of the European Union. European Union Politics, 12(1), pp.41-64.
	\item Rasmussen, A. 2008. The EU Conciliation Committee One or Several Principals?. European Union Politics, 9(1), pp.87-113.
	\item Rasmussen, A. 2012. Twenty Years of Co-decision Since Maastricht: Inter-and Intrainstitutional Implications. Journal of European Integration, 34(7), pp.735-751.
	\item Reh, C., Héritier, A., Bressanelli, E. and Koop, C. 2013. The informal politics of legislation explaining secluded decision making in the European Union. Comparative Political Studies, 46(9), pp.1112-1142.
	\item M{\"u}hlb{\"o}ck, M., \& Rittberger, B. 2015. The Council, the European Parliament, and the paradox of inter-institutional cooperation. European Integration online Papers (EIoP), 19.
	\item Brandsma, G. J. 2015. Co-decision after Lisbon: The politics of informal trilogues in European Union lawmaking. European Union Politics, 16(2), pp.300-319
	\item Dinan, D. 2010. Ever Closer Union: An Introduction to European Integration London: Palgrave Macmillan. Chapter 9.
	\item Corbett, R., Jacobs, F. and Shackleton, M. 2005. The European Parliament. London: John Harper 
	\item Costello, R. and Thomson, R. 2010. The Policy Impact of Leadership in Committees: Rapporteurs’ Influence on the European Parliament’s Opinions. European Union Politics 11(1), pp.1-26.
	\item Cross, J.P. 2013. Everyone’s a winner (almost): Bargaining success in the Council of Ministers of the European Union. European Union Politics, 14(1), pp.70–94.
	\item Cross, J.P. 2012. Interventions and negotiation in the Council of Ministers of the European Union. European Union Politics, 13(1), pp.47–69.
	\item Hayes-Renshaw, F., Van Aken, W. and Wallace, H. 2006. When and Why the EU Council of Ministers Votes Explicitly. Journal of Common Market Studies 44: 161-94.
	\item Hix, S., Noury, A. and Roland, G. 2007. Democratic Politics in the European Parliament. Cambridge: Cambridge University Press, Chapters 1, 5, 6, 9. 
	\item Judge, D. and Earnshaw, D. 2008. The European Parliament. London: Palgrave. (2nd ed.).
	\item K{\"o}nig, T, Lindburg, B, Lechner, S. and Pohlmeier, W. 2007. Bicameral Conflict Resolution in the European Union: An Empirical Analysis of Conciliation Committee Bargains. British Journal of Political Science 37, pp.281-312.
	\item Kreppel, A. 2002. The European Parliament and Supranational Party System: A Study in Institutional Development. Cambridge: Cambridge University Press. 
	\item Mattila, M. 2009. Roll Call Analysis of Voting in the EU Council of Ministers after the 2004 Enlargement. European Journal of Political Research 48, pp.840-57.
	\item McElroy, G. and Benoit, K. 2007. Party Groups and Policy Positions in the European Parliament. Party Politics 13, pp.5-28.
	\item McElroy, G. and Benoit, K. 2010. Party Policy and Group Affiliation in the European Parliament. British Journal of Political Science 40(2), pp.377-98.
	\item Naurin, D. and Lindahl, R. 2010. Out in the Cold? Flexible Integration and the Political Status of Euro Opt-Outs. European Union Politics 11(4), pp.485-509.
	\item Thomson, R. 2008. The Relative Power of Member States in the Council: Large and Small, Old and New. In Naurin, D. and Wallace, H. (eds.) Unveiling the Council of the European Union: Games Governments Play in Brussels. London: Palgrave Macmillan, pp.238-60.
	\item Thomson, R. 2009. Actor Alignments in the European Union Before and After Enlargement. European Journal of Political Research 48, pp.756-81.
	\item Ringe, N. 2005. Policy Preference Formation in Legislative Politics: Structures, Actors, and Focal Points. American Journal of Political Science 49(4), pp.731–746.
\end{itemize}

\subsubsection*{Week 6}

\subsubsection*{Judicial politics and compliance}

Study questions

\begin{itemize}
	\item Describe the structure of the European Court of Justice and the different procedures the ECJ has at its disposal to enforce EU law?
	\item What role does the European Court of Justice play in the political system of the EU? What powers does it yield and what are the limits of these powers?
	\item What do we mean by the term `judicial activism'? Does the ECJ engage in judicial activism often? How have member states reacted to examples of judicial activism by the ECJ?
	\item What difficulties might arise when Member States transpose EU laws into national laws?
	\item We observe significant variation in the levels of compliance with EU law both between Member States and over time. What can explain this variation?
	\item Why might member states fail to comply with EU legislation? What incentives are involved in compliance and non-compliance?
\end{itemize}

\noindent Key reading
	
\begin{itemize}
	\item Cini, M. and Borragán, N.P.S. 2019. European Union Politics. Oxford University Press. Chapter 13.
	\item Hix, S. and H\o yland, B. 2011. The political system of the European Union. Palgrave Macmillan. Chapter 4.
	\item Angelova, M., Dannwolf, T. and K{\"o}nig, T. 2012. How Robust are Compliance Findings. Journal of European Public Policy 19(8), pp.1269-91.
\end{itemize}

\noindent Further reading
	
\begin{itemize}
	\item Closa, C. 2018. The politics of guarding the Treaties: Commission scrutiny of rule of law compliance, Journal of European Public Policy
	\item Jaremba, U. and Mayoral, J.A. 2019. The Europeanization of national judiciaries: definitions, indicators and mechanisms, Journal of European Public Policy, 26(3), pp.386-406
	\item Peritz, L. 2018. Obstructing integration: Domestic politics and the European Court of Justice. European Union Politics, 19(3), pp.427–457.
	\item Hartlapp, M. 2018. Power Shifts via the Judicial Arena: How Annulments Cases between EU Institutions Shape Competence Allocation. JCMS: Journal of Common Market Studies, 56: pp.1429–1445.
	\item Kleine, M. 2014. Informal Governance in the European Union. Journal of European Public Policy, 21(2), pp.303-314.
	\item Zhelyazkova, A. and Yordanova, N. 2015. Signalling compliance: The link between notified EU directive implementation and infringement cases. European Union Politics 16(3), pp.408-428
	\item Carrubba, C.J., Gabel, M. and Hankla, C. 2008. Judicial behavior under political constraints: Evidence from the European Court of Justice. American Political Science Review, 102(04), pp.435-452. 
	\item Sweet, A.S. and Brunell, T. 2012. The European Court of Justice, state noncompliance, and the politics of override. American Political Science Review, 106(01), pp.204-213.
	\item Alter, K.J. 2000. The European Union's legal system and domestic policy: spillover or backlash?. International Organization, 54(03), pp.489-518.
	\item Bache, I., George, S. and Bulmer, S. 2011. Politics in the European Union Oxford: Oxford University Press, Chapter 23	
	\item Chalmers, D., Davies, G. and Monti, G. 2010. European Union Law. Cambridge University Press, Chapter 4.
	\item Garrett, G. 1995. The Politics of Legal Integration in the European Union. International Organization 49(1), pp.171-81.
	\item Luetgert, B. and Dannwolf, T. 2009. Mixing Method: A Nested Analysis of EU Member State Transposition Patterns. European Union Politics 10/3, pp.307-34.
	\item Thomson, R. 2010. Opposition through the Back Door in the Transposition of EU Directives. European Union Politics 12, pp.337-57.
	\item Thomson, R., Torenvlied, R., and Arregui, J. 2007. The Paradox of Compliance: Infringements and Delays in Transposing European Union Directives. British Journal of Political Science 37, pp.685-709.
	\item K{\"o}nig, T., and M{\"a}der, L. 2013. Non-conformable, partial and conformable transposition: A competing risk analysis of the transposition process of directives in the EU15. European Union Politics, 14(1), pp.46-69.
	\item K{\"o}nig, T. and M{\"a}der, L. 2014. The Strategic Nature of Compliance: An Empirical Evaluation of Law Implementation in the Central Monitoring System of the European Union. American Journal of Political Science, 58, pp.246–263. 
	\item Steunenberg, B. 2006. Turning Swift Policy-making into Deadlock and Delay National Policy Coordination and the Transposition of EU Directives. European Union Politics, 7(3), pp.293-319.
\end{itemize}


\subsubsection*{Week 7}

\subsubsection*{Elections, referenda, and public opinion}

Study questions

\begin{itemize}
	\item How has public opinion towards the EU changed over time? What explains this variation?
	\item What different theories have attempted to explain public attitudes towards the EU? Which of these theories is most convincing and why?
	\item What do we mean when we describe the EP elections as second order in nature? Is this a fair description of EP elections at this point in the history of the EU?
\end{itemize}
	
\noindent Key reading
	
\begin{itemize}
	\item Explore EUOPINIONS website: \url{http://eupinions.eu/de/home/}
	\item Cini, M. and Borragán, N.P.S. 2019. European Union Politics. Oxford University Press. Chapter 15
	\item Hix, S. and H\o yland, B. 2011. The Political System of the European Union. Palgrave Macmillan (3rd edition), Chapters 5-6.
	\item Boomgaarden, H.G., Schuck, A.R., Elenbaas, M. and De Vreese, C.H. 2011. Mapping EU attitudes: Conceptual and empirical dimensions of Euroscepticism and EU support. European Union Politics, 12(2), pp.241-266.
	\item Beach, D., Hansen, K., and Larsen, M. 2018. How Campaigns Enhance European Issues Voting During European Parliament Elections. Political Science Research and Methods, 6(4), pp.791-808.
\end{itemize}

\noindent Further reading
	
\begin{itemize}
	\item Aldrich, A. S. 2018. National Political Parties and Career Paths to the European Parliament. JCMS: Journal of Common Market Studies, 56: pp.1283–1304.
	\item Hix, S. and Marsh, M., 2007. Punishment or Protest? Understanding European Parliament Elections. Journal of Politics 69(2), pp.495-510.
	\item Vasilopoulou, S. and Talving, L. 2018. Opportunity or threat? Public attitudes towards EU freedom of movement. Journal of European Public Policy, pp.1-19.
	\item Rooduijn, M., 2015. The rise of the populist radical right in Western Europe. European View, 14(1), pp.3-11.
	\item Jeannet, A.M., 2018. Internal migration and public opinion about the European Union: a time series cross-sectional study. Socio-Economic Review.
	\item Kostadinova, P. and Giurcanu, M. 2018. Capturing the legislative priorities of transnational Europarties and the European Commission: A pledge approach, European Union Politics, 19(2), pp.363–379.
	\item Senninger, R. and Bischof, D. 2018. Working in unison: Political parties and policy issue transfer in the multilevel space, European Union Politics, 19(1), pp.140–162.
	\item Sorace, M. 2018. Legislative Participation in the EU: An analysis of questions, speeches, motions and declarations in the 7th European Parliament. European Union Politics, 19(2), pp.299–320.
	\item Frid-Nielsen, S. S. 2018. Human rights or security? Positions on asylum in European Parliament speeches. European Union Politics, 19(2), pp.344–362.
	\item Cavallaro, M., Flacher, D., and Zanetti, M. A. 2018. Radical right parties and European economic integration: Evidence from the seventh European Parliament. European Union Politics, 19(2), pp.321–343.
	\item Tuttnauer, O., 2018. If you can beat them, confront them: Party-level analysis of opposition behavior in European national parliaments. European Union Politics, 19(2), pp.278-298.
	\item Hernández, E. 2018. Democratic discontent and support for mainstream and challenger parties: Democratic protest voting. European Union Politics, 19(3) pp.458–480
	\item Schulte-Cloos, J. 2018. Do European Parliament elections foster challenger parties' success on the national level?. European Union Politics, p.1465116518773486.
	\item Anderson, C.J. and Hecht, J.D. 2018. The preference for Europe: Public opinion about European integration since 1952. European Union Politics, 19(4), pp.617-638.
	\item Beach, D., Hansen, K., and Larsen, M. 2018. How Campaigns Enhance European Issues Voting During European Parliament Elections. Political Science Research and Methods, 6(4), 791-808. 
	\item Lefkofridi, Z. and Katsanidou, A. 2018. A Step Closer to a Transnational Party System? Competition and Coherence in the 2009 and 2014 European Parliament. JCMS: Journal of Common Market Studies, 56: 1462–1482.
	\item Sorace, M. 2017. The European Union democratic deficit: Substantive representation in the European Parliament at the input stage. European Union Politics, p.1465116517741562.
	\item Nielsen, J.H. and Franklin, M.N. 2017. The 2014 European Parliament Elections: Still Second Order?. In The Eurosceptic 2014 European Parliament Elections (pp.1-16). Palgrave Macmillan UK.
	\item McEvoy, C. 2016. The Role of Political Efficacy on Public Opinion in the European Union. JCMS: Journal of Common Market Studies, 54(5), pp.1159-1174
	\item Hobolt, S.B. and de Vries, C.E. 2016. Public Support for European Integration. Annual Review of Political Science, 19, pp.413-432.
	\item Hobolt, S.B. 2016. The Brexit vote: a divided nation, a divided continent. Journal of European Public Policy, 23(9), pp.1259-1277.
	\item Braun, D., Hutter, S. and Kerscher, A. 2016. What type of Europe? The salience of polity and policy issues in European Parliament elections. European Union Politics, 17(4), pp.570-592.
	\item Baglioni, S. and Hurrelmann, A. 2016. The Eurozone crisis and citizen engagement in EU affairs. West European Politics, 39(1), pp.104-124.
	\item Oliver, T. 2016. European and international views of Brexit. Journal of European Public Policy, pp.1-8.
	\item Nielsen, J.H. 2016. Personality and Euroscepticism: The Impact of Personality on Attitudes Towards the EU. JCMS: Journal of Common Market Studies. 54(5), pp.1175–1198.
	\item Smith, K.E. 2016. Left out in the cold: Brexit, the EU and the perils of Trump’s world. LSE Brexit. \url{http://eprints.lse.ac.uk/68777/}.
	\item Hix, S., Hagemann, S. and Frantescu, D. 2016. Would Brexit matter? The UK’s voting record in the Council and the European Parliament. \url{http://eprints.lse.ac.uk/66261/}.
	\item Chopin, T. and Lequesne, C. 2016. Differentiation as a double-edged sword: member states’ practices and Brexit. International Affairs, 92(3), pp.531-545.
	\item Dagnis Jensen, M. and Snaith, H. 2016. When politics prevails: the political economy of a Brexit. Journal of European Public Policy, pp.1-9.
	\item Hobolt, S.B. and de Vries, C.E. 2016. Public Support for European Integration. Annual Review of Political Science, 19, pp.413-432.
	\item Hobolt, S. and Spoon, J. 2012. Motivating the European voter: parties, issues and campaigns in European Parliament elections. European Journal of Political Research 51(6), pp.701-727.
	\item Hobolt, Sara B. 2012. Citizen Satisfaction with Democracy in the European Union*. JCMS: Journal of Common Market Studies 50 (s1), pp.88–105.
	\item Armingeon, K, and B Ceka. 2014. The Loss of Trust in the European Union During the Great Recession Since 2007: the Role of Heuristics From the National Political System. European Union Politics 15 (1), pp.82–107.
	\item Ceka, B. 2013. The Perils of Political Competition: Explaining Participation and Trust in Political Parties in Eastern Europe. Comparative Political Studies 46 (12), pp.1610–35.
	\item Lefkofridi, Z. and Katsanidou, A. 2014. Multilevel representation in the European Parliament. European Union Politics, 15(1), pp.108-131.
	\item Kopecky, P. and Mudde, C. 2002. The Two Sides of Euroscepticism: Party Positions on European Integration in East Central Europe. European Union Politics, 3(3), pp.297–326.
	\item Franklin, M. and Hobolt, S.B. 2011. The Legacy of Lethargy: How Elections for the European Parliament Depress Turnout. Electoral Studies 30, pp.67-76.
	\item Gabel, M. 1998. Public Support for European Integration: An Empirical Test of Five Theories. Journal of Politics 60(3), pp.333–354.
	\item Garry, J. and Tilley, J. 2009. The Macroeconomic Factors Conditioning the Impact of Identity on Attitudes towards the EU. European Union Politics 10(3), pp.361–379.
	\item Hellstr{\"o}m, J. 2008. Who Leads, Who Follows? Re-examining the Party-Electorate Linkages on European Integration. Journal of European Public Policy 15(8), pp.1127-44.
	\item Hix, S. and Marsh, M. 2011. Second-order effects plus pan-European political swings: An analysis of European Parliament elections across time. Electoral Studies 30, pp.4–15.
	\item Hobolt, S.B. 2009. Europe in Question: Referendums on European Integration. Oxford University Press.
	\item Hobolt, S.B. 2012. Citizens satisfaction with democracy in the European Union. Journal of Common Market Studies 50 (1), pp.88-105.
	\item Hobolt, S.B., Spoon, J. and Tilley, J. 2009. A Vote Against Europe? Explaining Defection at the 1999 and 2004 European Parliament Elections. British Journal of Political Science 39(1), pp.93-115.
	\item Hobolt, S.B. and Wittrock, J. 2011. The second-order election model revisited: An experimental test of vote choices in European Parliament elections. Electoral Studies 30, pp.29–40.
	\item Hooghe, L. and Marks, G. 2006. Calculation, Community and Cues: Public Opinion on European Integration. European Union Politics 6(4), pp.419–43.
	\item Steenbergen, M.R., Edwards, E.E. and de Vries, C.E. 2007. Who’s Cueing Whom? Mass- Elite Linkages and the Future of European Integration. European Union Politics 8(1), pp.39-49.
\end{itemize}

\subsubsection*{Week 8}

\subsubsection*{Brexit}

Study questions

\begin{itemize}
	\item What factors explain the outcome of the Brexit referendum vote?
	\item What factors explain the outcome of the Brexit negotiations between the UK and the EU?
	\item How has Ireland, a much small country than the UK, managed to have its views enshrined in the withdrawal agreement?
	\item How is Brexit (if it goes ahead) likely to affect EU decision-making?
\end{itemize}

\noindent Required Readings

\begin{itemize}
	\item Cini, M. and Borragán, N.P.S. 2019. European Union Politics. Oxford University Press. Chapter 27.
	\item Hobolt, S. B. 2016. The Brexit vote: a divided nation, a divided continent. Journal of European Public Policy, 23(9), pp.1259-1277.
	\item Kirsch, W. 2016. Brexit and the Distribution of Power in the Council of the EU, CEPS Commentaries, pp.1-4. 
	\item Springford, J. 2018. Theresa May's Irish trilemma. \url{https://www.cer.eu/insights/theresa-mays-irish-trilemma}.
	\item Jones, E. 2018. Four Things We Should Learn from Brexit, Survival, 60:6, pp.35-44.
\end{itemize}

\noindent Further reading

\begin{itemize}
	\item Patel, O. \& Reh, C. 2016. Brexit: The Consequences for the EU Political System, UCL Constitution Unit Briefing Paper, pp.1-5. 
	\item Staal, K. 2016. Brexit Implications for Influence on EU Decision Making, The Stability of Regions, Culture, and Institutions VIVES Workshop, University of Leuven, Belgium, pp.1-6.
	\item Inglehart, R., \& Norris, P. 2016. Trump, Brexit, and the rise of populism: Economic have-nots and cultural backlash. HKS Working Paper No. RWP16-026. Available at SSRN: \url{https://ssrn.com/abstract=2818659}.
	\item European Commission. 2019. Withdrawal of the United Kingdom from the EU. Available here: \url{https://ec.europa.eu/taxation_customs/uk_withdrawal_en}
	\item Lavery, S., McDaniel, S. and Schmid, D., 2018. Finance fragmented? Frankfurt and Paris as European financial centres after Brexit. Journal of European Public Policy, pp.1-19.
	\item Jennings, W. and Lodge, M. 2018. Brexit, the tides and Canute: the fracturing politics of the British state, Journal of European Public Policy
	\item Qvortrup, M. 2016. Referendums on Membership and European Integration 1972–2015. The Political Quarterly, 87, pp.61-68.
	\item Henderson, A. , Jeffery, C. , Liñeira, R. , Scully, R. , Wincott, D. and Wyn Jones, R. 2016. England, Englishness and Brexit. The Political Quarterly, 87, pp.187-199.
	\item James, S. and Quaglia, L. 2018. The Brexit Negotiations and Financial Services: A Two-Level Game Analysis. The Political Quarterly, 89, pp.560-567.
	\item Renwick, A. , Allan, S. , Jennings, W. , Mckee, R. , Russell, M. and Smith, G. 2018. What Kind of Brexit do Voters want? Lessons from the Citizens’ Assembly on Brexit. The Political Quarterly, 89, pp.649-658.
	\item Laffan, B. 2018. Brexit: Re-opening Ireland's English Question. The Political Quarterly, 89, pp.568-575. 
	\item Chen W, Los B, McCann P, Ortega-Argilés R, Thissen M, van Oort F. 2018. The continental divide? Economic exposure to Brexit in regions and countries on both sides of The Channel. Papers in Regional Science. 97, pp.25–54.
	\item Richardson, J. 2018. Brexit: The EU Policy-Making State Hits the Populist Buffers. The Political Quarterly, 89, pp.118-126.
	\item Richards, L. , Heath, A. and Carl, N. 2018. Red Lines and Compromises: Mapping Underlying Complexities of Brexit Preferences. The Political Quarterly, 89, pp.280-290.
	\item Hantzsche, A. , Kara, A. and Young, G. 2018. The Economic Effects of the UK Government's Proposed Brexit Deal. The World Economy, Accepted Author Manuscript.
	\item Gasiorek, M. , Serwicka, I. and Smith, A. 2019. Which Manufacturing Industries and Sectors Are Most Vulnerable to Brexit?. The World Economy, Accepted Author Manuscript.
	\item Manners, I. 2018. Political Psychology of European Integration: The (Re)production of Identity and Difference in the Brexit Debate. Political Psychology, 39, pp.1213-1232.
	\item Warlouzet, L. 2018. Britain at the Centre of European Co-operation (1948–2016). JCMS: Journal of Common Market Studies, 56, pp.955–970.
	\item Dhingra, S., \& Sampson, T. 2016. Life after BREXIT: What are the UK’s options outside the European Union?. CEPBREXIT01. London School of Economics and Political Science, CEP, London, UK.
	\item Goodwin, M. J., \& Heath, O. 2016. The 2016 Referendum, Brexit and the Left Behind: An Aggregate-level Analysis of the Result. The Political Quarterly, 87(3), pp.323-332.
	\item Grant, C. 2016. The Impact of Brexit on the EU, Centre for European Reform, blog post, 24 June 2016. 
	\item Kierzenkowski, R., et al. 2016. The Economic Consequences of Brexit: A Taxing Decision, OECD Economic Policy Papers, No. 16, OECD Publishing, Paris.
	\url{http://dx.doi.org/10.1787/5jm0lsvdkf6k-en}.
	\item Kaufmann, E. 2016. It’s NOT the economy, stupid: Brexit as a story of personal values. British Politics and Policy at LSE.
	\item Whitman, R. G. 2016. Brexit or Bremain: what future for the UK's European diplomatic strategy?. International Affairs, 92(3), pp.509-529.
	\item Ottaviano, G. I. P., Pessoa, J. P., Sampson, T., \& Van Reenen, J. 2014. Brexit or Fixit? The trade and welfare effects of leaving the European Union.
	\item Oliver, T. 2016. European and international views of Brexit. Journal of European Public Policy, 23(9), pp.1321-1328.
	\item Menon, A., \& Salter, J. P. 2016. Brexit: initial reflections. International Affairs, 92(6), pp.1297-1318.
	\item Jensen, M. D., \& Snaith, H. 2016. When politics prevails: the political economy of a Brexit. Journal of European Public Policy, 23(9), pp.1302-1310.
	\item Oliver, T., \& Williams, M. J. 2016. Special relationships in flux: Brexit and the future of the US—EU and US—UK relationships. International Affairs, 92(3), pp.547-567.
\end{itemize}

\subsubsection*{Week 9}

\subsubsection*{Interest representation}

Study questions

\begin{itemize}
	\item What are interest groups and why do we study them in the EU context?
	\item Describe the different access point through which lobbyists attempt to influence EU policy making. Do the institutional structures of the Commission, EP, and Council aid or hinder lobbyists in their lobbying attempts?
	\item What different types of factors have been identified as influencing lobbying success in the EU?
	\item What attempts have been made to regulate EU lobbying? Have these attempts been successful?
\end{itemize}
	
\noindent Key reading

\begin{itemize}
	\item Cini, M. and Borragán, N.P.S. 2019. European Union Politics. Oxford University Press. Chapter 14.
	\item Hix, S. and H\o yland, B. 2011. The political system of the European Union. Palgrave Macmillan. Chapter 7.
	\item Bouwen, P. 2004. Exchanging access goods for access: A comparative study of business lobbying in the European Union Institutions. European Journal of Political Research, 43: 337–369 
	\item D{\"u}r, A., Bernhagen, P., and Marshall, D. 2015. Interest Group Success in the European Union When (and Why) Does Business Lose?. Comparative Political Studies, 48(8), pp.951-983.
\end{itemize}

\noindent Further reading

\begin{itemize}
	\item European Commission for Democracy Through Law (Venice Commission). 2013. Report on the Role of Extra-Institutional Actors in the Democratic System.
	\item Coen, D. 2007. Empirical and theoretical studies in EU lobbying. Journal of European Public Policy 14(3), pp.333-345.
	\item Marshall, D. 2010. Who to lobby and when: Institutional determinants of interest group strategies in European Parliament committees. European Union Politics, 11(4), pp.553-575.
	\item Bunea, A. 2018. Regulating European Union lobbying: in whose interest?. Journal of European Public Policy, pp.1-21.
	\item Fl{\"o}the, L. and Rasmussen, A., 2018. Public voices in the heavenly chorus? Group type bias and opinion representation. Journal of European Public Policy, pp.1-19.
	\item Judge, A. and Thomson, R., 2018. The responsiveness of legislative actors to stakeholders’ demands in the European Union. Journal of European Public Policy, pp.1-20.
	\item Wonka, A., De Bruycker, I., De Bièvre, D., Braun, C., and Beyers, J. 2018. Patterns of Conflict and Mobilization: Mapping Interest Group Activity in EU Legislative Policymaking. Politics and Governance, 6(3), 136-146.
	\item Hollman, M. and Murdoch, Z. 2018. Lobbying cycles in Brussels: Evidence from the rotating presidency of the Council of the European Union. European Union Politics, 19(4), pp.597-616.
	\item R\o ed, M., and Wøien Hansen, V. (2018) Explaining Participation Bias in the European Commission's Online Consultations: The Struggle for Policy Gain without too Much Pain. JCMS: Journal of Common Market Studies, 56: 1446–1461.
	\item De Bruycker, I. 2016. Power and position: Which EU party groups do lobbyists prioritize and why? Party Politics 22(4), pp.552-562.
	\item van der Graaf, A., Otjes, S. and Rasmussen, A. 2016. Weapon of the weak? The social media landscape of interest groups. European Journal of Communication, 31(2), pp.120-135.
	\item De Bruycker, I. 2016. Pressure and Expertise: Explaining the Information Supply of Interest Groups in EU Legislative Lobbying. JCMS: Journal of Common Market Studies 54(3), pp.599-616.
	\item Beyers, J., De Bruycker, I. and Baller, I. 2015. The alignment of parties and interest groups in EU legislative politics. A tale of two different worlds?. Journal of European Public Policy, 22(4), pp.534-551.
	\item Kl{\"u}ver, H., Mahoney, C. and Opper, M. 2015. Framing in context: how interest groups employ framing to lobby the European Commission. Journal of European Public Policy, 22(4), pp.481-498.
	\item Kl{\"u}ver, H. and Mahoney, C. 2015. Measuring interest group framing strategies in public policy debates. Journal of Public Policy, 35(02), pp.223-244.
	\item Eising, R., Rasch, D. and Rozbicka, P. 2015. Institutions, policies, and arguments: context and strategy in EU policy framing. Journal of European Public Policy, 22(4), pp.516-533.
	\item Bor{\"a}ng, F. and Naurin, D. 2015. `Try to see it my way!' Frame congruence between lobbyists and European Commission officials. Journal of European Public Policy 22(4), pp.499-515.
	\item Beyers, J., Bonafont, L. C., D{\"u}r, A., Eising, R., Fink-Hafner, D., Lowery, D., ... and Naurin, D. 2014. The INTEREURO project: Logic and structure. Interest Groups \& Advocacy, 3(2), 126-140.
	\item Chari, R. Murphy, G. and Hogan, J. 2007. Regulating Lobbyists: A Comparative Analysis of the USA, Canada, Germany and the European Union, The Political Quarterly, 78(3), pp.422-438.
	\item Coen, D. 1998. The European business interest and the nation state: large-firm lobbying in the European Union and member states. Journal of Public Policy, 18(01), pp.75-100.
	\item Mahoney, C. and Baumgartner, F. 2008. Converging perspectives on interest group research in Europe and America. West European Politics, 31(6), pp.1253–1273.
	\item Beyers, J. 2008. Policy Issues, Organizational Format and the Political Strategies of Interest Organizations. West European Politics 31(6), pp.1188-1211.
	\item Beyers, J. and Kerremans, B. 2004. Bureaucrats, Politicians, and Societal Interests: How Is European Policy Making Politicized?” Comparative Political Studies 37(10), pp.1119-1150.
	\item B{\"o}rzel, T. and Heard-Lauréote, K. 2009. Networks in EU Multi-level governance: Concepts and Contributions. Journal of Public Policy 29(2), pp.135-152.
	\item Coen, D and Richardson, J (eds) (2009) Lobbying the European Union, Oxford: Oxford University Press.
	\item D{\"u}r, A. 2008. Interest groups in the European Union: How Powerful are They?” West European Politics 31(6), pp.1212-1230.
	\item Eising, R. 2007. The access of business interests to EU institutions: towards elite pluralism? Journal of European Public Policy 14(3), pp.384-403.
	\item Eising, R. 2008. Interest Group in EU policymaking. Living Reviews in European Governance Vol. 3 (\url{http://europeangovernance.livingreviews.org/Articles/lreg-2008-4/}).
	\item Greenwood, J. 2007a. Interest Representation in the European Union, Hampshire: Palgrave Macmillan.
	\item Greenwood, J. 2007b. Review Article: Organized Civil Society and Democratic Legitimacy in the EU. British Journal of Political Science 37: 333-35. 
	\item Kl{\"u}ver, H. 2010. Measuring Interest Group Influence using Quantitative Text Analysis. European Union Politics 10(4), pp.535- 549.
	\item Mahoney, C. 2008. Brussels versus the Beltway. Advocacy in the United States and the European Union. Washington DC: Georgetown University Press.
	\item Mahoney, C. 2007. Lobbying Success in the United State and the European Union. Journal of Public Policy 27(2), pp.35-56.
	\item Quittkat, C. 2011. The European Commission’s Online Consultations: a Success Story? Journal of Common Market Studies 49(3), pp.653-674.
	\item Skodvin, T., Gullberg, A.T., and Aakre, S. 2010. Target-group influence and political feasibility: the case of climate policy design in Europe. Journal of European Public Policy 17(6), pp.854- 873.
\end{itemize}

\subsubsection*{Week 10}

\subsubsection*{The Euro-Crisis and its aftermath}

Study questions

\begin{itemize}
	\item What are the origins of the Euro crisis?
	\item Do existing institutions have the capacity to solve the Euro problem?
	\item Is it possible to have a currency union without a fiscal union?
	\item What are the consequences of the Euro crisis for democracy in Europe?
\end{itemize}

\noindent Key reading

\begin{itemize}
	\item Cini, M. and Borragán, N.P.S. 2019. European Union Politics. Oxford University Press. Chapter 26.
	\item Copelovitch, M., Frieden, J. and Walter, S. 2016. The political economy of the euro crisis. Comparative Political Studies, 49(7), pp.811-840.
	\item Tarlea, S., Bailer, S. and Degner, H. 2019. Explaining Governmental Preferences on Economic and Monetary Union Reform. European Union Politics 20(1).
	\item Rodrik, D. 2018. How Democratic Is the Euro? Available here: \url{https://www.project-syndicate.org/commentary/how-democratic-is-the-euro-by-dani-rodrik-2018-06?barrier=accesspaylog}.
\end{itemize}

\noindent Further readings

\begin{itemize}
	\item Wren-Lewis, S. 2018. Should Eurozone central bankers keep quiet about fiscal policy? Available here: \url{https://mainlymacro.blogspot.com/2018/07/should-eurozone-central-bankers-keep.html?m=1}.
	\item Armingeon, K. and Cranmer, S. 2018. Position-taking in the Euro crisis. Journal of European public policy, 25(4), pp.546-566.
	\item Steinbach, A. 2018. EU economic governance after the crisis: revisiting the accountability shift in EU economic governance. Journal of European Public Policy, pp.1-19.
	\item Matthijs, M. and McNamara, K. 2015. The Europe crisis’ theory effect: northern saints, southern sinners, and the demise of the Eurobond, Journal of European Integration 37(5), pp.229–45.
	\item Scharpf, F.W. 2011. Monetary Union, Fiscal Crisis and the Preemption of Democracy. pp.1–46.
	\item Kanthak, L. and Spies, D. C. 2018. Public support for European Union economic policies, European Union Politics, 19(1), pp.97–118.
	\item Degner, H. and Leuffen, D. 2018. Franco-German cooperation and the rescuing of the Eurozone. European Union Politics.
	\item Kanthak, L. and Spies, D. C. 2018. Public support for European Union economic policies. European Union Politics, 19(1), pp.97–118.
	\item Bauhr, M. and Charron, N. 2018. Why support International redistribution? Corruption and public support for aid in the eurozone. European Union Politics, 19(2), pp.233–254.
	\item Finke, D. and Bailer, S. 2018. Crisis bargaining in the European Union: Formal rules or market pressure? European Union Politics.
	\item Verdun, A. 2018. Institutional Architecture of the Euro Area. JCMS: Journal of Common Market Studies, 56, pp.74–84.
	\item Wasserfallen, F, Leuffen, D, and Kudrna, Z. 2019. Analysing European Union Decision-Making during the Eurozone Crisis with New Data. European Union Politics 20(1).
	\item Lehner, T. and Wasserfallen, F. 2019. Political Conflict in the Reform of the Eurozone. European Union Politics 20(1).
	\item Walter, S. 2016. Crisis politics in Europe. Comparative Political Studies 49, pp.841–873.
	\item Camisão, I. 2015. Irrelevant player? The Commission’s role during the Eurozone crisis. Journal of Contemporary European Research 11.
	\item Bauer, MW. and Becker, S. 2014. The unexpected winner of the crisis: The European Commission’s strengthened role in economic governance. Journal of European Integration 36 pp.213–229.
	\item Schimmelfennig, F. 2015. Liberal intergovernmentalism and the euro area crisis. Journal of European Public Policy 22 pp.177–195. 
	\item Lundgren, M, Bailer, S, and Dellmuth, L.M. 2019. Bargaining Success in the Reform of the Eurozone. European Union Politics 20(1).
	\item Degner, H and Leuffen, D. 2019. Franco-German Cooperation and the Rescuing of the Eurozone. European Union Politics 20(1).
	\item Regan, A. 2015. The imbalance of capitalisms in the Eurozone: Can the north and south of Europe converge? Comparative European Politics.
	\item Kaiser, J. and Kleinen-von K{\"o}nigsl{\"o}w, K. 2016. The Framing of the Euro Crisis in German and Spanish Online News Media between 2010 and 2014: Does a Common European Public Discourse Emerge?. JCMS: Journal of Common Market Studies.
	\item Kaiser, J. and Kleinen-von K{\"o}nigsl{\"o}w, K. 2016. Partisan journalism and the issue framing of the Euro crisis: Comparing political parallelism of German and Spanish online news. Journalism.
	\item Ioannou, D., Leblond, P. and Niemann, A. 2015. European integration and the crisis: practice and theory. Journal of European Public Policy, 22(2), pp.155-176.
	\item Braun, D. and Tausendpfund, M. 2014. The impact of the Euro Crisis on citizens’ support for the European Union. Journal of European Integration, 36(3), pp.231-245.
	\item Schimmelfennig, F. 2014. European integration in the euro crisis: The limits of postfunctionalism. Journal of European Integration, 36(3), pp.321-337.
	\item Clements, B., Nanou, K. and Verney, S. 2014. `We no longer love you, but we don’t want to leave you': the Eurozone crisis and popular Euroscepticism in Greece. Journal of European Integration, 36(3), pp.247-265.
	\item Tosun, J., Wetzel, A. and Zapryanova, G. 2014. The EU in crisis: advancing the debate. Journal of European Integration, 36(3), pp.195-211.
	\item Buti, M. and Carnot, N. 2012. The EMU Debt Crisis: Early Lessons and Reforms*. JCMS: Journal of Common Market Studies, 50(6), pp.899–911.
	\item Vilpišauskas, R. 2013. Eurozone crisis and European integration: functional spillover, political spillback?. Journal of European Integration, 35(3), pp.361-373.
	\item De Grauwe, P. and Ji, Y. 2012. Mispricing of Sovereign Risk and Macroeconomic Stability in the Eurozone*. JCMS: Journal of Common Market Studies, 50(6), pp.866–880.
	\item Drudi, F., Durre, A. and Mongelli, F.P. 2012. The Interplay of Economic Reforms and Monetary Policy: The Case of the Eurozone. JCMS: Journal of Common Market Studies, 50(6), pp.881–898.
	\item Eijffinger, S.C.W. 2012. Rating Agencies: Role and Influence of Their Sovereign Credit Risk Assessment in the Eurozone*. JCMS: Journal of Common Market Studies, 50(6), pp.912–921.
	\item Verdun, A. 2012. Introduction to the Symposium: Economic and Monetary Union and the Crisis of the Eurozone. JCMS: Journal of Common Market Studies, 50(6), pp.863–865.
	\item Dawson, M. 2015. The Legal and Political Accountability Structure of `Post-Crisis' EU Economic Governance. JCMS: Journal of Common Market Studies, 53(5), pp.976-993.
\end{itemize}

\subsubsection*{Week 11}

\subsubsection*{The EU’s democratic deficit and EU transparency}

Study questions

\begin{itemize}
	\item What is a democratic deficit? Does the EU suffer from a lack of democratic legitimacy? Should we hold the EU to the same ideals of democratic legitimacy as we do nation states?
	\item How transparent is the EU? How has this changed over time? Is increasing transparency a potential way in which to make the EU more democratic and legitimate?
\end{itemize}

\noindent Key reading

\begin{itemize}
	\item Cini, M. and Borragán, N.P.S. 2019. European Union Politics. Oxford University Press. Chapter 9.
	\item Follesdal, A. and Hix, S. 2006. Why There is a Democratic Deficit in the EU: A Response to Majone and Moravcsik. Journal of Common Market Studies 44(3), pp.533-62.
	\item Moravcsik, A. 2008. The Myth of Europe’s “Democratic Deficit”. Intereconomics: Journal of European Economic Policy 43(6), pp.331-40.
	\item European Commission. 2019. A-Z Index of Euromyths 1992 to 2017. Read 2-3 examples from here: \url{https://blogs.ec.europa.eu/ECintheUK/euromyths-a-z-index/}
	\item Presidency of the Council of the European Union, 2001. Presidency Conclusions, European Council Meeting in Laeken (The Laeken Delaration). \url{https://www.consilium.europa.eu/uedocs/cms_data/docs/pressdata/en/ec/68827.pdf}
	\item Naurin, D. 2007. Deliberation Behind Closed Doors. Transparency and Lobbying in the European Union. Colchester: ECPR Press. Ch1-2.
\end{itemize}

\noindent Further reading

\begin{itemize}
	\item Brandsma, G.J. 2018. Transparency of EU informal trilogues through public feedback in the European Parliament: promise unfulfilled. Journal of European Public Policy, pp.1-20.
	\item Kreuder-Sonnen, C. 2018. Political secrecy in Europe: crisis management and crisis exploitation. West European Politics, 41(4), pp.958-980.
	\item Blatter, J., Schmid, S. D., and Bl{\"a}ttler, A. C. (2017) Democratic Deficits in Europe: The Overlooked Exclusiveness of Nation-States and the Positive Role of the European Union. JCMS: Journal of Common Market Studies, 55: 449–467.
	\item Karlsson, C., and Persson, T. 2018. The Alleged Opposition Deficit in European Union Politics: Myth or Reality?. JCMS: Journal of Common Market Studies, 56: 888–905.
	\item B{\o}lstad, J. 2015. Dynamics of European integration: Public opinion in the core and periphery. European Union Politics, 16(1), pp.23-44.
	\item Crombez, C. 2003. The Democratic Deficit in the European Union: Much Ado about Nothing? European Union Politics 4, pp.101-20.
	\item Habermas, J. 2008. Europe: The Faltering Project. Cambridge: Polity Press.
	\item Hix, S.	2013. What's Wrong with the Europe Union and How to Fix it. John Wiley \& Sons.
	\item Majone, G. 2000. The Credibility Crisis of Community Regulation. Journal of Common Market Studies 38(2), pp.273–302.
	\item Moravcsik, A. 2002. In Defence of the Democratic Deficit: Reassessing Legitimacy in the European Union. Journal of Common Market Studies 40(4), pp.603-24.
	\item Kohler-Koch, B. and Rittberger, B. 2007. Debating the Democratic Legitimacy of the European Union. Lanham, MD: Rowman and Littlefield.
	\item Thomson, R. 2011. Resolving Controversy in the EU. Cambridge University Press. Chapter 12. Free pre-publication version available at: \url{http://www.robertthomson.info/research/resolving-controversy-in-the-eu}
	\item Zweifel, T. D. 2002. Who is without sin cast the first stone: the EU's democratic deficit in comparison. Journal of European Public Policy 9(5), pp.812-840.
	\item Cross, J.P. 2013a. Striking a pose: transparency and position taking in the Council of the European Union. European Journal of Political Research, 52(3), pp.291–315.
	\item Cross, J.P. 2014. The seen and the unseen in legislative politics: Explaining censorship in the Council of Ministers of the European Union. Journal of European Public Policy, 21(2), pp.268-285.
	\item de Fine Licht, J., Naurin, D., Esaiasson, P. and Gilljam, M. 2011. Does transparency generate legitimacy? An experimental study of procedure acceptance of open-and closed-door decision-making. QoG Working Paper Series, 2011(8), p.8.
	\item Levy, G. 2007. Decision making in committees: Transparency, reputation, and voting rules. American Economic Review, 97(1), pp.150–168.
	\item Lodge, J. 1994. Transparency and democratic legitimacy. JCMS: Journal of Common Market Studies, 32(3), pp.343–368.
	\item Settembri, P. 2005. Transparency and the EU Legislator: ``Let He Who is Without Sin Cast the First Stone". JCMS: Journal of Common Market Studies, 43(3), pp.637–654.
	\item Stasavage, D. 2006. Does transparency make a difference? The example of the European Council of Ministers. In Proceedings-British Academy 135, p.165. Oxford University Press Inc..
	\item Stasavage, D. 2004. Open-door or closed-door? Transparency in domestic and international bargaining. International Organization, 58(04), pp.667–703.
\end{itemize}

\subsubsection*{Week 12}

Revision and exam preparation
%\subsubsection*{Brexit}

Study questions

\begin{itemize}
	\item What factors explain the outcome of the Brexit referendum vote?
	\item What factors explain the outcome of the Brexit negotiations between the UK and the EU?
	\item How has Ireland, a much small country than the UK, managed to have its views enshrined in the withdrawal agreement?
	\item How is Brexit (if it goes ahead) likely to affect EU decision-making?
\end{itemize}

\noindent Required Readings

\begin{itemize}
	\item Cini, M. and Borragán, N.P.S. 2019. European Union Politics. Oxford University Press. Chapter 27.
	\item Hobolt, S. B. 2016. The Brexit vote: a divided nation, a divided continent. Journal of European Public Policy, 23(9), pp.1259-1277.
	\item Kirsch, W. 2016. Brexit and the Distribution of Power in the Council of the EU, CEPS Commentaries, pp.1-4. 
	\item Springford, J. 2018. Theresa May's Irish trilemma. \url{https://www.cer.eu/insights/theresa-mays-irish-trilemma}.
	\item Jones, E. 2018. Four Things We Should Learn from Brexit, Survival, 60:6, pp.35-44.
\end{itemize}

\noindent Further reading

\begin{itemize}
	\item Patel, O. \& Reh, C. 2016. Brexit: The Consequences for the EU Political System, UCL Constitution Unit Briefing Paper, pp.1-5. 
	\item Staal, K. 2016. Brexit Implications for Influence on EU Decision Making, The Stability of Regions, Culture, and Institutions VIVES Workshop, University of Leuven, Belgium, pp.1-6.
	\item Inglehart, R., \& Norris, P. 2016. Trump, Brexit, and the rise of populism: Economic have-nots and cultural backlash. HKS Working Paper No. RWP16-026. Available at SSRN: \url{https://ssrn.com/abstract=2818659}.
	\item European Commission. 2019. Withdrawal of the United Kingdom from the EU. Available here: \url{https://ec.europa.eu/taxation_customs/uk_withdrawal_en}
	\item Lavery, S., McDaniel, S. and Schmid, D., 2018. Finance fragmented? Frankfurt and Paris as European financial centres after Brexit. Journal of European Public Policy, pp.1-19.
	\item Jennings, W. and Lodge, M. 2018. Brexit, the tides and Canute: the fracturing politics of the British state, Journal of European Public Policy
	\item Qvortrup, M. 2016. Referendums on Membership and European Integration 1972–2015. The Political Quarterly, 87, pp.61-68.
	\item Henderson, A. , Jeffery, C. , Liñeira, R. , Scully, R. , Wincott, D. and Wyn Jones, R. 2016. England, Englishness and Brexit. The Political Quarterly, 87, pp.187-199.
	\item James, S. and Quaglia, L. 2018. The Brexit Negotiations and Financial Services: A Two-Level Game Analysis. The Political Quarterly, 89, pp.560-567.
	\item Renwick, A. , Allan, S. , Jennings, W. , Mckee, R. , Russell, M. and Smith, G. 2018. What Kind of Brexit do Voters want? Lessons from the Citizens’ Assembly on Brexit. The Political Quarterly, 89, pp.649-658.
	\item Laffan, B. 2018. Brexit: Re-opening Ireland's English Question. The Political Quarterly, 89, pp.568-575. 
	\item Chen W, Los B, McCann P, Ortega-Argilés R, Thissen M, van Oort F. 2018. The continental divide? Economic exposure to Brexit in regions and countries on both sides of The Channel. Papers in Regional Science. 97, pp.25–54.
	\item Richardson, J. 2018. Brexit: The EU Policy-Making State Hits the Populist Buffers. The Political Quarterly, 89, pp.118-126.
	\item Richards, L. , Heath, A. and Carl, N. 2018. Red Lines and Compromises: Mapping Underlying Complexities of Brexit Preferences. The Political Quarterly, 89, pp.280-290.
	\item Hantzsche, A. , Kara, A. and Young, G. 2018. The Economic Effects of the UK Government's Proposed Brexit Deal. The World Economy, Accepted Author Manuscript.
	\item Gasiorek, M. , Serwicka, I. and Smith, A. 2019. Which Manufacturing Industries and Sectors Are Most Vulnerable to Brexit?. The World Economy, Accepted Author Manuscript.
	\item Manners, I. 2018. Political Psychology of European Integration: The (Re)production of Identity and Difference in the Brexit Debate. Political Psychology, 39, pp.1213-1232.
	\item Warlouzet, L. 2018. Britain at the Centre of European Co-operation (1948–2016). JCMS: Journal of Common Market Studies, 56, pp.955–970.
	\item Dhingra, S., \& Sampson, T. 2016. Life after BREXIT: What are the UK’s options outside the European Union?. CEPBREXIT01. London School of Economics and Political Science, CEP, London, UK.
	\item Goodwin, M. J., \& Heath, O. 2016. The 2016 Referendum, Brexit and the Left Behind: An Aggregate-level Analysis of the Result. The Political Quarterly, 87(3), pp.323-332.
	\item Grant, C. 2016. The Impact of Brexit on the EU, Centre for European Reform, blog post, 24 June 2016. 
	\item Kierzenkowski, R., et al. 2016. The Economic Consequences of Brexit: A Taxing Decision, OECD Economic Policy Papers, No. 16, OECD Publishing, Paris.
	\url{http://dx.doi.org/10.1787/5jm0lsvdkf6k-en}.
	\item Kaufmann, E. 2016. It’s NOT the economy, stupid: Brexit as a story of personal values. British Politics and Policy at LSE.
	\item Whitman, R. G. 2016. Brexit or Bremain: what future for the UK's European diplomatic strategy?. International Affairs, 92(3), pp.509-529.
	\item Ottaviano, G. I. P., Pessoa, J. P., Sampson, T., \& Van Reenen, J. 2014. Brexit or Fixit? The trade and welfare effects of leaving the European Union.
	\item Oliver, T. 2016. European and international views of Brexit. Journal of European Public Policy, 23(9), pp.1321-1328.
	\item Menon, A., \& Salter, J. P. 2016. Brexit: initial reflections. International Affairs, 92(6), pp.1297-1318.
	\item Jensen, M. D., \& Snaith, H. 2016. When politics prevails: the political economy of a Brexit. Journal of European Public Policy, 23(9), pp.1302-1310.
	\item Oliver, T., \& Williams, M. J. 2016. Special relationships in flux: Brexit and the future of the US—EU and US—UK relationships. International Affairs, 92(3), pp.547-567.
\end{itemize}





