%!TEX root = INRL20160_Syllabus_1920.tex

\section*{Assessment}

\subsection*{End of year exam}

There will be a formal \textbf{two-hour examination} at the end of the semester as scheduled by the examination office. This examination will constitute 80\% of your final mark. Example exams demonstrating the structure and requirements of the end of year exam will be distributed in due course.

There will be \textbf{8 questions} on the exam and you have to \textbf{answer 2} of them.

\subsection*{Blog-post assignment}

In addition to the end of year exam, you will be required to submit a blog-post assignment. The blog-post assignments are intended to assess your substantive knowledge of European integration and apply the insights you have gained from the course to current issues facing the EU and European integration. Good blog posts should show consistency in argument, clear structure, simple and direct writing, good punctuation and evidence of wide reading. The relatively short length of the post means that students must prioritise what they include in their text. They need to think carefully about what elements of their argument need to be developed and what elements can be given less attention. You should endeavour to include graphs, data, and existing research findings in the academic literature, with accurate referencing where appropriate. Your answers should demonstrate your ability to:

\begin{itemize}
	\item Identify important, relevant and recent developments in European integration.
	\item Identify the debates and academic authors in the discipline that address these developments
	\item Understand and apply the main theoretical approaches covered in the course to analyse EU politics
	\item Critically assess these approaches by drawing on the secondary literature on European integration as well as empirical evidence and data
\end{itemize}

	To aid you in preparing your assignment, we will provide 2 lectures on best practice and writing skills. The lectures are aimed at helping you with your assignment and will revolve around the development of a relevant question and framing a concise argument to answer your chosen question. These lectures will be held on \textbf{Monday 27th January} and \textbf{Monday 3rd February} in the regular room and time of the normal lecture. 

	Following this, writing groups shall be organised in Week 4 and 5 so that your idea can be workshopped and developed under the tutorage of Natalia Umansky, the module teaching assistant. You are required to sign up to a workshop date via Brightspace at the start of the module.
	
	Prior to the reading groups, you will be required to submit a \textbf{250 word outline} of your blog post. This should include the question, and rationale behind the question, and your main line of argument, which at this early stage will be in a basic form. This outline will be provided to two other students who will be required to provide feedback on the question (such as, is it leading, loaded, or too wide); the argument, and the feasibility of the overall claim to be argued in the 1,000 words of the blog post. The date for these exercises and the dates of the reading groups will be provided in due course. The 250 word outline will be submitted via Brightspace ahead of the workshops.

	The following guidelines should be adhered to when preparing your final assignment submission:
	
\begin{itemize}
	\item The text should be \textit{strictly} under \textbf{1000 words}.
	\item This 1000 words \textbf{word count does not include the bibliography}. 
	\item This blog-post assignment will constitute \textbf{20\% of your final mark} in this module.
	\item Your blog-post assignment should be submitted no later than \textbf{5 p.m. on 6th March 2019}.
	\item You will be required to submit a \textbf{PDF copy of your assignment electronically via Brightspace.} Google how to do this if you do not know how to do so already.
	\item You will also have to submit a declaration of authorship form via Brightspace, which can be found here: \url{https://www.ucd.ie/spire/t4media/School%20Declaration%20of%20Authorship%20form%202019%20Web.pdf}
	\item \textbf{No hard copy is required.}
\end{itemize}

	A selection of the best blog posts shall be published (with the author's consent) on the Dublin European Institute's blog found at \url{www.europedebate.ie}. This widely read and highly visible platform provides the opportunity for students to contribute to the public debate about European integration in a concrete manner to a national and international audience. It also allows students to publicly demonstrate their mastery of their chosen topic in a professional context, and as such represents a great opportunity to build a profile for future employers be they in academia, the public sector or the private sector. 

	Examples of excellent example blog posts to inspire you can be found at \url{http://blogs.lse.ac.uk/europpblog/} and \url{www.europedebate.ie}. What you will notice about the blog posts on these sites is that they take insights gained from the political science literature, and use these insights to make a concise argument about current issues in EU integration and policy making. Please consult and explore this webpage at length to get an idea of what is expected of you.

%	One important difference between the above mentioned blog posts and your assignment, is that you are required to include more detailed and complete referencing in your blog assignment. You want to demonstrate that you have read the relevant literature and can reference it where necessary. The best blog-post assignments will combine clarity of argument with well researched points and demonstrate a clear mastery of the relevant literature.

%	The blog post is also intended to assess your study skills. Your answers should demonstrate your ability to:

%\begin{itemize}
%	\item Draw \textit{selectively} on a range of relevant material, including existing literature on European integration, official documents, contemporary news sources, data, and the internet
%	\item Understand, analyse and critically assess that material
%	\item Use the material to make and sustain a well-structured line of argument
%	\item Write in a concise and cogent style - The 1000 word limit forces you to prioritise information and think carefully about how to make a point in a relatively short amount of text.
%\end{itemize}
