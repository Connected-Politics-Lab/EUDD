%!TEX root = INRL20160_Syllabus_1920.tex

\section*{Introduction}

	This course draws on a range of political science research on European integration and European Union politics to analyse the development of the EU and how it operates today. The course addresses one of the most important questions in the study of European politics and international organisations: Why did a diverse group of states construct what is currently the world’s most extensive example of international integration? The course provides an extensive overview of the contemporary EU, including its institutions and policymaking processes using approaches from modern political science. We will assess how the EU is coping with economic and political challenges to its prosperity and legitimacy. By the end of the module, students will have a thorough grounding in EU politics and the manner in which EU institutions function and cooperate to make policy.
	