%!TEX root = INRL20160_Syllabus_1920.tex

\subsubsection*{Theories of European integration}

Study questions

\begin{itemize}
	\item Describe the main logic behind neofunctionalism as a theory of European integration. Can this theory successfully explain how we got to where we are now in terms of EU integration?
	\item Describe the main logic behind intergovernmentalism. Does this theory do a better job of explaining the European integration project than neofunctionalism?
	\item What are the strengths and weaknesses of both theories of integration?
	\item What are the three different types of institutionalism? How do they differ from one another?
	\item How do institutionalist theories differ from the `grand' theories of integration that went before?
	\item How successful are institutionalist theories at explaining EU politics?
\end{itemize}

\noindent Key reading

\begin{itemize}
	\item Cini, M. and Pérez-Solórzano Borragán, N. (eds.) 2019. European Union Politics, Oxford: Oxford University Press. Chapters 54-5
	\item Moravcsik, A. 1993. Preferences and Power in the European Community. Journal of Common Market Studies 31(4), pp.473-524.
	\item Aspinwall, M. and Schneider, G. 2000. Same Menu, Separate Tables: The Institutionalist Turn in Political Science and the Study of European Integration. European Journal of Political Research 38, pp.1-36.
\end{itemize}


\noindent Further reading

\begin{itemize}
	\item Hoffmann, S. 1966. Obstinate or obsolete? The fate of the nation-state and the case of Western Europe. Daedalus, pp.862-915.
	\item Haas, E.B. 1976. Turbulent Fields and the Theory of Regional Integration. International Organization 30, pp.173-212.
	\item Monnet, J. 1978. Memoirs. New York: Doubleday.
	\item Bache, I. and George, S. 2011. Politics in the European Union. Oxford: Oxford University Press, Chapter 1.
	\item Haas, E.B. 1958. The Uniting of Europe. Stanford: Stanford University Press.
	\item Haas, E.B. 1961. International Integration: The European and Universal Process. International Organization 15(3), pp.366-92.
	\item Haas, E.B. 1975. The Obsolescence of Regional Integration Theory. Berkeley: Institute of International Studies.
	\item Hooghe, L and Marks G. 2001. Multi-Level Governance and European Integration. Oxford: Rowman and Littlefield.
	\item Jupille, J. and Caporaso, J.A. 1999. Institutionalism and the European Union: Beyond Comparative Politics and International Relations. Annual Review Political Science 2, pp.429-44.
	\item Moravcsik, A. 1998. The Choice For Europe: Social Purpose And State Power From Messina To Maastricht. Cornell University Press, Chapter 1, 7.
	\item Nugent, N. 2010. The Government and Politics of the European Union. London: Palgrave Macmillan, Chapter 23.
	\item Pollack, M.A. 2003. The Engines of European Integration: Delegation, Agency and Agenda Setting in the EU. Oxford: Oxford University Press.
	\item Pollock, M.A. 2005. Theorizing the EU: International Organization, Domestic Polity, or Experiment in New Governance? Annual Review Political Science. 8, pp.357–98.
	\item Rosamond, B. 2000. Theories of European Integration London: MacMillan, Chapters 2 and 3. 
	\item Sandholtz, W. And Sweet Stone, A. 1998. European Integration and Supranational Governance. Oxford: Oxford University Press.
	\item Weiner, A. and Diez, T. 2009. European Integration Theory. Oxford: Oxford University Press (2nd Edition).
	\item Hooghe, L. and Marks, G. 2009. A Postfunctionalist Theory of European Integration: From Permissive Consensus to Constraining Dissensus. British Journal of Political Science 39(1), pp.1-23.
	\item Pierson P. 1996., The path to European Integration: An Historical Institutionalist Perspective. Comparative Political Studies. 29(2), pp.123-63.
	\item Tsebelis, G. and Garrett, G. 2001., The Institutional Foundations of Intergovernmentalism and Supranationalism. International Organization 55(2), pp.357-90.
	\item Bickerton, C.J., Hodson, D. and Puetter, U. 2014. The new intergovernmentalism: European integration in the post-Maastricht era. Journal of Common Market Studies 53(4), pp.703–22.
	\item Schimmelfennig, F. 2015a. What’s the news in `new intergovernmentalism'? A critique of Bickerton, Hodson and Puetter. JCMS: Journal of Common Market Studies 53(4), pp.723–30.
	\item Schimmelfennig, F. 2015. Liberal intergovernmentalism and the euro area crisis. Journal of European Public Policy, 22(2), pp.177-195.
	\item Puetter, U. 2016. The centrality of consensus and deliberation in contemporary EU politics and the new intergovernmentalism. Journal of European Integration, 38(5), pp.601-615.
	\item Puetter, U. and Fabbrini, S. 2016. Catalysts of integration–the role of core intergovernmental forums in EU politics. Journal of European Integration, 38(5), pp.633-642.
	\item Kleine, M., and Pollack, M. 2018. Liberal Intergovernmentalism and Its Critics. JCMS: Journal of Common Market Studies, 56, pp.1493–1509.
	\item Moravcsik, A. 2018. Preferences, Power and Institutions in 21st-century Europe. JCMS: Journal of Common Market Studies, 56, pp.1648-1674.
	\item Meunier, S., and Vachudova, M. A. 2018. Liberal Intergovernmentalism, Illiberalism and the Potential Superpower of the European Union. JCMS: Journal of Common Market Studies, 56, pp.1631–1647.
	\item Hix, S. 2018. When Optimism Fails: Liberal Intergovernmentalism and Citizen Representation. JCMS: Journal of Common Market Studies, 56, pp.1595–1613.
	\item Schimmelfennig, F. 2018. Liberal Intergovernmentalism and the Crises of the European Union. JCMS: Journal of Common Market Studies, 56, pp.1578–1594.
	\item Schimmelfennig, F. 2018. European integration (theory) in times of crisis. A comparison of the Euro and Schengen crises, Journal of European Public Policy, 257), pp.969-989
	\item Schmidt, V. A. 2018. Rethinking EU Governance: From Old to New Approaches to Who Steers Integration. JCMS: Journal of Common Market Studies, 56, pp.1544–1561.
	\item Naurin, D. 2018. Liberal Intergovernmentalism in the Councils of the EU: A Baseline Theory?. JCMS: Journal of Common Market Studies, 56, pp.1526–1543.
	\item McNamara, K. R. 2018. Authority Under Construction: The European Union in Comparative Political Perspective. JCMS: Journal of Common Market Studies, 56, pp.1510–1525. 
	\item Jones, E. 2018. Towards a theory of disintegration, Journal of European Public Policy, 25(3), pp.440-451.
\end{itemize}
