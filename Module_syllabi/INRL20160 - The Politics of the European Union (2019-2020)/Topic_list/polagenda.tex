\subsubsection*{The EU policy agenda}

Study questions

\begin{itemize}
	\item Describe punctuated equilibrium theory and explain how it helps us understand policy-agenda dynamics in the European Union.
	\item How has the policy agenda of the European Council changed over time? What factors drive these changes?
	\item How do policy agendas differ between the European Council and the Commission? What explains these differences?
\end{itemize}

\noindent Key reading
	
\begin{itemize}
	\item Jones, B.D. and Baumgartner, F.R. 2012. From there to here: Punctuated equilibrium to the general punctuation thesis to a theory of government information processing. Policy Studies Journal, 40(1), pp.1-20.
	\item Princen, S. 2011. Agenda-setting strategies in EU policy processes. Journal of European Public Policy, 18(7), pp.927-943.
	\item Carammia, M., Princen, S. and Timmermans, A., 2016. From Summitry to EU Government: An Agenda Formation Perspective on the European Council. JCMS: Journal of Common Market Studies, 54(4), pp.809-825.
	\item Alexandrova, P., 2017. Institutional issue proclivity in the EU: the European Council vs the Commission. Journal of European Public Policy, 24(5), pp.755-774.
\end{itemize}

\noindent Further reading

\begin{itemize}
	\item Alexandrova, P., Carammia, M. and Timmermans, A. 2012. Policy punctuations and issue diversity on the European Council agenda. Policy Studies Journal, 40(1), pp.69-88.
	\item Princen, S. 2016. 19. Agenda setting in the European Union: from sui generis to mainstream. Handbook of Public Policy Agenda Setting, p.348.
	\item Alexandrova, P. 2015b. Explaining political attention allocation with the help of issue character: evidence from the European Council. European Political Science Review, 8(3), pp.405-435.
	\item Tosun, J., Biesenbender, S. and Schulze, K. 2015. Building the EU’s Energy Policy Agenda: Insights Gained. In Energy Policy Making in the EU (pp. 247-263). Springer London.
	\item Greene, D. and Cross, J.P. 2016. Exploring the Political Agenda of the European Parliament Using a Dynamic Topic Modeling Approach. arXiv preprint arXiv:1607.03055.
	\item Alexandrova, P. and Timmermans, A. 2013. National interest versus the common good: The Presidency in European Council agenda setting. European Journal of Political Research, 52(3), pp.316-338.
	\item Alexandrova, P., Carammia, M., Princen, S. and Timmermans, A., 2014. Measuring the European Council agenda: Introducing a new approach and dataset. European Union Politics, 15(1), pp.152-167.
	\item Princen, S. 2015. Studying Agenda Setting. In Research Methods in European Union Studies (pp. 123-135). Palgrave Macmillan UK.
	\item Alexandrova, P. 2015. Upsetting the agenda: the clout of external focusing events in the European Council. Journal of Public Policy, 35(03), pp.505-530.
	\item Tosun, J., Biesenbender, S. and Schulze, K. 2015. Building the EU’s Energy Policy Agenda: An Introduction. In Energy Policy Making in the EU (pp. 1-17). Springer London.
	\item Alexandrova, P., Rasmussen, A. and Toshkov, D. 2016. Agenda responsiveness in the European Council: public priorities, policy problems and political attention. West European Politics, 39(4), pp.605-627.
	
\end{itemize}
