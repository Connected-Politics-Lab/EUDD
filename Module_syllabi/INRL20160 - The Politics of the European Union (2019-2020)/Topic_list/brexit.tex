\subsubsection*{Brexit}

Study questions

\begin{itemize}
	\item What factors explain the outcome of the Brexit referendum vote?
	\item What factors explain the outcome of the Brexit negotiations between the UK and the EU?
	\item How has Ireland, a much small country than the UK, managed to have its views enshrined in the withdrawal agreement?
	\item How is Brexit (if it goes ahead) likely to affect EU decision-making?
\end{itemize}

\noindent Required Readings

\begin{itemize}
	\item Cini, M. and Borragán, N.P.S. 2019. European Union Politics. Oxford University Press. Chapter 27.
	\item Hobolt, S. B. 2016. The Brexit vote: a divided nation, a divided continent. Journal of European Public Policy, 23(9), pp.1259-1277.
	\item Kirsch, W. 2016. Brexit and the Distribution of Power in the Council of the EU, CEPS Commentaries, pp.1-4. 
	\item Springford, J. 2018. Theresa May's Irish trilemma. \url{https://www.cer.eu/insights/theresa-mays-irish-trilemma}.
	\item Jones, E. 2018. Four Things We Should Learn from Brexit, Survival, 60:6, pp.35-44.
\end{itemize}

\noindent Further reading

\begin{itemize}
	\item Patel, O. \& Reh, C. 2016. Brexit: The Consequences for the EU Political System, UCL Constitution Unit Briefing Paper, pp.1-5. 
	\item Staal, K. 2016. Brexit Implications for Influence on EU Decision Making, The Stability of Regions, Culture, and Institutions VIVES Workshop, University of Leuven, Belgium, pp.1-6.
	\item Inglehart, R., \& Norris, P. 2016. Trump, Brexit, and the rise of populism: Economic have-nots and cultural backlash. HKS Working Paper No. RWP16-026. Available at SSRN: \url{https://ssrn.com/abstract=2818659}.
	\item European Commission. 2019. Withdrawal of the United Kingdom from the EU. Available here: \url{https://ec.europa.eu/taxation_customs/uk_withdrawal_en}
	\item Lavery, S., McDaniel, S. and Schmid, D., 2018. Finance fragmented? Frankfurt and Paris as European financial centres after Brexit. Journal of European Public Policy, pp.1-19.
	\item Jennings, W. and Lodge, M. 2018. Brexit, the tides and Canute: the fracturing politics of the British state, Journal of European Public Policy
	\item Qvortrup, M. 2016. Referendums on Membership and European Integration 1972–2015. The Political Quarterly, 87, pp.61-68.
	\item Henderson, A. , Jeffery, C. , Liñeira, R. , Scully, R. , Wincott, D. and Wyn Jones, R. 2016. England, Englishness and Brexit. The Political Quarterly, 87, pp.187-199.
	\item James, S. and Quaglia, L. 2018. The Brexit Negotiations and Financial Services: A Two-Level Game Analysis. The Political Quarterly, 89, pp.560-567.
	\item Renwick, A. , Allan, S. , Jennings, W. , Mckee, R. , Russell, M. and Smith, G. 2018. What Kind of Brexit do Voters want? Lessons from the Citizens’ Assembly on Brexit. The Political Quarterly, 89, pp.649-658.
	\item Laffan, B. 2018. Brexit: Re-opening Ireland's English Question. The Political Quarterly, 89, pp.568-575. 
	\item Chen W, Los B, McCann P, Ortega-Argilés R, Thissen M, van Oort F. 2018. The continental divide? Economic exposure to Brexit in regions and countries on both sides of The Channel. Papers in Regional Science. 97, pp.25–54.
	\item Richardson, J. 2018. Brexit: The EU Policy-Making State Hits the Populist Buffers. The Political Quarterly, 89, pp.118-126.
	\item Richards, L. , Heath, A. and Carl, N. 2018. Red Lines and Compromises: Mapping Underlying Complexities of Brexit Preferences. The Political Quarterly, 89, pp.280-290.
	\item Hantzsche, A. , Kara, A. and Young, G. 2018. The Economic Effects of the UK Government's Proposed Brexit Deal. The World Economy, Accepted Author Manuscript.
	\item Gasiorek, M. , Serwicka, I. and Smith, A. 2019. Which Manufacturing Industries and Sectors Are Most Vulnerable to Brexit?. The World Economy, Accepted Author Manuscript.
	\item Manners, I. 2018. Political Psychology of European Integration: The (Re)production of Identity and Difference in the Brexit Debate. Political Psychology, 39, pp.1213-1232.
	\item Warlouzet, L. 2018. Britain at the Centre of European Co-operation (1948–2016). JCMS: Journal of Common Market Studies, 56, pp.955–970.
	\item Dhingra, S., \& Sampson, T. 2016. Life after BREXIT: What are the UK’s options outside the European Union?. CEPBREXIT01. London School of Economics and Political Science, CEP, London, UK.
	\item Goodwin, M. J., \& Heath, O. 2016. The 2016 Referendum, Brexit and the Left Behind: An Aggregate-level Analysis of the Result. The Political Quarterly, 87(3), pp.323-332.
	\item Grant, C. 2016. The Impact of Brexit on the EU, Centre for European Reform, blog post, 24 June 2016. 
	\item Kierzenkowski, R., et al. 2016. The Economic Consequences of Brexit: A Taxing Decision, OECD Economic Policy Papers, No. 16, OECD Publishing, Paris.
	\url{http://dx.doi.org/10.1787/5jm0lsvdkf6k-en}.
	\item Kaufmann, E. 2016. It’s NOT the economy, stupid: Brexit as a story of personal values. British Politics and Policy at LSE.
	\item Whitman, R. G. 2016. Brexit or Bremain: what future for the UK's European diplomatic strategy?. International Affairs, 92(3), pp.509-529.
	\item Ottaviano, G. I. P., Pessoa, J. P., Sampson, T., \& Van Reenen, J. 2014. Brexit or Fixit? The trade and welfare effects of leaving the European Union.
	\item Oliver, T. 2016. European and international views of Brexit. Journal of European Public Policy, 23(9), pp.1321-1328.
	\item Menon, A., \& Salter, J. P. 2016. Brexit: initial reflections. International Affairs, 92(6), pp.1297-1318.
	\item Jensen, M. D., \& Snaith, H. 2016. When politics prevails: the political economy of a Brexit. Journal of European Public Policy, 23(9), pp.1302-1310.
	\item Oliver, T., \& Williams, M. J. 2016. Special relationships in flux: Brexit and the future of the US—EU and US—UK relationships. International Affairs, 92(3), pp.547-567.
\end{itemize}