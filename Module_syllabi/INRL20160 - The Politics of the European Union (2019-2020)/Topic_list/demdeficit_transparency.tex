\subsubsection*{The EU’s democratic deficit and EU transparency}

Study questions

\begin{itemize}
	\item What is a democratic deficit? Does the EU suffer from a lack of democratic legitimacy? Should we hold the EU to the same ideals of democratic legitimacy as we do nation states?
	\item How transparent is the EU? How has this changed over time? Is increasing transparency a potential way in which to make the EU more democratic and legitimate?
\end{itemize}

\noindent Key reading

\begin{itemize}
	\item Cini, M. and Borragán, N.P.S. 2019. European Union Politics. Oxford University Press. Chapter 9.
	\item Follesdal, A. and Hix, S. 2006. Why There is a Democratic Deficit in the EU: A Response to Majone and Moravcsik. Journal of Common Market Studies 44(3), pp.533-62.
	\item Moravcsik, A. 2008. The Myth of Europe’s “Democratic Deficit”. Intereconomics: Journal of European Economic Policy 43(6), pp.331-40.
	\item European Commission. 2019. A-Z Index of Euromyths 1992 to 2017. Read 2-3 examples from here: \url{https://blogs.ec.europa.eu/ECintheUK/euromyths-a-z-index/}
	\item Presidency of the Council of the European Union, 2001. Presidency Conclusions, European Council Meeting in Laeken (The Laeken Delaration). \url{https://www.consilium.europa.eu/uedocs/cms_data/docs/pressdata/en/ec/68827.pdf}
	\item Naurin, D. 2007. Deliberation Behind Closed Doors. Transparency and Lobbying in the European Union. Colchester: ECPR Press. Ch1-2.
\end{itemize}

\noindent Further reading

\begin{itemize}
	\item Brandsma, G.J. 2018. Transparency of EU informal trilogues through public feedback in the European Parliament: promise unfulfilled. Journal of European Public Policy, pp.1-20.
	\item Kreuder-Sonnen, C. 2018. Political secrecy in Europe: crisis management and crisis exploitation. West European Politics, 41(4), pp.958-980.
	\item Blatter, J., Schmid, S. D., and Bl{\"a}ttler, A. C. (2017) Democratic Deficits in Europe: The Overlooked Exclusiveness of Nation-States and the Positive Role of the European Union. JCMS: Journal of Common Market Studies, 55: 449–467.
	\item Karlsson, C., and Persson, T. 2018. The Alleged Opposition Deficit in European Union Politics: Myth or Reality?. JCMS: Journal of Common Market Studies, 56: 888–905.
	\item B{\o}lstad, J. 2015. Dynamics of European integration: Public opinion in the core and periphery. European Union Politics, 16(1), pp.23-44.
	\item Crombez, C. 2003. The Democratic Deficit in the European Union: Much Ado about Nothing? European Union Politics 4, pp.101-20.
	\item Habermas, J. 2008. Europe: The Faltering Project. Cambridge: Polity Press.
	\item Hix, S.	2013. What's Wrong with the Europe Union and How to Fix it. John Wiley \& Sons.
	\item Majone, G. 2000. The Credibility Crisis of Community Regulation. Journal of Common Market Studies 38(2), pp.273–302.
	\item Moravcsik, A. 2002. In Defence of the Democratic Deficit: Reassessing Legitimacy in the European Union. Journal of Common Market Studies 40(4), pp.603-24.
	\item Kohler-Koch, B. and Rittberger, B. 2007. Debating the Democratic Legitimacy of the European Union. Lanham, MD: Rowman and Littlefield.
	\item Thomson, R. 2011. Resolving Controversy in the EU. Cambridge University Press. Chapter 12. Free pre-publication version available at: \url{http://www.robertthomson.info/research/resolving-controversy-in-the-eu}
	\item Zweifel, T. D. 2002. Who is without sin cast the first stone: the EU's democratic deficit in comparison. Journal of European Public Policy 9(5), pp.812-840.
	\item Cross, J.P. 2013a. Striking a pose: transparency and position taking in the Council of the European Union. European Journal of Political Research, 52(3), pp.291–315.
	\item Cross, J.P. 2014. The seen and the unseen in legislative politics: Explaining censorship in the Council of Ministers of the European Union. Journal of European Public Policy, 21(2), pp.268-285.
	\item de Fine Licht, J., Naurin, D., Esaiasson, P. and Gilljam, M. 2011. Does transparency generate legitimacy? An experimental study of procedure acceptance of open-and closed-door decision-making. QoG Working Paper Series, 2011(8), p.8.
	\item Levy, G. 2007. Decision making in committees: Transparency, reputation, and voting rules. American Economic Review, 97(1), pp.150–168.
	\item Lodge, J. 1994. Transparency and democratic legitimacy. JCMS: Journal of Common Market Studies, 32(3), pp.343–368.
	\item Settembri, P. 2005. Transparency and the EU Legislator: ``Let He Who is Without Sin Cast the First Stone". JCMS: Journal of Common Market Studies, 43(3), pp.637–654.
	\item Stasavage, D. 2006. Does transparency make a difference? The example of the European Council of Ministers. In Proceedings-British Academy 135, p.165. Oxford University Press Inc..
	\item Stasavage, D. 2004. Open-door or closed-door? Transparency in domestic and international bargaining. International Organization, 58(04), pp.667–703.
\end{itemize}