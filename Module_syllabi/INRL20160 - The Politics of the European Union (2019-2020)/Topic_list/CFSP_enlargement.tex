\subsubsection*{The EU on the world stage and EU enlargement}

Study questions

\begin{itemize}
	\item What do we mean when we say that the EU is a `soft security' actor? Is the EU likely to remain a `soft security' actor or could this change?
	\item Does the creation of a High representative and the European External Action Service mean that the EU can be considered a fully functional foreign policy actor? What factors might limit the EU's ability to act as such?
	\item How can we explain the decision to enlarge the EU?
	\item Is enlargement a politicised process? Has this changed over time? 
	\item What does Schimmelfennig mean when he talks about rhetorical action as an explanation of EU enlargement?
	\item What role does conditionality play in the enlargement process? How has conditionality evolved over time?
\end{itemize}

\noindent Key reading

\begin{itemize}
	\item Cini, M. and Borragán, N.P.S. 2016. European Union Politics. Oxford University Press. Chapter 17-18.
	\item Hix, S. and H\o yland, B. 2011. The political system of the European Union. Palgrave Macmillan. Chapter 12.
	\item European Council, 2003. A secure Europe in a Better World. Here:	\url{http://eeas.europa.eu/csdp/about-csdp/european-security-strategy/index_en.htm}
	\item European Council, 2008. Report on the Implementation of the European Security Strategy: Providing Security in a Changing World. Here:	\url{http://eeas.europa.eu/csdp/about-csdp/european-security-strategy/index_en.htm}
	\item Schimmelfennig, F. 2001. The community trap: Liberal norms, rhetorical action, and the Eastern enlargement of the European Union. International Organization, 55(1), pp.47–80.
\end{itemize}

\noindent Further reading

\begin{itemize}
	\item Rosén, G. 2015. EU Confidential: The European Parliament's Involvement in EU Security and Defence Policy. JCMS: Journal of Common Market Studies, 53(2), pp.383-398.
	\item Marangoni, A.C. and Raube, K. 2014. Virtue or vice? The coherence of the EU’s external policies. Journal of European Integration, 36(5), pp.473-489.
	\item Keukeleire, S. and Delreux, T. 2015. Competing structural powers and challenges for the EU's structural foreign policy. Global Affairs, 1(1), pp.43-50.
	\item Dinan, D. 2010. Ever Closer Union: An Introduction to European Integration London: Palgrave Macmillan, Chapters 16-18.
	\item Bache, I., George, S. and Bulmer, S. 2011. Politics in the European Union. Oxford: Oxford University Press Chapters 32-34.
	\item Bickerton, C.J., Irondelle, B. and Menon, A. 2011. Security Co-operation beyond the Nation-State: The EU’s Common Security and Defence Policy. Journal of Common Market Studies 49, pp.1-21.
	\item Bieler, A. 2002. The Struggle Over EU Enlargement: A Historical Materialist Analysis of European Integration. Journal of European Public Policy 9(4), pp.575-597.
	\item Moravcsik, A. and Vachudova, M.A. 2003. National Interests, State Power and EU Enlargement. East European Politics and Societies 17, pp.42-57.
	\item Oberthur, S. and Gehring, T. 2006. Institutional interaction in global environmental governance: synergy and conflict among international and EU policies. Cambridge: MIT Press.
	\item Poletti, A. 2010. Drowning Protection in the Multilateral Bath: WTO Judicialisation and European Agriculture in the Doha Round. The British Journal of Politics and International Relations 12, pp.615-33.
	\item Posner, E. 2009. Making Rules for Global Finance: Transatlantic Regulatory Cooperation at the Turn of the Millennium. International Organization 63, pp.665-99. 
	\item Sbragia, A. 2010. The EU, the US and trade policy: competitive interdependence in the management of globalization. Journal of European Public Policy 17, pp.368-82.
	\item Holzinger, K. and Schimmelfennig, F. 2012. Differentiated Integration in the European Union: Many Concepts, Sparse Theory, Few Data. Journal of European Public Policy, 19(2), pp.292–305.
	\item Nguyen, E.S. 2008. Drivers and Brakemen: State Decisions on the Road to European Integration. European Union Politics, 9(2), pp.269–293.
	\item Schimmelfennig, F. and Sedelmeier, U. 2002. Theorizing EU enlargement: research focus, hypotheses, and the state of research. Journal of European Public Policy, 9(4), pp.500–528.
	\item Schimmelfennig, F., Engert, S. and Knobel, H. 2003. Costs, commitment and compliance: the impact of EU democratic conditionality on Latvia, Slovakia and Turkey. JCMS: Journal of Common Market Studies, 41(3), pp.495–518.
	\item White, S., McAllister, I., Light, M. 2002. Enlargement and the New Outsiders. Journal of Common Market Studies 40(1), pp.135-153. 
	\item Young, A.R. 2007. Trade Politics Ain’t What It Used to Be: The European Union in the Doha Round. Journal of Common Market Studies 45(4), pp.789-811.
\end{itemize}