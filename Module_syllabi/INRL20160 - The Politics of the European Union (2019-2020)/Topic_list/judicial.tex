\subsubsection*{Judicial politics and compliance}

Study questions

\begin{itemize}
	\item Describe the structure of the European Court of Justice and the different procedures the ECJ has at its disposal to enforce EU law?
	\item What role does the European Court of Justice play in the political system of the EU? What powers does it yield and what are the limits of these powers?
	\item What do we mean by the term `judicial activism'? Does the ECJ engage in judicial activism often? How have member states reacted to examples of judicial activism by the ECJ?
	\item What difficulties might arise when Member States transpose EU laws into national laws?
	\item We observe significant variation in the levels of compliance with EU law both between Member States and over time. What can explain this variation?
	\item Why might member states fail to comply with EU legislation? What incentives are involved in compliance and non-compliance?
\end{itemize}

\noindent Key reading
	
\begin{itemize}
	\item Cini, M. and Borragán, N.P.S. 2019. European Union Politics. Oxford University Press. Chapter 13.
	\item Hix, S. and H\o yland, B. 2011. The political system of the European Union. Palgrave Macmillan. Chapter 4.
	\item Angelova, M., Dannwolf, T. and K{\"o}nig, T. 2012. How Robust are Compliance Findings. Journal of European Public Policy 19(8), pp.1269-91.
\end{itemize}

\noindent Further reading
	
\begin{itemize}
	\item Closa, C. 2018. The politics of guarding the Treaties: Commission scrutiny of rule of law compliance, Journal of European Public Policy
	\item Jaremba, U. and Mayoral, J.A. 2019. The Europeanization of national judiciaries: definitions, indicators and mechanisms, Journal of European Public Policy, 26(3), pp.386-406
	\item Peritz, L. 2018. Obstructing integration: Domestic politics and the European Court of Justice. European Union Politics, 19(3), pp.427–457.
	\item Hartlapp, M. 2018. Power Shifts via the Judicial Arena: How Annulments Cases between EU Institutions Shape Competence Allocation. JCMS: Journal of Common Market Studies, 56: pp.1429–1445.
	\item Kleine, M. 2014. Informal Governance in the European Union. Journal of European Public Policy, 21(2), pp.303-314.
	\item Zhelyazkova, A. and Yordanova, N. 2015. Signalling compliance: The link between notified EU directive implementation and infringement cases. European Union Politics 16(3), pp.408-428
	\item Carrubba, C.J., Gabel, M. and Hankla, C. 2008. Judicial behavior under political constraints: Evidence from the European Court of Justice. American Political Science Review, 102(04), pp.435-452. 
	\item Sweet, A.S. and Brunell, T. 2012. The European Court of Justice, state noncompliance, and the politics of override. American Political Science Review, 106(01), pp.204-213.
	\item Alter, K.J. 2000. The European Union's legal system and domestic policy: spillover or backlash?. International Organization, 54(03), pp.489-518.
	\item Bache, I., George, S. and Bulmer, S. 2011. Politics in the European Union Oxford: Oxford University Press, Chapter 23	
	\item Chalmers, D., Davies, G. and Monti, G. 2010. European Union Law. Cambridge University Press, Chapter 4.
	\item Garrett, G. 1995. The Politics of Legal Integration in the European Union. International Organization 49(1), pp.171-81.
	\item Luetgert, B. and Dannwolf, T. 2009. Mixing Method: A Nested Analysis of EU Member State Transposition Patterns. European Union Politics 10/3, pp.307-34.
	\item Thomson, R. 2010. Opposition through the Back Door in the Transposition of EU Directives. European Union Politics 12, pp.337-57.
	\item Thomson, R., Torenvlied, R., and Arregui, J. 2007. The Paradox of Compliance: Infringements and Delays in Transposing European Union Directives. British Journal of Political Science 37, pp.685-709.
	\item K{\"o}nig, T., and M{\"a}der, L. 2013. Non-conformable, partial and conformable transposition: A competing risk analysis of the transposition process of directives in the EU15. European Union Politics, 14(1), pp.46-69.
	\item K{\"o}nig, T. and M{\"a}der, L. 2014. The Strategic Nature of Compliance: An Empirical Evaluation of Law Implementation in the Central Monitoring System of the European Union. American Journal of Political Science, 58, pp.246–263. 
	\item Steunenberg, B. 2006. Turning Swift Policy-making into Deadlock and Delay National Policy Coordination and the Transposition of EU Directives. European Union Politics, 7(3), pp.293-319.
\end{itemize}
