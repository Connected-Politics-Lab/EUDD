\subsubsection*{Legislative politics}

Study questions

\begin{itemize}
	\item Describe the different legislative procedures used in the EU. How do they differ from one another? 
	\item Describe the internal structure of the Council of Ministers. How are legislative powers and influence distributed between 1) the member states in terms of voting power? and 2) between the different levels of negotiation within the Council?
	\item Describe the internal structure of the European Parliament. What are the main dimensions of political conflict in the EP? How stable are coalitions between the EP party groups?
\end{itemize}

\noindent Key reading

\begin{itemize}
	\item Cini, M. and Borragán, N.P.S. 2019. European Union Politics. Oxford University Press. Chapter 11 (part about Council of Ministers), Chapter 12, 16
	\item Hix, S. and	Hoyland B. 2011. The political system of the European Union. Palgrave, Chapter 3. 
	\item Hix, S., Noury, A. and Roland G. 2009. Voting Patterns and Alliance Formation in the European Parliament. Philosophical Transactions of the Royal Society B, 364: 821-831.
	\item Cross, J.P. and Hermansson, H. 2017. Legislative amendments and informal politics in the European Union: A text reuse approach. European Union Politics, 18(4), pp.581-602.
\end{itemize}

\noindent Further reading

\begin{itemize}
	\item Wratil, C. 2018. Modes of government responsiveness in the European Union: Evidence from Council negotiation positions. European Union Politics, 19(1), pp.52–74.
	\item Thomson, R. and Hosli, M. 2006. Who Has Power in the EU? Journal of Common Market Studies 44(2), pp.391-417.
	\item Huhe, N., Naurin, D., and Thomson, R. 2018. The evolution of political networks: Evidence from the Council of the European Union. European Union Politics, 19(1), pp.25–51.
	\item Rasmussen, A. and Reh, C. 2013. The consequences of concluding codecision early: trilogues and intra-institutional bargaining success. Journal of European Public Policy, 20(7), pp.1006-1024.
	\item Rasmussen, A. 2011. Early conclusion in bicameral bargaining: Evidence from the co-decision legislative procedure of the European Union. European Union Politics, 12(1), pp.41-64.
	\item Rasmussen, A. 2008. The EU Conciliation Committee One or Several Principals?. European Union Politics, 9(1), pp.87-113.
	\item Rasmussen, A. 2012. Twenty Years of Co-decision Since Maastricht: Inter-and Intrainstitutional Implications. Journal of European Integration, 34(7), pp.735-751.
	\item Reh, C., Héritier, A., Bressanelli, E. and Koop, C. 2013. The informal politics of legislation explaining secluded decision making in the European Union. Comparative Political Studies, 46(9), pp.1112-1142.
	\item M{\"u}hlb{\"o}ck, M., \& Rittberger, B. 2015. The Council, the European Parliament, and the paradox of inter-institutional cooperation. European Integration online Papers (EIoP), 19.
	\item Brandsma, G. J. 2015. Co-decision after Lisbon: The politics of informal trilogues in European Union lawmaking. European Union Politics, 16(2), pp.300-319
	\item Dinan, D. 2010. Ever Closer Union: An Introduction to European Integration London: Palgrave Macmillan. Chapter 9.
	\item Corbett, R., Jacobs, F. and Shackleton, M. 2005. The European Parliament. London: John Harper 
	\item Costello, R. and Thomson, R. 2010. The Policy Impact of Leadership in Committees: Rapporteurs’ Influence on the European Parliament’s Opinions. European Union Politics 11(1), pp.1-26.
	\item Cross, J.P. 2013. Everyone’s a winner (almost): Bargaining success in the Council of Ministers of the European Union. European Union Politics, 14(1), pp.70–94.
	\item Cross, J.P. 2012. Interventions and negotiation in the Council of Ministers of the European Union. European Union Politics, 13(1), pp.47–69.
	\item Hayes-Renshaw, F., Van Aken, W. and Wallace, H. 2006. When and Why the EU Council of Ministers Votes Explicitly. Journal of Common Market Studies 44: 161-94.
	\item Hix, S., Noury, A. and Roland, G. 2007. Democratic Politics in the European Parliament. Cambridge: Cambridge University Press, Chapters 1, 5, 6, 9. 
	\item Judge, D. and Earnshaw, D. 2008. The European Parliament. London: Palgrave. (2nd ed.).
	\item K{\"o}nig, T, Lindburg, B, Lechner, S. and Pohlmeier, W. 2007. Bicameral Conflict Resolution in the European Union: An Empirical Analysis of Conciliation Committee Bargains. British Journal of Political Science 37, pp.281-312.
	\item Kreppel, A. 2002. The European Parliament and Supranational Party System: A Study in Institutional Development. Cambridge: Cambridge University Press. 
	\item Mattila, M. 2009. Roll Call Analysis of Voting in the EU Council of Ministers after the 2004 Enlargement. European Journal of Political Research 48, pp.840-57.
	\item McElroy, G. and Benoit, K. 2007. Party Groups and Policy Positions in the European Parliament. Party Politics 13, pp.5-28.
	\item McElroy, G. and Benoit, K. 2010. Party Policy and Group Affiliation in the European Parliament. British Journal of Political Science 40(2), pp.377-98.
	\item Naurin, D. and Lindahl, R. 2010. Out in the Cold? Flexible Integration and the Political Status of Euro Opt-Outs. European Union Politics 11(4), pp.485-509.
	\item Thomson, R. 2008. The Relative Power of Member States in the Council: Large and Small, Old and New. In Naurin, D. and Wallace, H. (eds.) Unveiling the Council of the European Union: Games Governments Play in Brussels. London: Palgrave Macmillan, pp.238-60.
	\item Thomson, R. 2009. Actor Alignments in the European Union Before and After Enlargement. European Journal of Political Research 48, pp.756-81.
	\item Ringe, N. 2005. Policy Preference Formation in Legislative Politics: Structures, Actors, and Focal Points. American Journal of Political Science 49(4), pp.731–746.
\end{itemize}