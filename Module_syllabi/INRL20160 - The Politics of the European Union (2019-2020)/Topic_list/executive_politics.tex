\subsubsection*{Executive politics and writing workshops}

Study questions

\begin{itemize}
	\item How is the European Council structured? What strengths and weaknesses arise from this structure in terms of its ability to influence the direction of the EU?
	\item How is the European Commission structured? How does it fit into the institutional structure of the EU? 
	\item Does the institutional structure of the Commission help or hinder its ability to make policy?
	\item What roles do the European Council and Commission play in the political system of the EU? How do their roles differ?
	\item Explain principle-agent theory. Does this theory successfully describe the sources of Commission power and influence in EU politics?
\end{itemize}

\noindent Key reading

\begin{itemize}
	\item Cini, M. and Borragán, N.P.S. 2019. European Union Politics. Oxford University Press. Chapters 10, Chapter 11 (part about European Council)
	\item Hix, S. and H\o yland, B. 2011. The political system of the European Union. Palgrave Macmillan. Chapter 2. 
	\item Pollack, M A. 1997. Delegation, Agency, and Agenda Setting in the European Community. International Organization 51(1), pp.99–134.
\end{itemize}

\noindent Further reading

\begin{itemize}
	\item Rauh, C. 2019. EU politicization and policy initiatives of the European Commission: the case of consumer policy, Journal of European Public Policy, 26(3), pp.344-365
	\item Delreux, T. and Adriaensen, J. eds., 2017. The Principal Agent Model and the European Union. Basingstoke: Palgrave Macmillan.
	\item Delreux, T. and Adriaensen, J. 2017. Twenty years of principal-agent research in EU politics: how to cope with complexity?. European Political Science, pp.1-18.
	\item Concei\c{c}\~{a}o-Heldt, E. 2010. Who controls whom? Dynamics of power delegation and agency losses in EU trade politics. JCMS: Journal of Common Market Studies, 48(4), pp.1107-1126.
	\item Nugent, N. and Rhinard, M. 2016. Is the European Commission Really in Decline?. JCMS: Journal of Common Market Studies. 54(5), pp.1199-1215.
	\item Brown, S.A. 2016. The Commission and the Crisis: Chief Loser or Unexpected Winner?. In The European Commission and Europe's Democratic Process, pp.69-78. Palgrave Macmillan UK.
	\item Bauer, M.W. and Becker S. 2014. The unexpected winner of the crisis: the European Commission’s strengthened role in economic governance. Journal of European Integration. 36(3), pp.213–29.
	\item da Concei\c{c}\~{a}o-Heldt, E. 2016. Why the European Commission is not the ``unexpected winner" of the Euro crisis: A comment on Bauer and Becker. Journal of European Integration, 38(1), pp.95-100.
	\item Bauer, M.W. and Becker, S. 2016. Absolute Gains Are Still Gains: Why the European Commission Is a Winner of the Crisis, and Unexpectedly So. A Rejoinder to Eugénia da Concei\c{c}\~{a}o-Heldt. Journal of European Integration, 38(1), pp.101-106.
	\item Fabbrini, S. and Puetter, U. 2016. Integration without supranationalisation: studying the lead roles of the European Council and the Council in post-Lisbon EU politics. Journal of European Integration, 38(5), pp.481-495.
	\item Bailer, S. 2014. An Agent Dependent on the EU Member States? The Determinants of the European Commission’s Legislative Success in the European Union, Journal of European Integration, 36(1), pp.37-53.
	\item Egeberg, M. 2006. Executive Politics as Usual: Role Behaviour and Conflict Dimensions in the College of European Commissioners. Journal of European Public Policy 13(1), pp.1-15.
	\item Franchino, F. 2009. Experience and the Distribution of Portfolio Payoffs in the European Commission. European Journal of Political Research 48(1), pp.1–30.
	\item Franchino, F. 1999. Delegating Powers in the European Community. British Journal of Political Science 34(2), pp.269–93.
	\item Dinan, D. 2010. Ever Closer Union: An Introduction to European Integration. London: Palgrave Macmillan, Chapters 7-8.
	\item Bunse, S. 2009. Small States and EU Governance: Leadership through the Council Presidency. Basingstoke: Palgrave Macmillan.
	\item Cini, M. 1996. The European Commission: Leadership, Organisation and Culture in the EU Administration. Manchester: Manchester University Press.
	\item Franchino, F. 2009. Experience and the distribution of portfolio payoffs in the European Commission. European Journal of Political Research 48(1), pp.1-30.
	\item H{\"a}ge, F.M. 2008. Who Decides in the Council of the European Union? Journal of Common Market Studies 46(3), pp.533-58.
	\item H{\"a}ge, F.M. 2016. Political attention in the Council of the European Union: A new dataset of working party meetings, 1995–2014. European Union Politics 17(4), pp.683–703
	\item Hayes-Renshaw, F. and Wallace H. 2006. The Council of Ministers. Basingstoke: Palgrave Macmillan.
	\item Hooghe, L. 1999. Images of Europe: Orientations to European Integration Among Senior Officials of the Commission. British Journal of Political Science 29:345–367.
	\item Hooghe, L. 2002. The European Commission and the Integration of Europe. New York: Cambridge University Press. 
	\item Hooghe, L. 2005. Several Roads Lead to International Norms, But Few via International Socialization: A Case Study of the European Commission. International Organization 59: 861-898.
	\item Rasmussen, A. 2007. Challenging the Commission’s Right of Initiative? Conditions for Institutional Change and Stability. West European Politics 30(2), pp.244-64.
	\item Tallberg, J. 2008. Bargaining Power in the European Council. Journal of Common Market Studies 46(3), pp.685-708.
	\item Thomson, R. 2008. National Actors in International Organizations: The Case of the European Commission. Comparative Political Studies 41: 169-92.
	\item Tsebelis, G. and Garrett G. 2000. Legislative Politics in the EU. European Union Politics 1, pp.9–36.
	\item Wallace, H., Pollock, M, and Young, A.R. 2010. Policy-Making in the European Union. Oxford: Oxford University Press, Chapter 4.
	\item Wonka, A. 2007. Technocratic and Independent? The Appointment of European Commissioners and its Policy Implications. Journal of European Public Policy 14(2), pp.169-89.
	\item Slapin, J.B. 2006. Who Is Powerful?: Examining Preferences and Testing Sources of Bargaining Strength at European Intergovernmental Conferences. European Union Politics 7 (1), pp.51–76.
	\item Hug, S, and K{\"o}nig, T. 2002. In View of Ratification: Governmental Preferences and Domestic Constraints at the Amsterdam Intergovernmental Conference. International Organization 56 (02), pp.447–76.
\end{itemize}