\subsubsection*{Elections, referenda, and public opinion}

Study questions

\begin{itemize}
	\item How has public opinion towards the EU changed over time? What explains this variation?
	\item What different theories have attempted to explain public attitudes towards the EU? Which of these theories is most convincing and why?
	\item What do we mean when we describe the EP elections as second order in nature? Is this a fair description of EP elections at this point in the history of the EU?
\end{itemize}
	
\noindent Key reading
	
\begin{itemize}
	\item Explore EUOPINIONS website: \url{http://eupinions.eu/de/home/}
	\item Cini, M. and Borragán, N.P.S. 2019. European Union Politics. Oxford University Press. Chapter 15
	\item Hix, S. and H\o yland, B. 2011. The Political System of the European Union. Palgrave Macmillan (3rd edition), Chapters 5-6.
	\item Boomgaarden, H.G., Schuck, A.R., Elenbaas, M. and De Vreese, C.H. 2011. Mapping EU attitudes: Conceptual and empirical dimensions of Euroscepticism and EU support. European Union Politics, 12(2), pp.241-266.
	\item Beach, D., Hansen, K., and Larsen, M. 2018. How Campaigns Enhance European Issues Voting During European Parliament Elections. Political Science Research and Methods, 6(4), pp.791-808.
\end{itemize}

\noindent Further reading
	
\begin{itemize}
	\item Aldrich, A. S. 2018. National Political Parties and Career Paths to the European Parliament. JCMS: Journal of Common Market Studies, 56: pp.1283–1304.
	\item Hix, S. and Marsh, M., 2007. Punishment or Protest? Understanding European Parliament Elections. Journal of Politics 69(2), pp.495-510.
	\item Vasilopoulou, S. and Talving, L. 2018. Opportunity or threat? Public attitudes towards EU freedom of movement. Journal of European Public Policy, pp.1-19.
	\item Rooduijn, M., 2015. The rise of the populist radical right in Western Europe. European View, 14(1), pp.3-11.
	\item Jeannet, A.M., 2018. Internal migration and public opinion about the European Union: a time series cross-sectional study. Socio-Economic Review.
	\item Kostadinova, P. and Giurcanu, M. 2018. Capturing the legislative priorities of transnational Europarties and the European Commission: A pledge approach, European Union Politics, 19(2), pp.363–379.
	\item Senninger, R. and Bischof, D. 2018. Working in unison: Political parties and policy issue transfer in the multilevel space, European Union Politics, 19(1), pp.140–162.
	\item Sorace, M. 2018. Legislative Participation in the EU: An analysis of questions, speeches, motions and declarations in the 7th European Parliament. European Union Politics, 19(2), pp.299–320.
	\item Frid-Nielsen, S. S. 2018. Human rights or security? Positions on asylum in European Parliament speeches. European Union Politics, 19(2), pp.344–362.
	\item Cavallaro, M., Flacher, D., and Zanetti, M. A. 2018. Radical right parties and European economic integration: Evidence from the seventh European Parliament. European Union Politics, 19(2), pp.321–343.
	\item Tuttnauer, O., 2018. If you can beat them, confront them: Party-level analysis of opposition behavior in European national parliaments. European Union Politics, 19(2), pp.278-298.
	\item Hernández, E. 2018. Democratic discontent and support for mainstream and challenger parties: Democratic protest voting. European Union Politics, 19(3) pp.458–480
	\item Schulte-Cloos, J. 2018. Do European Parliament elections foster challenger parties' success on the national level?. European Union Politics, p.1465116518773486.
	\item Anderson, C.J. and Hecht, J.D. 2018. The preference for Europe: Public opinion about European integration since 1952. European Union Politics, 19(4), pp.617-638.
	\item Beach, D., Hansen, K., and Larsen, M. 2018. How Campaigns Enhance European Issues Voting During European Parliament Elections. Political Science Research and Methods, 6(4), 791-808. 
	\item Lefkofridi, Z. and Katsanidou, A. 2018. A Step Closer to a Transnational Party System? Competition and Coherence in the 2009 and 2014 European Parliament. JCMS: Journal of Common Market Studies, 56: 1462–1482.
	\item Sorace, M. 2017. The European Union democratic deficit: Substantive representation in the European Parliament at the input stage. European Union Politics, p.1465116517741562.
	\item Nielsen, J.H. and Franklin, M.N. 2017. The 2014 European Parliament Elections: Still Second Order?. In The Eurosceptic 2014 European Parliament Elections (pp.1-16). Palgrave Macmillan UK.
	\item McEvoy, C. 2016. The Role of Political Efficacy on Public Opinion in the European Union. JCMS: Journal of Common Market Studies, 54(5), pp.1159-1174
	\item Hobolt, S.B. and de Vries, C.E. 2016. Public Support for European Integration. Annual Review of Political Science, 19, pp.413-432.
	\item Hobolt, S.B. 2016. The Brexit vote: a divided nation, a divided continent. Journal of European Public Policy, 23(9), pp.1259-1277.
	\item Braun, D., Hutter, S. and Kerscher, A. 2016. What type of Europe? The salience of polity and policy issues in European Parliament elections. European Union Politics, 17(4), pp.570-592.
	\item Baglioni, S. and Hurrelmann, A. 2016. The Eurozone crisis and citizen engagement in EU affairs. West European Politics, 39(1), pp.104-124.
	\item Oliver, T. 2016. European and international views of Brexit. Journal of European Public Policy, pp.1-8.
	\item Nielsen, J.H. 2016. Personality and Euroscepticism: The Impact of Personality on Attitudes Towards the EU. JCMS: Journal of Common Market Studies. 54(5), pp.1175–1198.
	\item Smith, K.E. 2016. Left out in the cold: Brexit, the EU and the perils of Trump’s world. LSE Brexit. \url{http://eprints.lse.ac.uk/68777/}.
	\item Hix, S., Hagemann, S. and Frantescu, D. 2016. Would Brexit matter? The UK’s voting record in the Council and the European Parliament. \url{http://eprints.lse.ac.uk/66261/}.
	\item Chopin, T. and Lequesne, C. 2016. Differentiation as a double-edged sword: member states’ practices and Brexit. International Affairs, 92(3), pp.531-545.
	\item Dagnis Jensen, M. and Snaith, H. 2016. When politics prevails: the political economy of a Brexit. Journal of European Public Policy, pp.1-9.
	\item Hobolt, S.B. and de Vries, C.E. 2016. Public Support for European Integration. Annual Review of Political Science, 19, pp.413-432.
	\item Hobolt, S. and Spoon, J. 2012. Motivating the European voter: parties, issues and campaigns in European Parliament elections. European Journal of Political Research 51(6), pp.701-727.
	\item Hobolt, Sara B. 2012. Citizen Satisfaction with Democracy in the European Union*. JCMS: Journal of Common Market Studies 50 (s1), pp.88–105.
	\item Armingeon, K, and B Ceka. 2014. The Loss of Trust in the European Union During the Great Recession Since 2007: the Role of Heuristics From the National Political System. European Union Politics 15 (1), pp.82–107.
	\item Ceka, B. 2013. The Perils of Political Competition: Explaining Participation and Trust in Political Parties in Eastern Europe. Comparative Political Studies 46 (12), pp.1610–35.
	\item Lefkofridi, Z. and Katsanidou, A. 2014. Multilevel representation in the European Parliament. European Union Politics, 15(1), pp.108-131.
	\item Kopecky, P. and Mudde, C. 2002. The Two Sides of Euroscepticism: Party Positions on European Integration in East Central Europe. European Union Politics, 3(3), pp.297–326.
	\item Franklin, M. and Hobolt, S.B. 2011. The Legacy of Lethargy: How Elections for the European Parliament Depress Turnout. Electoral Studies 30, pp.67-76.
	\item Gabel, M. 1998. Public Support for European Integration: An Empirical Test of Five Theories. Journal of Politics 60(3), pp.333–354.
	\item Garry, J. and Tilley, J. 2009. The Macroeconomic Factors Conditioning the Impact of Identity on Attitudes towards the EU. European Union Politics 10(3), pp.361–379.
	\item Hellstr{\"o}m, J. 2008. Who Leads, Who Follows? Re-examining the Party-Electorate Linkages on European Integration. Journal of European Public Policy 15(8), pp.1127-44.
	\item Hix, S. and Marsh, M. 2011. Second-order effects plus pan-European political swings: An analysis of European Parliament elections across time. Electoral Studies 30, pp.4–15.
	\item Hobolt, S.B. 2009. Europe in Question: Referendums on European Integration. Oxford University Press.
	\item Hobolt, S.B. 2012. Citizens satisfaction with democracy in the European Union. Journal of Common Market Studies 50 (1), pp.88-105.
	\item Hobolt, S.B., Spoon, J. and Tilley, J. 2009. A Vote Against Europe? Explaining Defection at the 1999 and 2004 European Parliament Elections. British Journal of Political Science 39(1), pp.93-115.
	\item Hobolt, S.B. and Wittrock, J. 2011. The second-order election model revisited: An experimental test of vote choices in European Parliament elections. Electoral Studies 30, pp.29–40.
	\item Hooghe, L. and Marks, G. 2006. Calculation, Community and Cues: Public Opinion on European Integration. European Union Politics 6(4), pp.419–43.
	\item Steenbergen, M.R., Edwards, E.E. and de Vries, C.E. 2007. Who’s Cueing Whom? Mass- Elite Linkages and the Future of European Integration. European Union Politics 8(1), pp.39-49.
\end{itemize}