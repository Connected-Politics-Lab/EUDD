\subsubsection*{Interest representation}

Study questions

\begin{itemize}
	\item What are interest groups and why do we study them in the EU context?
	\item Describe the different access point through which lobbyists attempt to influence EU policy making. Do the institutional structures of the Commission, EP, and Council aid or hinder lobbyists in their lobbying attempts?
	\item What different types of factors have been identified as influencing lobbying success in the EU?
	\item What attempts have been made to regulate EU lobbying? Have these attempts been successful?
\end{itemize}
	
\noindent Key reading

\begin{itemize}
	\item Cini, M. and Borragán, N.P.S. 2019. European Union Politics. Oxford University Press. Chapter 14.
	\item Hix, S. and H\o yland, B. 2011. The political system of the European Union. Palgrave Macmillan. Chapter 7.
	\item Bouwen, P. 2004. Exchanging access goods for access: A comparative study of business lobbying in the European Union Institutions. European Journal of Political Research, 43: 337–369 
	\item D{\"u}r, A., Bernhagen, P., and Marshall, D. 2015. Interest Group Success in the European Union When (and Why) Does Business Lose?. Comparative Political Studies, 48(8), pp.951-983.
\end{itemize}

\noindent Further reading

\begin{itemize}
	\item European Commission for Democracy Through Law (Venice Commission). 2013. Report on the Role of Extra-Institutional Actors in the Democratic System.
	\item Coen, D. 2007. Empirical and theoretical studies in EU lobbying. Journal of European Public Policy 14(3), pp.333-345.
	\item Marshall, D. 2010. Who to lobby and when: Institutional determinants of interest group strategies in European Parliament committees. European Union Politics, 11(4), pp.553-575.
	\item Bunea, A. 2018. Regulating European Union lobbying: in whose interest?. Journal of European Public Policy, pp.1-21.
	\item Fl{\"o}the, L. and Rasmussen, A., 2018. Public voices in the heavenly chorus? Group type bias and opinion representation. Journal of European Public Policy, pp.1-19.
	\item Judge, A. and Thomson, R., 2018. The responsiveness of legislative actors to stakeholders’ demands in the European Union. Journal of European Public Policy, pp.1-20.
	\item Wonka, A., De Bruycker, I., De Bièvre, D., Braun, C., and Beyers, J. 2018. Patterns of Conflict and Mobilization: Mapping Interest Group Activity in EU Legislative Policymaking. Politics and Governance, 6(3), 136-146.
	\item Hollman, M. and Murdoch, Z. 2018. Lobbying cycles in Brussels: Evidence from the rotating presidency of the Council of the European Union. European Union Politics, 19(4), pp.597-616.
	\item R\o ed, M., and Wøien Hansen, V. (2018) Explaining Participation Bias in the European Commission's Online Consultations: The Struggle for Policy Gain without too Much Pain. JCMS: Journal of Common Market Studies, 56: 1446–1461.
	\item De Bruycker, I. 2016. Power and position: Which EU party groups do lobbyists prioritize and why? Party Politics 22(4), pp.552-562.
	\item van der Graaf, A., Otjes, S. and Rasmussen, A. 2016. Weapon of the weak? The social media landscape of interest groups. European Journal of Communication, 31(2), pp.120-135.
	\item De Bruycker, I. 2016. Pressure and Expertise: Explaining the Information Supply of Interest Groups in EU Legislative Lobbying. JCMS: Journal of Common Market Studies 54(3), pp.599-616.
	\item Beyers, J., De Bruycker, I. and Baller, I. 2015. The alignment of parties and interest groups in EU legislative politics. A tale of two different worlds?. Journal of European Public Policy, 22(4), pp.534-551.
	\item Kl{\"u}ver, H., Mahoney, C. and Opper, M. 2015. Framing in context: how interest groups employ framing to lobby the European Commission. Journal of European Public Policy, 22(4), pp.481-498.
	\item Kl{\"u}ver, H. and Mahoney, C. 2015. Measuring interest group framing strategies in public policy debates. Journal of Public Policy, 35(02), pp.223-244.
	\item Eising, R., Rasch, D. and Rozbicka, P. 2015. Institutions, policies, and arguments: context and strategy in EU policy framing. Journal of European Public Policy, 22(4), pp.516-533.
	\item Bor{\"a}ng, F. and Naurin, D. 2015. `Try to see it my way!' Frame congruence between lobbyists and European Commission officials. Journal of European Public Policy 22(4), pp.499-515.
	\item Beyers, J., Bonafont, L. C., D{\"u}r, A., Eising, R., Fink-Hafner, D., Lowery, D., ... and Naurin, D. 2014. The INTEREURO project: Logic and structure. Interest Groups \& Advocacy, 3(2), 126-140.
	\item Chari, R. Murphy, G. and Hogan, J. 2007. Regulating Lobbyists: A Comparative Analysis of the USA, Canada, Germany and the European Union, The Political Quarterly, 78(3), pp.422-438.
	\item Coen, D. 1998. The European business interest and the nation state: large-firm lobbying in the European Union and member states. Journal of Public Policy, 18(01), pp.75-100.
	\item Mahoney, C. and Baumgartner, F. 2008. Converging perspectives on interest group research in Europe and America. West European Politics, 31(6), pp.1253–1273.
	\item Beyers, J. 2008. Policy Issues, Organizational Format and the Political Strategies of Interest Organizations. West European Politics 31(6), pp.1188-1211.
	\item Beyers, J. and Kerremans, B. 2004. Bureaucrats, Politicians, and Societal Interests: How Is European Policy Making Politicized?” Comparative Political Studies 37(10), pp.1119-1150.
	\item B{\"o}rzel, T. and Heard-Lauréote, K. 2009. Networks in EU Multi-level governance: Concepts and Contributions. Journal of Public Policy 29(2), pp.135-152.
	\item Coen, D and Richardson, J (eds) (2009) Lobbying the European Union, Oxford: Oxford University Press.
	\item D{\"u}r, A. 2008. Interest groups in the European Union: How Powerful are They?” West European Politics 31(6), pp.1212-1230.
	\item Eising, R. 2007. The access of business interests to EU institutions: towards elite pluralism? Journal of European Public Policy 14(3), pp.384-403.
	\item Eising, R. 2008. Interest Group in EU policymaking. Living Reviews in European Governance Vol. 3 (\url{http://europeangovernance.livingreviews.org/Articles/lreg-2008-4/}).
	\item Greenwood, J. 2007a. Interest Representation in the European Union, Hampshire: Palgrave Macmillan.
	\item Greenwood, J. 2007b. Review Article: Organized Civil Society and Democratic Legitimacy in the EU. British Journal of Political Science 37: 333-35. 
	\item Kl{\"u}ver, H. 2010. Measuring Interest Group Influence using Quantitative Text Analysis. European Union Politics 10(4), pp.535- 549.
	\item Mahoney, C. 2008. Brussels versus the Beltway. Advocacy in the United States and the European Union. Washington DC: Georgetown University Press.
	\item Mahoney, C. 2007. Lobbying Success in the United State and the European Union. Journal of Public Policy 27(2), pp.35-56.
	\item Quittkat, C. 2011. The European Commission’s Online Consultations: a Success Story? Journal of Common Market Studies 49(3), pp.653-674.
	\item Skodvin, T., Gullberg, A.T., and Aakre, S. 2010. Target-group influence and political feasibility: the case of climate policy design in Europe. Journal of European Public Policy 17(6), pp.854- 873.
\end{itemize}