\subsubsection*{The Euro-Crisis and its aftermath}

Study questions

\begin{itemize}
	\item What are the origins of the Euro crisis?
	\item Do existing institutions have the capacity to solve the Euro problem?
	\item Is it possible to have a currency union without a fiscal union?
	\item What are the consequences of the Euro crisis for democracy in Europe?
\end{itemize}

\noindent Key reading

\begin{itemize}
	\item Cini, M. and Borragán, N.P.S. 2019. European Union Politics. Oxford University Press. Chapter 26.
	\item Copelovitch, M., Frieden, J. and Walter, S. 2016. The political economy of the euro crisis. Comparative Political Studies, 49(7), pp.811-840.
	\item Tarlea, S., Bailer, S. and Degner, H. 2019. Explaining Governmental Preferences on Economic and Monetary Union Reform. European Union Politics 20(1).
	\item Rodrik, D. 2018. How Democratic Is the Euro? Available here: \url{https://www.project-syndicate.org/commentary/how-democratic-is-the-euro-by-dani-rodrik-2018-06?barrier=accesspaylog}.
\end{itemize}

\noindent Further readings

\begin{itemize}
	\item Wren-Lewis, S. 2018. Should Eurozone central bankers keep quiet about fiscal policy? Available here: \url{https://mainlymacro.blogspot.com/2018/07/should-eurozone-central-bankers-keep.html?m=1}.
	\item Armingeon, K. and Cranmer, S. 2018. Position-taking in the Euro crisis. Journal of European public policy, 25(4), pp.546-566.
	\item Steinbach, A. 2018. EU economic governance after the crisis: revisiting the accountability shift in EU economic governance. Journal of European Public Policy, pp.1-19.
	\item Matthijs, M. and McNamara, K. 2015. The Europe crisis’ theory effect: northern saints, southern sinners, and the demise of the Eurobond, Journal of European Integration 37(5), pp.229–45.
	\item Scharpf, F.W. 2011. Monetary Union, Fiscal Crisis and the Preemption of Democracy. pp.1–46.
	\item Kanthak, L. and Spies, D. C. 2018. Public support for European Union economic policies, European Union Politics, 19(1), pp.97–118.
	\item Degner, H. and Leuffen, D. 2018. Franco-German cooperation and the rescuing of the Eurozone. European Union Politics.
	\item Kanthak, L. and Spies, D. C. 2018. Public support for European Union economic policies. European Union Politics, 19(1), pp.97–118.
	\item Bauhr, M. and Charron, N. 2018. Why support International redistribution? Corruption and public support for aid in the eurozone. European Union Politics, 19(2), pp.233–254.
	\item Finke, D. and Bailer, S. 2018. Crisis bargaining in the European Union: Formal rules or market pressure? European Union Politics.
	\item Verdun, A. 2018. Institutional Architecture of the Euro Area. JCMS: Journal of Common Market Studies, 56, pp.74–84.
	\item Wasserfallen, F, Leuffen, D, and Kudrna, Z. 2019. Analysing European Union Decision-Making during the Eurozone Crisis with New Data. European Union Politics 20(1).
	\item Lehner, T. and Wasserfallen, F. 2019. Political Conflict in the Reform of the Eurozone. European Union Politics 20(1).
	\item Walter, S. 2016. Crisis politics in Europe. Comparative Political Studies 49, pp.841–873.
	\item Camisão, I. 2015. Irrelevant player? The Commission’s role during the Eurozone crisis. Journal of Contemporary European Research 11.
	\item Bauer, MW. and Becker, S. 2014. The unexpected winner of the crisis: The European Commission’s strengthened role in economic governance. Journal of European Integration 36 pp.213–229.
	\item Schimmelfennig, F. 2015. Liberal intergovernmentalism and the euro area crisis. Journal of European Public Policy 22 pp.177–195. 
	\item Lundgren, M, Bailer, S, and Dellmuth, L.M. 2019. Bargaining Success in the Reform of the Eurozone. European Union Politics 20(1).
	\item Degner, H and Leuffen, D. 2019. Franco-German Cooperation and the Rescuing of the Eurozone. European Union Politics 20(1).
	\item Regan, A. 2015. The imbalance of capitalisms in the Eurozone: Can the north and south of Europe converge? Comparative European Politics.
	\item Kaiser, J. and Kleinen-von K{\"o}nigsl{\"o}w, K. 2016. The Framing of the Euro Crisis in German and Spanish Online News Media between 2010 and 2014: Does a Common European Public Discourse Emerge?. JCMS: Journal of Common Market Studies.
	\item Kaiser, J. and Kleinen-von K{\"o}nigsl{\"o}w, K. 2016. Partisan journalism and the issue framing of the Euro crisis: Comparing political parallelism of German and Spanish online news. Journalism.
	\item Ioannou, D., Leblond, P. and Niemann, A. 2015. European integration and the crisis: practice and theory. Journal of European Public Policy, 22(2), pp.155-176.
	\item Braun, D. and Tausendpfund, M. 2014. The impact of the Euro Crisis on citizens’ support for the European Union. Journal of European Integration, 36(3), pp.231-245.
	\item Schimmelfennig, F. 2014. European integration in the euro crisis: The limits of postfunctionalism. Journal of European Integration, 36(3), pp.321-337.
	\item Clements, B., Nanou, K. and Verney, S. 2014. `We no longer love you, but we don’t want to leave you': the Eurozone crisis and popular Euroscepticism in Greece. Journal of European Integration, 36(3), pp.247-265.
	\item Tosun, J., Wetzel, A. and Zapryanova, G. 2014. The EU in crisis: advancing the debate. Journal of European Integration, 36(3), pp.195-211.
	\item Buti, M. and Carnot, N. 2012. The EMU Debt Crisis: Early Lessons and Reforms*. JCMS: Journal of Common Market Studies, 50(6), pp.899–911.
	\item Vilpišauskas, R. 2013. Eurozone crisis and European integration: functional spillover, political spillback?. Journal of European Integration, 35(3), pp.361-373.
	\item De Grauwe, P. and Ji, Y. 2012. Mispricing of Sovereign Risk and Macroeconomic Stability in the Eurozone*. JCMS: Journal of Common Market Studies, 50(6), pp.866–880.
	\item Drudi, F., Durre, A. and Mongelli, F.P. 2012. The Interplay of Economic Reforms and Monetary Policy: The Case of the Eurozone. JCMS: Journal of Common Market Studies, 50(6), pp.881–898.
	\item Eijffinger, S.C.W. 2012. Rating Agencies: Role and Influence of Their Sovereign Credit Risk Assessment in the Eurozone*. JCMS: Journal of Common Market Studies, 50(6), pp.912–921.
	\item Verdun, A. 2012. Introduction to the Symposium: Economic and Monetary Union and the Crisis of the Eurozone. JCMS: Journal of Common Market Studies, 50(6), pp.863–865.
	\item Dawson, M. 2015. The Legal and Political Accountability Structure of `Post-Crisis' EU Economic Governance. JCMS: Journal of Common Market Studies, 53(5), pp.976-993.
\end{itemize}