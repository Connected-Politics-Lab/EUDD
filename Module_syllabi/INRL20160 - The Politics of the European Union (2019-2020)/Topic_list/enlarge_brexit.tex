\subsubsection*{The EU on the world stage and EU enlargement}

Study questions

\begin{itemize}
	\item How can we explain the decision to enlarge the EU?
	\item Is enlargement a politicised process? Has this changed over time? 
	\item What does Schimmelfennig mean when he talks about rhetorical action as an explanation of EU enlargement?
	\item What role does conditionality play in the enlargement process? How has conditionality evolved over time?
\end{itemize}

\noindent Key reading

\begin{itemize}
	\item Cini, M. and Borragán, N.P.S. 2016. European Union Politics. Oxford University Press. Chapter 17-18.
	\item Schimmelfennig, F. 2001. The community trap: Liberal norms, rhetorical action, and the Eastern enlargement of the European Union. International Organization, 55(1), pp.47–80.
	\item Hobolt, S. B. 2016. The Brexit vote: a divided nation, a divided continent. Journal of European Public Policy, 23(9), pp.1259-1277.
	\item Hix, S. 2018. Brexit: Where is the EU–UK Relationship Heading?. JCMS: Journal of Common Market Studies, 56, pp.11–27.
	\item Kirsch, W. 2016. Brexit and the Distribution of Power in the Council of the EU, CEPS Commentaries, pp.1-4. 
	\item Springford, J. 2018. Theresa May's Irish trilemma. \url{https://www.cer.eu/insights/theresa-mays-irish-trilemma}.
\end{itemize}

\noindent Further reading

\begin{itemize}
	\item Patel, O. \& Reh, C. 2016. Brexit: The Consequences for the EU Political System, UCL Constitution Unit Briefing Paper, pp.1-5. 
	\item European Commission. 2019. Withdrawal of the United Kingdom from the EU. Available here: \url{https://ec.europa.eu/taxation_customs/uk_withdrawal_en}
	\item Laffan, B. 2018. Brexit: Re-opening Ireland's English Question. The Political Quarterly, 89, pp.568-575. 
	\item Hix, S. and H\o yland, B. 2011. The political system of the European Union. Palgrave Macmillan. Chapter 12.
	\item Inglehart, R., \& Norris, P. 2016. Trump, Brexit, and the rise of populism: Economic have-nots and cultural backlash. HKS Working Paper No. RWP16-026. Available at SSRN: \url{https://ssrn.com/abstract=2818659}.
	\item Staal, K. 2016. Brexit Implications for Influence on EU Decision Making, The Stability of Regions, Culture, and Institutions VIVES Workshop, University of Leuven, Belgium, pp.1-6.
	\item Bieler, A. 2002. The Struggle Over EU Enlargement: A Historical Materialist Analysis of European Integration. Journal of European Public Policy 9(4), pp.575-597.
	\item Moravcsik, A. and Vachudova, M.A. 2003. National Interests, State Power and EU Enlargement. East European Politics and Societies 17, pp.42-57.
	\item Holzinger, K. and Schimmelfennig, F. 2012. Differentiated Integration in the European Union: Many Concepts, Sparse Theory, Few Data. Journal of European Public Policy, 19(2), pp.292–305.
	\item Nguyen, E.S. 2008. Drivers and Brakemen: State Decisions on the Road to European Integration. European Union Politics, 9(2), pp.269–293.
	\item Schimmelfennig, F. and Sedelmeier, U. 2002. Theorizing EU enlargement: research focus, hypotheses, and the state of research. Journal of European Public Policy, 9(4), pp.500–528.
	\item Schimmelfennig, F., Engert, S. and Knobel, H. 2003. Costs, commitment and compliance: the impact of EU democratic conditionality on Latvia, Slovakia and Turkey. JCMS: Journal of Common Market Studies, 41(3), pp.495–518.
	\item White, S., McAllister, I., Light, M. 2002. Enlargement and the New Outsiders. Journal of Common Market Studies 40(1), pp.135-153. 
	\item Young, A.R. 2007. Trade Politics Ain’t What It Used to Be: The European Union in the Doha Round. Journal of Common Market Studies 45(4), pp.789-811.
	\item Lavery, S., McDaniel, S. and Schmid, D., 2018. Finance fragmented? Frankfurt and Paris as European financial centres after Brexit. Journal of European Public Policy, pp.1-19.
	\item Jennings, W. and Lodge, M. 2018. Brexit, the tides and Canute: the fracturing politics of the British state, Journal of European Public Policy
	\item Qvortrup, M. 2016. Referendums on Membership and European Integration 1972–2015. The Political Quarterly, 87, pp.61-68.
	\item Henderson, A. , Jeffery, C. , Liñeira, R. , Scully, R. , Wincott, D. and Wyn Jones, R. 2016. England, Englishness and Brexit. The Political Quarterly, 87, pp.187-199.
	\item James, S. and Quaglia, L. 2018. The Brexit Negotiations and Financial Services: A Two-Level Game Analysis. The Political Quarterly, 89, pp.560-567.
	\item Renwick, A. , Allan, S. , Jennings, W. , Mckee, R. , Russell, M. and Smith, G. 2018. What Kind of Brexit do Voters want? Lessons from the Citizens’ Assembly on Brexit. The Political Quarterly, 89, pp.649-658.
	\item Laffan, B. 2018. Brexit: Re-opening Ireland's English Question. The Political Quarterly, 89, pp.568-575. 
	\item Chen W, Los B, McCann P, Ortega-Argilés R, Thissen M, van Oort F. 2018. The continental divide? Economic exposure to Brexit in regions and countries on both sides of The Channel. Papers in Regional Science. 97, pp.25–54.
	\item Richardson, J. 2018. Brexit: The EU Policy-Making State Hits the Populist Buffers. The Political Quarterly, 89, pp.118-126.
	\item Richards, L. , Heath, A. and Carl, N. 2018. Red Lines and Compromises: Mapping Underlying Complexities of Brexit Preferences. The Political Quarterly, 89, pp.280-290.
	\item Hantzsche, A. , Kara, A. and Young, G. 2018. The Economic Effects of the UK Government's Proposed Brexit Deal. The World Economy, Accepted Author Manuscript.
	\item Gasiorek, M., Serwicka, I. and Smith, A. 2019. Which Manufacturing Industries and Sectors Are Most Vulnerable to Brexit?. The World Economy, Accepted Author Manuscript.
	\item Manners, I. 2018. Political Psychology of European Integration: The (Re)production of Identity and Difference in the Brexit Debate. Political Psychology, 39, pp.1213-1232.
	\item Warlouzet, L. 2018. Britain at the Centre of European Co-operation (1948–2016). JCMS: Journal of Common Market Studies, 56, pp.955–970.
	\item Dhingra, S., \& Sampson, T. 2016. Life after BREXIT: What are the UK’s options outside the European Union?. CEPBREXIT01. London School of Economics and Political Science, CEP, London, UK.
	\item Goodwin, M. J., \& Heath, O. 2016. The 2016 Referendum, Brexit and the Left Behind: An Aggregate-level Analysis of the Result. The Political Quarterly, 87(3), pp.323-332.
	\item Grant, C. 2016. The Impact of Brexit on the EU, Centre for European Reform, blog post, 24 June 2016. 
	\item Kierzenkowski, R., et al. 2016. The Economic Consequences of Brexit: A Taxing Decision, OECD Economic Policy Papers, No. 16, OECD Publishing, Paris.
	\url{http://dx.doi.org/10.1787/5jm0lsvdkf6k-en}.
	\item Kaufmann, E. 2016. It’s NOT the economy, stupid: Brexit as a story of personal values. British Politics and Policy at LSE.
	\item Whitman, R. G. 2016. Brexit or Bremain: what future for the UK's European diplomatic strategy?. International Affairs, 92(3), pp.509-529.
	\item Ottaviano, G. I. P., Pessoa, J. P., Sampson, T., \& Van Reenen, J. 2014. Brexit or Fixit? The trade and welfare effects of leaving the European Union.
	\item Oliver, T. 2016. European and international views of Brexit. Journal of European Public Policy, 23(9), pp.1321-1328.
	\item Menon, A., \& Salter, J. P. 2016. Brexit: initial reflections. International Affairs, 92(6), pp.1297-1318.
	\item Jensen, M. D., \& Snaith, H. 2016. When politics prevails: the political economy of a Brexit. Journal of European Public Policy, 23(9), pp.1302-1310.
	\item Oliver, T., \& Williams, M. J. 2016. Special relationships in flux: Brexit and the future of the US—EU and US—UK relationships. International Affairs, 92(3), pp.547-567.
\end{itemize}