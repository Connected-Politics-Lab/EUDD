\documentclass[a4paper,12pt]{article}

\usepackage{natbib}

\usepackage{fullpage}
\usepackage{color}
\usepackage{url}

\renewcommand{\familydefault}{\sfdefault}

\setlength{\parskip}{4mm}
\setlength{\parindent}{0cm}

\begin{document}

\title{POL30490 - POL30500 \\Advanced Seminar in Politics}

\author{Dr. James P. Cross\\\\School of Politics and International Relations\\University College Dublin}

\date{2017--2018}

\maketitle

\section*{Introduction}

	This 10 - credit course (divided into 2 x 5 - credit modules) gives advanced undergraduate students the background, training and supervision needed to engage in a substantial piece of research. A number of members of staff of the School of Politics and International Relations will provide foundational background and methodology lectures in their various fields of expertise. Students will then produce a literature review and a research proposal outlining a topic that they are interested in doing research on. With supervision, students will submit a 6,000 word Thesis. Participation in this module involves a large amount of independent work, and the self-discipline and the initiative required to do substantial research.

\subsubsection*{Learning outcomes}

\begin{itemize}
  \item How to do advanced research in Politics and International Relations, involving sustained argument and a good understanding of the relevant literature and data.
  \item How to formulate a research proposal.
  \item The background information and methodological techniques required to do advanced research.
  \item The skills and discipline required to do advanced, independent research.
\end{itemize}

\section*{Course structure}

	This is a year-long course divided into two parts (POL304490 and POL30500). The first part prepares students for undertaking their own research by demonstrating how UCD SPIRe faculty work towards publishing academic work. The second part of the course involves students writing their own Thesis. In both parts of the module, the student interacts with a supervisor, who provides guidance on how to prepare and undertake an original research project.

\subsection*{Supervision and supervisor assignment}

	Early in the module, you will be allocated a supervisor from among the SPIRe staff and will work with them on developing and executing your research plan. This implies that you must start thinking about a research topic and narrowing down a research question immediately.
	
	Supervisor assignment will occur based on the area of research you indicate you are interested in working on. You are required to identify a broad area of research you are interested in writing about and identify 3 potential staff members that you would like to work with based on the correspondence between your research interests and those of the current SPIRe staff. Please review the UCD staff profiles and their research interests here: \url{http://www.ucd.ie/spire/research/} and list 3 potential supervisors you would be interested in working with. This list along with your preferred area of research should be emailed to \url{james.cross@ucd.ie} by \textbf{20/9/2017}). Supervisors with expertise in your area of interest will be assigned where possible, but given supply and demand dynamics a direct match cannot be guaranteed.
	

\subsection*{Part I: Seminar series and research proposal preparation (POL30490)}

	Part I of the course will consist of a series of seminars aimed at introducing students to state-of-the-art research from different sub-fields of politics and international relations. The first and last seminar will focus on research methods and writing an honours thesis more generally, while the other sessions will have guest lectures from different faculty members, each focusing on a different substantive topic. The sessions will encourage you to engage in discussion with the lecturers on how to do research in the various fields. The idea is to make you more familiar with how research is undertaken in the discipline and how there are many different approaches to doing so, in preparation of developing your own Honours Thesis. 
	
	Typically the lecturer will provide you with a recent research paper, by themselves or one they are particularly familiar with, to discuss with you things like how this research came about, what the main puzzle or claim is, how it has been approached, what the findings or conclusions are, what the main argument or evidence is, etc.. You will get a good grasp of what the state of the art is in a range of subdisciplines and what social science research looks like at both a methodological and a practical level.
	
	The assessment for this part of the course is based on the submission of 1) a detailed literature review relating to the area of research you are interested in writing in for your Honours Thesis, and 2) a detailed research proposal outlining the research question you are planning to answer and the manner in which you are going to answer it.
	

\section*{Seminar topics and lecturers}

\begin{table}[ht]
	\centering
	\begin{tabular}{|r|l|p{.6\textwidth}|}
	\hline
	Seminar & Lecturer(s) & Topic \\
	\hline
	13/9/17 & James P. Cross & {\bf Introduction}: Approaches to social science research and common methodological concerns \\
	\hline
	20/9/17 & No seminar & \\
	\hline
	27/9/17 & Alexander Dukalskis & {\bf Authoritarian regimes} \\
	& Johan Elkink & \\
	\hline
	4/10/17 & Andy Storey & {\bf Political economy} \\
	&  & \\
	\hline
	11/10/17 & Vincent Durac & {\bf IR \& Middle east politics} \\
	& Tobias Theiler & \\
	\hline
	18/10/17 & Magda Staniek & {\bf Comparative politics} \\
	& Cristina Bucur & \\
	\hline
	25/10/17 & Graham Finlay & {\bf Political theory} \\
	& Alexa Zellentin & \\
	\hline
	1/11/17 & Ben Tonra & {\bf EU foreign policy} \\
	& Nikola Tomic & \\
	\hline
	8/11/17 & Jennifer Todd &  {\bf Conflict and identity} \\
	& Michael Keating & \\
	\hline
	15/11/17 & David Farrell & {\bf Elections and party systems} \\
	& Caroline McEvoy & \\
	\hline
	22/11/17 & James P. Cross & {\bf Writing an honours thesis}\\
	\hline
	\end{tabular}
	%\caption{}
	%\label{}
\end{table}


\section*{Seminar readings}

	Completing assigned readings ahead of each seminar is an essential part of this module. Readings will be made available in advance through Blackboard. You should take notes and prepare questions about each of the assigned readings, taking a critical eye towards the text under review. Focus on questions about how the research we designed and undertaken, what methodologies were used, and what practical problems were overcome in order to complete the research. These questions will form the basis of the seminars.	
	
	
\section*{Assessment and grading}

\subsection*{The literature review}

	The first module assignment is a \textbf{2,000 word literature review}. This word count excludes the bibliography. A literature review is designed to provide readers with an overview of the state-of-the-art research in a chosen field. Your aim should be to discuss existing research relating in a given field of interest, summarise and synthesise the findings presented, and identify weaknesses or gaps in the literature that you might later go on to address in your thesis. A literature review is an essential element of a research proposal and thesis. 
	
	The process of preparing a literature review can be divided into a series of steps:\footnote{Source: \url{http://libguides.slu.edu/c.php?g=185752&p=1227832}}

\begin{itemize}
	\item Choose a topic. Look at recent literature for ideas and do a bit of preliminary searching of the existing literature.
	\item Clarify your review question and the scope of your review.
	\item Brainstorm search terms to use and think about your search strategy. Google Scholar is a very useful tool for compiling a list of relevant literature, so familiarise yourself with its use.
	\item Begin searching for articles. I strongly recommend you keep a search log to document which databases you searched and what search terms you used.
Capture and manage search results. You may want to export results to Endnote or another reference tracking software to easily create a bibliography later on.
	\item Screen results for inclusion based on criteria you define
	\item Evaluate the  the articles. A worksheet which includes the bibliographic information about the article and summarises elements of the article such as research design, interventions, findings, main variables etc. may give you a helpful overview
	\item Synthesise results and write up the review.
\end{itemize}

\subsection*{Resources and guides for writing a literature review}

\begin{itemize}
	\item Baglione, L. A. (2015). Writing a research paper in political science: A practical guide to inquiry, structure, and methods. Cq Press.
	\item Pennings, P., Keman, H., \& Kleinnijenhuis, J. (2006). Doing research in political science: An introduction to comparative methods and statistics. Sage.
	\item Galvan, J. L., \& Galvan, M. C. (2017). Writing literature reviews: A guide for students of the social and behavioral sciences. Routledge.
	\item Kara, H. (2012). Research and evaluation for busy practitioners: a time-saving guide. Policy Press.
	\item Useful websites:
	\begin{itemize}
		\item \url{http://writingcenter.unc.edu/tips-and-tools/literature-reviews/}
		\item \url{http://libguides.slu.edu/c.php?g=185752&p=1227832}
		\item  \url{http://www.indstate.edu/cas/polisci/graduate-program-mpa/mpa-handbook-other-details/program-handbook-appendix-i-writing}
	\end{itemize}
\end{itemize}


\subsection*{Research proposal}

	The second part of the assessment for Part I of this course is writing a \textbf{1,000 word research proposal}. This word count excludes the bibliography. A research proposal should outline:

\begin{enumerate}
	\item Your proposed topic
	\item Your central research question
	\item The research methods and sources you plan to use
	\item The main contribution or argument of the proposed research
	\item References to the literature that you are building on and gaps in that literature
	\item A description of how you plan to make a contribution to this literature
\end{enumerate}

	The proposal will serve as the basis for further work on your thesis itself.  You will be given feedback based on your proposal so that you can go on to carry out your research.
	
\begin{itemize}
	\item Lipson, C. (2005). How to write a BA thesis: a practical guide from your first ideas to your finished paper. University of Chicago Press.
	\item Useful websites:
	\begin{itemize}
		\item \url{https://depts.washington.edu/pswrite/Handouts/HowtoWriteResearchProposals.pdf}
		\item \url{http://www.polisci.northwestern.edu/undergraduate/honors/how-to-apply/proposal-tips.html}
	\end{itemize} 
\end{itemize}

\subsection*{Assignment submission (Part I)}
	
	Submission of all assignments is through Blackboard. Two hardcopies of the thesis should also be left with the School office. The relative weight of each task and the associated deadlines are as follows:

\bigskip

\begin{table}[ht]
	\centering
	\begin{tabular}{lll}
		\hline
		Task & Weight & Deadline \\
		\hline
		Literature review & 25\% & 17/11/2017\\
		Research proposal & 25\% & 6/12/2017 \\
		\hline
	\end{tabular}
	\caption{Assignments for Part I of the Advanced Seminar (POL30490)}
    %\label{}
\end{table}


\section*{Part II: Honours thesis (POL30500)}

	Students are responsible for the progress and substance of their own Honours Thesis project. This includes establishing with the supervisor a timetable for the submission of written drafts and/or meetings and then fulfilling the agreement; completion of the dissertation by the submission date; ensuring that any difficulties with the research project are drawn to the attention of the supervisor without delay; conducting their research in a professional manner and complying fully with the School and University’s policy on academic integrity and plagiarism.
	
	The thesis should be approximately 6,000 words and no more than 6,500. The bibliography is not included in the word count and an appendix of no more than 500 words can be added if necessary, but should not be considered a core part of the submission --i.e. examiners might or might not read the appendix.

	You will need to submit two hard copies of the thesis, one in your supervisor's mailbox and one in mine. The thesis should have at least 1.15 line spacing and reasonably wide margins to allow for comments. You will also need to submit a soft copy through SafeAssign on Blackboard.
	
\subsection{The supervisor's role}

\begin{itemize}
	\item The supervisor’s principal role is to offer advice on the construction of an appropriate structure and methodology for the research, to direct the student to appropriate resources, and to comment on the research as it progresses.
	\item Within a week after the student-supervisor allocation is decided, every supervisee should contact their assigned supervisor to schedule an initial meeting. This initial meeting will help ensure that students begin to think constructively about their projects, start identifying relevant literature and start working on the various assignments that are required.
	\item Supervisors do not normally offer feedback on multiple drafts of the same assignment. Students who want feedback on a complete draft assignment should be advised to submit it at least ten days before the due date or by whatever earlier date the supervisor approves. Supervisor availability cannot be presumed and must be negotiated individually.
	\item Supervisors may not give a provisional mark on draft assignments.
\end{itemize}

\subsection*{Honours Thesis Structure}

As a guide for structure, you should aim to have the following sections in your thesis:

\begin{enumerate}
    \item Introduction
	\begin{itemize}
		\item What is your research question and why is it important?
		\item What outcome(s) are you trying to explain? 
		\item Try to focus in on a question that can be answered within the word limit. Many students make the mistake of trying to answer very broad and unfocused questions. Think about how your question can be narrowed down in focus if you think it is too broad.
		\item What cases or data will you use to answer your research question? Why is your methodology suited for the case/data
		\item In the last paragraph of this section describe the structure of the rest of the thesis. Let the reader know how you are going to answer the research question posed.
	\end{itemize}
	
	\item Literature review
	\begin{itemize}
		\item What is the current state of the art in the literature relating to your research question?
		\item What gap in the literature exists that your research is going to fill?
		\item You need to be careful to only include relevant literature so that you do not spend too much time talking about other peoples research. We are interested in your original contribution, not what has been done before. You have to use this section to place your research and yourself in the current debate.
	\end{itemize}

	\item Theory
	\begin{itemize}
		\item Which factors or explanations do you think explain the outcomes you are interested in? Try to focus in on a limited number of explanations that you think are relevant.
		\item Explain the mechanism through which your explanation leads to the outcomes you are studying.
		\item Include definitions and discussion of the concepts you think are relevant here.
		\item Outline specific testable hypotheses if you want these to be part of your argument.
		\item Besides the major hypotheses/explanations you are focusing on, you should also include any other factors that you think explain the observed outcomes (intervening variables, alternative explanations etc.). If you think they are part of the explanation, your reasoning should be made explicit in our theory section.
		\item Do not forget to assess the weaknesses of your theory, explain when and why it would not work.
	\end{itemize}

	\item Methodology
	\begin{itemize}
		\item How are you going to test your theories/hypotheses/explanations?
		\item What indicators are you going to use to measure or capture the theoretical concepts and explanations you expect explain the outcomes you are trying to explain. 
		\item Why are these indicators appropriate?
		\item Where does the data you are using come from?
		\item If you are collecting new data, how will you do so? What is the advantage of collecting your own data compared to what is available?
	\end{itemize}

	\item Analysis and results
	\begin{itemize}
		\item Describe your findings
		\item Explain how your findings relate to your theory/hypotheses. Do they support your proposed explanation of the phenomenon under consideration or not?
		\item Explain how your results relate to the current literature/existing findings
		\item Include tables/figures here if relevant
		\item When producing a table, remember decimals have a meaning: sometimes you care about 1 EuroCent, but not when analysing the government budget.
		\item When producing figures, make sure to label each axis clearly so the reader knows what is being presented.
	\end{itemize}

	\item Discussion/conclusion
	\begin{itemize}
		\item Restate the research question and explain how you have answered it based on the findings presented
		\item Do not introduce new explanations or factors relating to your research question here. Talk about the results described in the previous section.
		\item Here you also draw out the implications of your findings for both the existing literature and for the research question/problem you have studied.
	\end{itemize}

	\item Bibliography
	\begin{itemize}
		\item All works that are cited in your thesis must be included in the bibliography.
		\item Use a consistent style (Harvard recommended).
	\end{itemize}
\end{enumerate}


\begin{itemize}
	\item Remember:
	\begin{itemize}
		\item Be concise, be precise
		\item Ask only questions to which there may be answers
		\item Provide logical underpinnings to all hypotheses
		\item Illustrate your puzzle and proposed resolution with a few choice examples
		\item Indicate the broader relevance of your research
		\item Use a style guide: consistency is your friend and you want your thesis to be legible. A secondhand version of the Economist style guide is \textbf{very} cheap and \textbf{very} helpful: \url{http://amzn.eu/dsZDfsu}
	\end{itemize}
\end{itemize}


\subsection*{Useful literature}

\begin{itemize}
	\item Lipson, C. (2005). How to write a BA thesis: a practical guide from your first ideas to your finished paper. University of Chicago Press.
	\item Useful websites:
	\begin{itemize}
		\item \url{http://www.charleslipson.com/How-to-write-a-thesis.htm}
		\item \url{https://polisci.as.uky.edu/writing-dissertation}
	\end{itemize}
\end{itemize}


\subsection*{Thesis Submission}

	Thesis submission is also through Blackboard, with two hardcopies of the thesis also to be left with the School office.

\begin{table}[ht]
	\centering
	\begin{tabular}{lll}
		\hline
		Task & Weight & Deadline \\
		\hline
		Honours thesis & 50\% & 4/5/2017 \\
		\hline
	\end{tabular}
	\caption{Assignments for Part II of the Advanced Seminar (Spring term)}
\end{table}


\section*{Plagiarism}

	Although this should be obvious, plagiarism---copying someone else's text without acknowledgement or beyond ``fair use'' quantities---is not allowed. Please carefully check the UCD policies concerning plagiarism\footnote{{\tt http://www.ucd.ie/regist/documents/plagiarism\_policy\_and\_procedures.pdf.}} and its more extensive description of what is plagiarism and what is not\footnote{{\tt http://www.ucd.ie/library/students/information\_skills/plagiari.html}}. Arguing that you ``didn't know'' will not be a valid excuse when we find evidence of plagiarism---if it not really clear what is expected of you in this regard, ask.

\section*{Contact details}

	I do not have fixed office hours, so if you want to make sure I am present, you can make an appointment by email. The easiest way to reach me is by email ({\tt james.cross@ucd.ie}).

	To stay up to date with developments in the UCD School of Politics and International Relations, also keep an eye on the following social media:\\
Web: {\tt http://www.ucd.ie/politics/}\\
Blog: {\tt http://politicalscience.ie/}\\
Twitter: {\tt http://twitter.com/ucdpolitics}\\
Facebook: {\tt http://www.facebook.com/ucdspire}

\bibliographystyle{apsr}
\bibliography{ap}

\end{document}
