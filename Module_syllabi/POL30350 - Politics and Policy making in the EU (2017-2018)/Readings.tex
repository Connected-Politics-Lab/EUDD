%!TEX root = POL30350_syllabus.tex

\section*{MODULE READINGS}

\subsection*{Textbook}

	The following text shall be used extensively throughout the course, so it is recommended that it is purchased:

\begin{itemize}
	\item Lelieveldt, H., \& Princen, S. (2015). The Politics of the European Union, 2nd Edition. Cambridge University Press. Available through \url{http://www.amazon.co.uk} in paperback or kindle versions.
\end{itemize}

\subsection*{Methods in EU research - A guide to reading and interpreting cutting edge scholarship}

	This module is research driven, so a familiarity with how political research is undertaken and how to interpret the results presented in the literature is very useful. The following texts provide a good introduction to the different approaches you are likely to come across in the assigned readings.

\noindent Readings

\begin{itemize}
	\item Van Evera, S. (1997). Guide to methods for students of political science. Cornell University Press. Chapter 1.
	\item Bellmare, M. (2012). A Primer on Causality.
	\item Bellmare, M. (2012). A Primer on Linear Regression.
	\item Thomson, R., Stokman, F. N., Achen, C. H., \& K{\"o}nig, T. (Eds.). (2006). The European union decides. Cambridge University Press. Chapter 1 \& 2.
	\item Ragin, C. C. (2014). The comparative method: Moving beyond qualitative and quantitative strategies. University of California Press.
	\item Gschwend, T., \& Schimmelfennig, F. (Eds.). (2007). Research design in political science: how to practice what they preach. Palgrave Macmillan.
\end{itemize}


\section*{WEEKLY READING LIST}

\subsection*{PART I: THEORETICAL FRAMEWORK(S) AND METHODS IN EU RESEARCH}

\subsubsection*{Week 1: Politics and policy making in the EU - An introduction to the conceptual tools to be used in this course}

Study questions

\begin{itemize}
	\item What are the respective variants of the institutionalist approach; who are the main actors according to each approach and what role do they play in the EU policy process? 
	%\item 
\end{itemize}

\noindent Required Readings

\begin{itemize}
	\item Lelieveldt, H., \& Princen, S. (2015). The Politics of the European Union, 2nd Edition. Cambridge University Press. Chapter 1,2.
	\item Aspinwall, M. D., \& Schneider, G. (2000). Same menu, separate tables: The institutionalist turn in political science and the study of European integration. European Journal of Political Research, 38(1), 1-36.
	%\item Shepsle, K. A. (2006). Rational choice institutionalism. The Oxford Handbook of Political Institutions, 23-38.
\end{itemize}

\noindent Further reading

\begin{itemize}
	\item Pollack, M. A. (2007). Rational choice and EU politics. Handbook of European Union Politics, 31-55.
	\item Hinich M. J. \& Munger, M. C. (1997). Analytical politics.
	\item Shepsle, K. A., \& Weingast, B. R. (1981). Structure-induced equilibrium and legislative choice. Public choice, 37(3), 503-519.
	\item Shepsle, K. A., \& Weingast, B. R. (1987). The institutional foundations of committee power. The American Political Science Review, 85-104.
	\item Arrow, K. J. (1963). Social Choice and Individual Values (No. 12). Yale University Press.
	\item Tsebelis, G. (2002). Veto players: How political institutions work. Princeton University Press.
	\item Checkel, J. T. (1999). Social construction and integration. Journal of European Public Policy, 6(4), 545-560.
	\item Christiansen, T., J{\o}rgensen, K. E., \& Wiener, A. (Eds.). (2001). The social construction of Europe. Sage.
	\item Jupille, J., Caporaso, J. A., \& Checkel, J. T. (2003). Integrating Institutions Rationalism, Constructivism, and the Study of the European Union. Comparative Political Studies, 36(1-2), 7-40.
	\item Benoit, K., \& Laver, M. (2012). The dimensionality of political space: Epistemological and methodological considerations. European Union Politics, 13(2), 194-218.
\end{itemize}


%\subsubsection*{Week 2: Methods in EU research - A guide to reading and interpreting cutting edge scholarship}

%Study questions

%\begin{itemize}
%	\item What methods are most commonly used to investigate questions relating to EU politics?
%	\item What is the spatial model of politics and what are its' component parts? Why is this a useful approach for thinking about political conflict?
%	\item What are the strengths and weaknesses of quantitative and qualitative methods in EU research respectively? 
%	\item Why are interviews an important source of information on EU politics?
%\end{itemize}

%\noindent Required Readings

%\begin{itemize}
%	\item Van Evera, S. (1997). Guide to methods for students of political science. Cornell University Press. Chapter 1.
%	\item Bellmare, M. (2012). A Primer on Causality.
%	\item Bellmare, M. (2012). A Primer on Linear Regression.
%	\item Thomson, R., Stokman, F. N., Achen, C. H., \& K{\"o}nig, T. (Eds.). (2006). The European union decides. Cambridge University Press. Chapter 1 \& 2.
	%\item Visser et al. Surveys
%\end{itemize}

%\noindent Further reading

%\begin{itemize}
%	\item Ragin, C. C. (2014). The comparative method: Moving beyond qualitative and quantitative strategies. University of California Press.
%	\item Gschwend, T., \& Schimmelfennig, F. (Eds.). (2007). Research design in political science: how to practice what they preach. Palgrave Macmillan.
%\end{itemize}

\subsection*{PART II: NEGOTIATING EU PRIMARY AND SECONDARY LAW}

\subsubsection*{Week 2: Negotiating the treaties}

Study questions

\begin{itemize}
	\item What are the sources of states’ bargaining power during treaty negotiations? How do domestic constraints condition states’ bargaining strength at IGCs?  
    \item There is conflicting evidence regarding which players and what factors are most likely to affect the outcome of a treaty. What is the role of supranational actors in explaining IGC bargains?
    \item How has the Treaty of Lisbon shifted the EU’s institutional balance of power between the major institutions of the EU?
\end{itemize}

\noindent Required Readings

\begin{itemize}
	\item Lelieveldt, H., \& Princen, S. (2015). The Politics of the European Union, 2nd Edition. Cambridge University Press. Chapter 1,10
	\item Moravcsik, A. \& Nicolaidis, K. (1999). Explaining the Treaty of
Amsterdam: Interests, Influence, Institutions. Journal of Common Market
Studies. 37(1): 59-85.
	\item Slapin, J. B. (2008). Bargaining power at Europe's intergovernmental conferences: testing institutional and intergovernmental theories. International Organization, 62(01), 131-162.
	\item Finke, D. (2009). Challenges to intergovernmentalism: an empirical analysis of EU treaty negotiations since Maastricht. West European Politics, 32(3), 466-495.
\end{itemize}

\noindent Further reading

\begin{itemize}
	\item Hug, S., \& K{\"o}nig, T. (2002). In view of ratification: Governmental preferences and domestic constraints at the Amsterdam Intergovernmental Conference. International Organization, 56(02), 447-476.
	\item Finke, D., K{\"o}nig, T., Proksch, S. O., \& Tsebelis, G. (2012). Reforming the European Union: realizing the impossible. Princeton University Press.
	\item Slapin, J. B. (2006). Who is powerful? Examining preferences and testing sources of bargaining strength at European intergovernmental conferences. European Union Politics, 7(1), 51-76.
	\item Slapin, J. B. (2009). Exit, voice, and cooperation: Bargaining power in international organizations and federal systems. Journal of Theoretical Politics, 21(2), 187-211.
	\item Finke, D. (2010). European integration and its limits: intergovernmental conflicts and their domestic origins. ECPR Press.
	\item Franchino, F. (2013). Challenges to liberal intergovernmentalism. European Union Politics, 14(2), 324-337.
	\item Finke, D. (2009). Governmental Positions on European Treaty Reforms: Towards a Dynamic Approach.
	\item Reinhard, J., Biesenbender, J., \& Holzinger, K. (2014). Do arguments matter? Argumentation and negotiation success at the 1997 Amsterdam Intergovernmental Conference. European Political Science Review, 6(02), 283-307.
	\item Finke, D. (2007). Constitutional Politics in the European Union: Stability and Change of Intergovernmental Conflict Patterns.
	\item Finke, D. (2013). Reforming International Institutions: The Domestic Origins and Conditional Logic of Governmental Reform Preferences. International Studies Quarterly, 57(2), 288-302.
	\item Reinhard, J. (2012). ‘Because we are all Europeans!’When do EU Member States use normative arguments?. Journal of European Public Policy, 19(9), 1336-1356.
	\item K{\"o}nig, T., \& Slapin, J. (2004). Bringing Parliaments Back in The Sources of Power in the European Treaty Negotiations. Journal of Theoretical Politics, 16(3), 357-394.
	\item K{\"o}nig, T., \& Slapin, J. B. (2006). From unanimity to consensus: an analysis of the negotiations at the EU's constitutional convention. World Politics, 58(03), 413-445.
	\item Finke, D., \& K{\"o}nig, T. (2009). Why risk popular ratification failure? A comparative analysis of the choice of the ratification instrument in the 25 Member States of the EU. Constitutional Political Economy, 20(3-4), 341-365.
	\item K{\"o}nig, T., Daimer, S., \& Finke, D. (2008). The Treaty Reform of the EU: Constitutional Agenda-Setting, Intergovernmental Bargains and the Presidency's Crisis Management of Ratification Failure. JCMS: Journal of Common Market Studies, 46(2), 337-363.
	\item Lenz, H., Dorussen, H., \& Ward, H. (2007). Public commitment strategies in intergovernmental negotiations on the EU Constitutional Treaty. The Review of International Organizations, 2(2), 131-152.
	
\end{itemize}

\subsubsection*{Week 3: The Commission - Negotiating a legislative proposal?}

Study questions

\begin{itemize}
	\item Some recent studies find congruence between the Commission and member states’ policy positions suggesting that the Commission might not be “completely independent in the performance of their duties” as specified in the treaty and not quite as supranational as it has been often portrayed. To what can we attribute this policy congruence? i.e., what are some of the key factors that constrain the Commission’s independence and condition its policy positions, as identified in the readings? 
	\item How convincing do you find the argument that the Commission has become less of a supranational actor and more of an agent of national interests? Have the Commission's policies become increasingly less ``European"? What are the implications for the legislative decision-making in the EU if this were the case?
	\item What role do interest groups play in the process of putting together a legislative proposal? Is including interest groups in the process a good thing?
    \item Which conceptual approach to analysing the Commission’s legislative role is more appropriate: the Commission as a unitary actor approach or the Commission as a collective of individual actors approach?
\end{itemize}

\noindent Required Readings

\begin{itemize}
	\item Lelieveldt, H., \& Princen, S. (2015). The Politics of the European Union, 2nd Edition. Cambridge University Press. Chapter 6,8
	\item Egeberg, M. (2014). The European Commission: From agent to political institution. Public Administration, 92(1), 240-246.
	\item Bor{\"a}ng, F., \& Naurin, D. (2015). ‘Try to see it my way!’Frame congruence between lobbyists and European Commission officials. Journal of European Public Policy, 22(4), 499-515.
	\item D{\"u}r, A., Bernhagen, P., \& Marshall, D. (2015). Interest Group Success in the European Union When (and Why) Does Business Lose?. Comparative Political Studies, 48(8), 951-983
	\item Beyers, J., Bonafont, L. C., D{\"u}r, A., Eising, R., Fink-Hafner, D., Lowery, D., ... \& Naurin, D. (2014). The INTEREURO project: Logic and structure. Interest Groups \& Advocacy, 3(2), 126-140.
	\item Explore the following website: \url{http://www.intereuro.eu/public/}.
\end{itemize}

\noindent Further reading

\begin{itemize}
	\item Kl{\"u}ver, H., Mahoney, C. and Opper, M. (2015). Framing in context: how interest groups employ framing to lobby the European Commission. Journal of European Public Policy, 22(4), 481-498.
	\item Egeberg, M., Gornitzka, Å., \& Trondal, J. (2014). A not so technocratic executive? Everyday interaction between the European Parliament and the Commission. West European Politics, 37(1), 1-18.
	\item Christensen, J. (2015). Recruitment and Expertise in the European Commission. West European Politics, 38(3), 649-678.
	\item Wonka, A. (2007). Technocratic and Independent? The Appointment of European Commissioners and its Policy Implications. Journal of European Public Policy 14(2): 169-89.
	\item Egeberg, M., Trondal, J., \& Vestlund, N. M. (2015). The quest for order: unravelling the relationship between the European Commission and European Union agencies. Journal of European Public Policy, 22(5), 609-629.
	\item Wonka, A. (2008). ‘Decision-Making Dynamics in the European Commission: Partisan, National or Sectoral?’, Journal of European Public Policy 15(8), pp.1145–63.
	\item Hartlapp, M., Metz, J., \& Rauh, C. (2014). Which policy for Europe?: power and conflict inside the European Commission. Oxford University Press.
	\item Egeberg, M., Trondal, J., \& Vestlund, N. (2014). Situating EU agencies in the political-administrative space. ARENA Working Papers 6.
    \item Kassim, H. (2008). ‘“Mission Impossible”, but Mission Accomplished: the Kinnock Reforms and the European Commission’, Journal of European Public Policy 15(5), 648–68
    \item Smith, A., (2014). How the European Commission’s Policies Are Made: Problematization, Instrumentation and Legitimation, Journal of European Integration, 36(1), 55-72.
    \item Schmidt, S.K. (2000). ‘Only an Agenda Setter? The European Commission’s Power Over the Council of Ministers’, European Union Politics 1(1), pp.37–61
    \item Peterson, J. (2012). “The College of Commissioners.” In: John Peterson \& Michael Shackleton (eds.). The Institutions of the European Union. 3rd edition. Oxford: Oxford University Press: 96-123.
    \item Quittkat, C. (2011). The European Commission's Online Consultations: A Success Story? Journal of Common Market Studies 49(3): 653-674.
    \item Pollack, M.A., (1997). Delegation, agency, and agenda setting in the European Community. International Organization, 51(1), pp.99–134.
	\item Beyers, J., Bonafont, L. C., D{\"u}r, A., Eising, R., Fink-Hafner, D., Lowery, D., ... \& Naurin, D. (2014). The INTEREURO project: Logic and structure. Interest Groups \& Advocacy, 3(2), 126-140.
	\item Bernhagen, P., D{\"u}r, A., \& Marshall, D. (2014). Measuring lobbying success spatially. Interest Groups \& Advocacy, 3(2), 202-218.
	\item Eising, R., Rasch, D., \& Rozbicka, P. (2015). Institutions, policies, and arguments: context and strategy in EU policy framing. Journal of European Public Policy, 22(4), 516-533.
	\item Bunea, A., \& Ibenskas, R. (2015). Quantitative text analysis and the study of EU lobbying and interest groups. European Union Politics, 16(3).
	\item Kl{\"u}ver, H. (2015). The promises of quantitative text analysis in interest group research: a reply to Bunea and Ibenskas. European Union Politics, 16(3).
	\item Bunea, A. (2014). Sharing ties and preferences: Stakeholders’ position alignments in the European Commission’s open consultations. European Union Politics, 16(2).
\end{itemize}

\subsubsection*{Week 4: The Council - Negotiating a consensus between member states?}

Study questions

\begin{itemize}
	\item How is the Council of Ministers organised? Who holds power in the Council?
	\item Does the hierarchical structure of the Council help or hinder its ability to make decisions?
	\item What factors drive the `consensus style' decision-making we tend to observe in the Council?
\end{itemize}

\noindent Required Readings

\begin{itemize}
	\item Lelieveldt, H., \& Princen, S. (2015). The Politics of the European Union, 2nd Edition. Cambridge University Press. Chapter 3
	\item Lewis, J. (2003). Institutional Environments And Everyday EU Decision Making Rationalist or Constructivist?. Comparative Political Studies, 36(1-2), 97-124.
	\item Aksoy, D. (2012). Institutional arrangements and logrolling: Evidence from the European Union. American Journal of Political Science, 56(3), 538-552.
	\item Naurin, D. (2015). Generosity in intergovernmental negotiations: The impact of state power, pooling and socialisation in the Council of the European Union. European Journal of Political Research.
\end{itemize}	

\noindent Further reading

\begin{itemize}
	\item Arregui, J., \& Thomson, R. (2009). States' bargaining success in the European Union. Journal of European Public Policy, 16(5), 655-676.
	\item Arregui, J., \& Thomson, R. (2014). Domestic adjustment costs, interdependence and dissent in the Council of the European Union. European Journal of Political Research, 53(4), 692-708.
	\item Bailer, S., Mattila, M., \& Schneider, G. (2015). Money makes the EU go round: The objective foundations of conflict in the council of ministers. JCMS: Journal of Common Market Studies, 53(3), 437-456.
	\item Smeets, S. (2015). Unanimity and exposure in the EU Council of Ministers–or how the Dutch won and lost the ICTY debate. European Journal of Political Research, 54(2), 288-304.
	\item Finke, D., \& Herbel, A. (2015). Beyond rules and resources: Parliamentary scrutiny of EU policy proposals. European Union Politics, 1465116515584202.
	\item Scherpereel, J. A., \& Perez, L. K. (2015). Turnover in the Council of the European Union: what it is and why it matters. JCMS: Journal of Common Market Studies, 53(3), 658-673.
	\item Veen, T. (2011). The Political Economy of Collective Decision-making: Conflicts and Coalitions in the Council of the European Union. Springer.
	\item Gr{\o}n, C. H., \& Salomonsen, H. H. (2015). Who's at the table? An analysis of ministers’ participation in EU Council of Ministers meetings. Journal of European Public Policy, 22(8), 1071-1088.
	\item Schalk, J., Torenvlied, R., Weesie, J., \& Stokman, F. (2007). The power of the Presidency in EU Council decision-making. European Union Politics, 8(2), 229-250.
	\item Cross, J. P. (2012). Interventions and negotiation in the Council of Ministers of the European Union. European Union Politics, 13(1), 47-69.
	\item Cross, J. P. (2013). Everyone’s a winner (almost): Bargaining success in the Council of Ministers of the European Union. European Union Politics, 14(1), 70-94.
	\item Golub, J. (2012). How the European Union does not work: national bargaining success in the Council of Ministers. Journal of European Public Policy, 19(9), 1294-1315.
	\item Tallberg, J. (2008). Bargaining Power in the European Council*. JCMS: Journal of Common Market Studies, 46(3), 685-708.
	\item Warntjen, A. (2008). The Council Presidency Power Broker or Burden? An Empirical Analysis. European Union Politics, 9(3), 315-338.
	\item Smeets, S., \& Vennix, J. (2014). ‘How to make the most of your time in the Chair’: EU presidencies and the management of Council debates. Journal of European Public Policy, 21(10), 1435-1451.
	\item Jensen, M. D., \& Nedergaard, P. (2014). Uno, duo, trio? Varieties of Trio Presidencies in the Council of Ministers. JCMS: Journal of Common Market Studies, 52(5), 1035-1052.
	\item Kreppel, A. (2014). Ideology in the EU’s Second Chamber: A New Understanding of the Character and Impact of the Council on EU Policy Making. In APSA 2014 Annual Meeting Paper. Available at SSRN: \url{https://ssrn.com/abstract=2451566}	
\end{itemize}


\subsubsection*{Week 5: The European Parliament - Negotiating, committees and voting}

Study questions

\begin{itemize}
	\item Does the EP function like a `normal Parliament'?
	\item Should supranational partisan motivations override national interests for MEPs in the legislative process of the EU?
	\item How do the institutional structures of the EP impact upon the behaviour of MEPs?
	\item What roles to do committees play in the EP? How are committee roles assigned, and does this give European party groups power to affect the behaviour of MEPs?
	\item Does the separation of powers between the legislative and the executive authority in the EU help or hinder the policy process and the overall functioning of the Union?
\end{itemize}

\noindent Required Readings

\begin{itemize}
	\item Lelieveldt, H., \& Princen, S. (2015). The Politics of the European Union, 2nd Edition. Cambridge University Press. Chapter 7
	\item Hix, S., Noury, A., \& Roland, G. (2005). Power to the parties: cohesion and competition in the European Parliament, 1979–2001. British Journal of Political Science, 35(02), 209-234.
	\item Yoshinaka, A., McElroy, G., \& Bowler, S. (2010). The appointment of rapporteurs in the European Parliament. Legislative Studies Quarterly, 35(4), 457-486.
	\item Yordanova, N. (2009). The Rationale behind Committee Assignment in the European Parliament Distributive, Informational and Partisan Perspectives. European Union Politics, 10(2), 253-280.
	\item Costello, R. \& Thomson, R. (2010). The policy impact of leadership in committees: Rapporteurs’ influence on the European Parliament’s opinions. European Union Politics, 11(2), 219-240.
	\item Slapin, J. B., \& Proksch, S. O. (2010). Look who’s talking: Parliamentary debate in the European Union. European Union Politics, 11(3), 333-357.
	\item What role do national parliaments play in EU politics?
\end{itemize}

\noindent Further reading

\begin{itemize}
	\item Greene, D., \& Cross, J. P. (2017). Exploring the Political Agenda of the European Parliament Using a Dynamic Topic Modeling Approach. Political Analysis, 25(1), 77-94.
	\item Rauh, C., \& De Wilde, P. (2017). The opposition deficit in EU accountability: Evidence from over 20 years of plenary debate in four member states. European Journal of Political Research. (Ahead of print).
	\item Crum, B. (2017). Parliamentary accountability in multilevel governance: what role for parliaments in post-crisis EU economic governance? Journal of European Public Policy, (Ahead of print).
	\item Costello, R., \& Thomson, R. (2014). Bicameralism, nationality and party cohesion in the European Parliament. Party Politics, .
	\item Corbett, R., Jacobs, F., \& Shackleton, M. (2000). The European Parliament (Vol. 5). London: John Harper.
	\item Hix, S., Noury, A. G., \& Roland, G. (2007). Democratic politics in the European Parliament. Cambridge University Press.
	\item McElroy, G., \& Benoit, K. (2007). Party groups and policy positions in the European Parliament. Party Politics, 13(1), 5-28.
	\item Hix, S. (2002). Parliamentary behavior with two principals: preferences, parties, and voting in the European Parliament. American Journal of Political Science, 688-698.
	\item Hix, S., Noury, A., \& Roland, G. (2006). Dimensions of politics in the European Parliament. American Journal of Political Science, 50(2), 494-520.
	\item Hix, S., \& Marsh, M. (2007). Punishment or protest? Understanding European parliament elections. Journal of Politics, 69(2), 495-510.
	\item Hix, S., \& Noury, A. (2009). After enlargement: voting patterns in the sixth European Parliament. Legislative Studies Quarterly, 34(2), 159-174.
	\item Proksch, S. O., \& Slapin, J. B. (2010). Position taking in European Parliament speeches. British Journal of Political Science, 40(03), 587-611.
	\item Lindberg, B., Rasmussen, A., \& Warntjen, A. (2008). Party politics as usual? The role of political parties in EU legislative decision-making. Journal of European Public Policy, 15(8), 1107-1126.
	\item Bressanelli, E. (2012). National parties and group membership in the European Parliament: ideology or pragmatism?. Journal of European Public Policy, 19(5), 737-754.
	\item McElroy, G., \& Benoit, K. (2009). Party group switching in the European Parliament. Palgrave Macmillan.
	\item Finke, D. (2012). Proposal stage coalition-building in the European Parliament. European Union Politics, 13(4), 487-512.
	\item McElroy, G. (2006). Committee representation in the European Parliament. European Union Politics, 7(1), 5-29.
	\item Armingeon, K., \& Ceka, B. (2013). The loss of trust in the European Union during the great recession since 2007: The role of heuristics from the national political system. European Union Politics, 15(1).
	\item Lefkofridi, Z., \& Katsanidou, A. (2014). Multilevel representation in the European Parliament. European Union Politics, 15(1), 108-131.
	\item Trumm, S. (2015). Voting Procedures and Parliamentary Representation in the European Parliament. JCMS: Journal of Common Market Studies.
	\item Schmitt, H., Hobolt, S. B., \& Popa, S. A. (2015). Does personalization increase turnout? Spitzenkandidaten in the 2014 European Parliament elections. European Union Politics.
	\item Winzen, T. (2011). Technical or political? An exploration of the work of officials in the committees of the European Parliament. The Journal of Legislative Studies, 17(1), 27-44.
\end{itemize}


\subsubsection*{Week 6: Inter-institutional politics - Negotiating inter-institutional agreements}

Study questions

\begin{itemize}
	\item The EU has been described as a `hyper-consensus system of government' (Hix, 2008) because of the need to accommodate a large number of veto players in the policy - making process. Describe and give examples of how actors within particular institutions are able to use formal and informal powers of these institutions to align legislation with their preferences.
	\item In a highly consensual policy process like that found in the EU, where policies are often moving targets, what are the most decisive factors that contribute to the successful adoption of policies?
	\item What sorts of coalitions are required for the adoption of successive policies in the areas of structural funds and employment. At what levels of governance do these coalitions exist?
	\item How might informal political arrangements reduce the difficulty associated with reaching agreements under more formal legislative procedures?
\end{itemize}

\noindent Required Readings

\begin{itemize}
	\item Lelieveldt, H., \& Princen, S. (2015). The Politics of the European Union, 2nd Edition. Cambridge University Press. Chapter 4.
	\item Reh, C., Héritier, A., Bressanelli, E., \& Koop, C. (2013). The informal politics of legislation explaining secluded decision making in the European Union. Comparative Political Studies, 46(9), 1112-1142.
	\item Cross, J. P., \& Hermansson, H. (2017). Legislative amendments and informal politics in the European Union: A text reuse approach. European Union Politics, (Ahead of print).
\end{itemize}

\noindent Further reading

\begin{itemize}
	\item Rasmussen, A., \& Reh, C. (2013). The consequences of concluding codecision early: trilogues and intra-institutional bargaining success. Journal of European Public Policy, 20(7), 1006-1024.
	\item K{\"o}nig, T., Lindberg, B., Lechner, S., \& Pohlmeier, W. (2007). Bicameral conflict resolution in the European Union: an empirical analysis of conciliation committee bargains. British Journal of Political Science, 37(02), 281-312.
	\item Tsebelis, G. (1994). The power of the European Parliament as a conditional agenda setter. American Political Science Review, 88(01), 128-142.
	\item Bailer, S., (2014). An Agent Dependent on the EU Member States? The Determinants of the European Commission’s Legislative Success in the European Union, Journal of European Integration, 36(1), 37-53.
	\item Rasmussen, A. (2011). Early conclusion in bicameral bargaining: Evidence from the co-decision legislative procedure of the European Union. European Union Politics, 12(1), 41-64.
	\item Rasmussen, A. (2008). The EU Conciliation Committee One or Several Principals?. European Union Politics, 9(1), 87-113.
	\item Rasmussen, A. (2012). Twenty Years of Co-decision Since Maastricht: Inter-and Intrainstitutional Implications. Journal of European Integration, 34(7), 735-751.
	\item Hix, S., \& H{\o}yland, B. (1999). The political system of the European Union (p. 357). London: Macmillan. Chapter 3
	\item Hansen, V. W. (2014). Incomplete information and bargaining in the EU: An explanation of first-reading non-agreements. European Union Politics, 15(4), 472-495.
	\item M{\"u}hlb{\"o}ck, M., \& Rittberger, B. (2015). The Council, the European Parliament, and the paradox of inter-institutional cooperation. European Integration online Papers (EIoP), 19.
	\item Costello, R., \& Thomson, R. (2011). The nexus of bicameralism: Rapporteurs’ impact on decision outcomes in the European Union. European Union Politics, 12(3), 337-357.
	\item Costello, R. (2011). Does bicameralism promote stability? Inter-institutional relations and coalition formation in the European Parliament. West European Politics, 34(1), 122-144.
    \item Costello, R.\& Thomson, R. (2013). The distribution of power among EU institutions: who wins under codecision and why? Journal of European Public Policy 20(7): 1025-1039.
	\item B{\o}lstad, J., \& Cross, J. P. (2016). Not all treaties are created equal: The effects of treaty changes on legislative efficiency in the EU. JCMS: Journal of Common Market Studies, 54(4), 793-808.
    \item Burns, C., Rasmussen, A, \& Reh, C. (2013). Legislative codecision and its impact on the political system of the European Union. Journal of European Public Policy 20(7): 941-952.
    \item K{\"o}nig, T. (2008). Analysing the Process of EU Legislative Decision-Making. To Make a Long Story Short. European Union Politics 9(1): 145-165.
    \item Kardasheva, R. (2009). “The Power to Delay: The European Parliament's Influence in the Consultation Procedure.” Journal of Common Market Studies 47(2): 385-409.
    \item Bocquillon, P. \& Dobbels, M. (2013). An elephant on the 13th floor of the Berlaymont? European Council and Commission relations in legislative agenda-setting. Journal of European Public Policy 21(1): 20-38.
	\item Brandsma, G. J. (2015). Co-decision after Lisbon: The politics of informal trilogues in European Union lawmaking. European Union Politics, 16(2).
\end{itemize}


\subsubsection*{Week 7: Writing week}

\begin{itemize}
	\item Students should use this week to complete their blog post assignment. The assignment is due before 3pm on the \textbf{27th October 2017}.
\end{itemize}

\subsubsection*{Week 8: The ECJ - Powers, implementation and compliance}

Study questions

\begin{itemize}
	\item What role does the ECJ play in EU politics and through what channels can it have an influence over policy outcomes?
	\item How independent is the ECJ? What institutional rules and judicial rulings have given rise to the level of ECJ independence we observe?
	\item Why might member states not comply with EU law and what can the EU do about cases of non-compliance?
\end{itemize}

\noindent Required Readings

\begin{itemize}
	\item Lelieveldt, H., \& Princen, S. (2015). The Politics of the European Union, 2nd Edition. Cambridge University Press. Chapter 11
	\item K{\"o}nig, T., \& M{\"a}der, L. (2014). The strategic nature of compliance: an empirical evaluation of law implementation in the central monitoring system of the European Union. American Journal of Political Science, 58(1), 246-263.
	\item Tallberg, J. (2002). Paths to compliance: Enforcement, management, and the European Union. International Organization, 56(03), 609-643.
	\item Zhelyazkova, A. (2013). Complying with EU directives' requirements: the link between EU decision-making and the correct transposition of EU provisions. Journal of European Public Policy, 20(5), 702-721.
\end{itemize}

\noindent Further reading

\begin{itemize}
	\item Thomann, E., \& Sager, F. (2017). Moving beyond legal compliance: innovative approaches to EU multilevel implementation. Journal of European Public Policy, 24(9), 1252-1268.
	\item Thomann, E., \& Zhelyazkova, A. (2017). Moving beyond (non-) compliance: the customization of European Union policies in 27 countries. Journal of European Public Policy, 24(9), 1269-1288.
	\item Mastenbroek, E. (2017). Guardians of EU law? Analysing roles and behaviour of Dutch legislative drafters involved in EU compliance. Journal of European Public Policy, 24(9), 1289-1307.
	\item Gollata, J. A., \& Newig, J. (2017). Policy implementation through multi-level governance: analysing practical implementation of EU air quality directives in Germany. Journal of European Public Policy, 24(9), 1308-1327.
	\item Scholten, M. (2017). Mind the trend! Enforcement of EU law has been moving to ‘Brussels’. Journal of European Public Policy,  24(9), 1348-1366.
	\item Heidbreder, E. G. (2017). Strategies in multilevel policy implementation: moving beyond the limited focus on compliance. Journal of European Public Policy,  24(9), 1367-1384.
	\item Thomann, E., \& Sager, F. (2017). Toward a better understanding of implementation performance in the EU multilevel system. Journal of European Public Policy, 24(9), 1385-1407. 
	\item Tallberg, J. (2004). European governance and supranational institutions: making states comply. Routledge.
	\item Alter, K. (2010). The European Court's political power: selected essays. Oxford University Press.
	\item Hartlapp, M. (2007). On Enforcement, Management and Persuasion: Different Logics of Implementation Policy in the EU and the ILO*. JCMS: Journal of Common Market Studies, 45(3), 653-674.
	\item Falkner, G., Treib, O., \& Holzleithner, E. (2008). Compliance in the enlarged European Union: living rights or dead letters?. Ashgate Publishing, Ltd..
	\item Dimitrova, A. L., \& Toshkov, D. (2009). Post-accession compliance between administrative co-ordination and political bargaining. European Integration online Papers (EIoP), (2).
	\item Thomson, R. (2009). Same effects in different worlds: the transposition of EU directives. Journal of European Public Policy, 16(1), 1-18.
	\item Thomson, R. (2010). Opposition through the back door in the transposition of EU directives. European Union Politics, 11(4), 577-596.
	\item Toshkov, D. (2008). Embracing European law compliance with EU directives in Central and Eastern Europe. European Union Politics, 9(3), 379-402.
	\item Toshkov, D. (2010). Taking stock: a review of quantitative studies of transposition and implementation of EU law. Institute for European Integration Research.
	\item Angelova, M., Dannwolf, T., \& K{\"o}nig, T. (2012). How robust are compliance findings? A research synthesis. Journal of European Public Policy, 19(8), 1269-1291.
	\item Falkner, G. (2007). Time to discuss: Data to crunch or problems to solve? A rejoinder to Robert Thomson. West European Politics, 30(5), 1009-1021.
	\item Zhelyazkova, A., \& Torenvlied, R. (2011). The successful transposition of European provisions by member states: application to the Framework Equality Directive. Journal of European public policy, 18(5), 690-708.
	\item Zhelyazkova, A., \& Yordanova, N. (2015). Signalling ‘compliance’: The link between notified EU directive implementation and infringement cases. European Union Politics, 1465116515576394.
	\item Kaeding, M. (2006). Determinants of transposition delay in the European Union. Journal of Public Policy, 26(03), 229-253.
	\item K{\"o}nig, T., \& Luig, B. (2014). Ministerial gatekeeping and parliamentary involvement in the implementation process of EU directives. Public Choice, 160(3-4), 501-519.
	\item Finke, D., \& Dannwolf, T. (2015). Who let the dogs out? The effect of parliamentary scrutiny on compliance with EU law. Journal of European Public Policy, 22(8), 1127-1147.
	
\end{itemize}


\subsection*{PART III: NEGOTIATING THE EU BUDGET AND THE RESPONSE TO THE EUROCRISIS}

\subsubsection*{Week 9: Actors, institutions and power in EU budget negotiations}

Study questions

\begin{itemize}
	\item What actors and institutions are involved in negotiating the EU budget?
	\item How are the powers of these actors to shape the EU budget influenced by the institutional rules structuring the budget-making process?
	\item How has the budget-making process in the EU evolved over time? How do the economic contexts of the time affect the budget-making procedure?
\end{itemize}

\noindent Required Readings

\begin{itemize}
	\item Citi, M. (2015). European Union budget politics: Explaining stability and change in spending allocations. European Union Politics, 16(2), 260-280.
	\item Crombez, C. \& Høyland, B. (2015). The budgetary procedure in the European Union and the implications of the Treaty of Lisbon. European Union Politics 16(1): 67–89.
	\item Benedetto, G. (2017). Power, money and reversion points: the European Union's annual budgets since 2010. Journal of European Public Policy, 24(5), 633-652.
	\item Goetz, K. H., \& Patz, R. (2016). Pressured budgets and the European Commission: towards a more centralized EU budget administration?. Journal of European Public Policy, 23(7), 1038-1056.
\end{itemize}

\noindent Further reading

\begin{itemize}
	\item Citi, M. (2013). EU budgetary dynamics: incremental or punctuated equilibrium?. Journal of European Public Policy, 20(8), 1157-1173.
	\item Citi, M. (2014). Reforming the EU budget: A Time Series Analysis of Institutional and Partisan Effects. Available at SSRN 2432087.
	\item Aksoy, D. (2010). Who gets what, when, and how revisited: Voting and proposal powers in the allocation of the EU budget. European Union Politics 11(2): 171–194.
	\item Alt, J. E. \& Lassen, D. D. (2006). Transparency, political polarization, and political budget cycles in OECD countries. American Journal of Political Science 50(3): 530–550.
	\item Bailer, S., Mattila, M., \& Schneider, G. (2015). Money makes the EU go round: The objective foundations of conflict in the council of ministers. JCMS: Journal of Common Market Studies, 53(3), 437-456.
	\item Blavoukos, S. \& Pagoulatos, G. (2011). Accounting for coalition-building in the European Union: Budget negotiations and the south. European Journal of Political Research 50(4): 559–581.
	\item Bojar, A. (2014). Intra-governmental bargaining and political budget cycles in the European Union. European Union Politics 15(1): 132–151.
	\item Breunig, C. (2006). The more things change, the more things stay the same: A comparative analysis of budget punctuations. Journal of European Public Policy 13(7): 1069–1085.
	\item Dellmuth, L. M. \& Stoffel, M. F. (2012). Distributive politics and intergovernmental transfers: The local allocation of European Union structural funds. European Union Politics 13(3): 413–433.
	\item Goldbach, R. \& Fahrholz, C. (2011). The euro area’s common default risk: Evidence on the Commission’s impact on European fiscal affairs. European Union Politics 12(4): 507–528.
	\item Jones, B. D., Baumgartner, F. R., Breunig, C., et al. (2009). A general empirical law of public budgets: A comparative analysis. American Journal of Political Science 53(4): 855–873.
	\item K{\"o}nig, T. \& Troeger, V. E. (2005). Budgetary politics and veto players. Swiss Political Science Review 11(4): 47–75.
	\item Laffan, B. (2000). The big budgetary bargains: From negotiation to authority. Journal of European Public Policy 7(5): 725–743.
	\item Mink, M. \& de Haan, J. (2006). Are there political budget cycles in the euro area? European Union Politics 7(2): 191–211.
	\item Chang, M. (2013). Fiscal policy coordination and the future of the community method. Journal of European Integration, 35(3), 255-269.
	
\end{itemize}


\subsubsection*{Week 10: The emergence and evolution of new institutions in response to crisis}

Study questions

\begin{itemize}
	\item What issues in the institutional structure of the Euro came to the fore following the Euro crisis?
	\item What institutional developments have occurred to address some of these weaknesses?
	\item How have these new institutions and rules affected the balance of power between the existing actors in EU politics?
\end{itemize}

\noindent Required Readings

\begin{itemize}
	\item Schimmelfennig, F. (2015). Liberal intergovernmentalism and the euro area crisis. Journal of European Public Policy, 22(2), 177-195.
	\item Salines, M., Gl{\"o}ckler, G., \& Truchlewski, Z. (2012). Existential crisis, incremental response: the eurozone's dual institutional evolution 2007–2011. Journal of European Public Policy, 19(5), 665-681.
	\item Verdun, A. (2015). A historical institutionalist explanation of the EU's responses to the euro area financial crisis. Journal of European Public Policy, 22(2), 219-237.
	\item Bauer, M. W., \& Becker, S. (2014). The unexpected winner of the crisis: The European Commission’s strengthened role in economic governance. Journal of European Integration, 36(3), 213-229.
\end{itemize}

\noindent Further reading

\begin{itemize}
	\item Gl{\"o}ckler, G., Lindner, J., \& Salines, M. (2017). Explaining the sudden creation of a banking supervisor for the euro area. Journal of European Public Policy, 24(8), 1135-1153.
	\item Crum, B. (2017). Parliamentary accountability in multilevel governance: what role for parliaments in post-crisis EU economic governance? Journal of European Public Policy, (Ahead of print).
	\item Kastner, L. (2017). Business lobbying under salience–financial industry mobilization against the European financial transaction tax. Journal of European Public Policy, (Ahead of print).
	\item Scicluna, N. (2017). Integration through the disintegration of law? The ECB and EU constitutionalism in the crisis. Journal of European Public Policy, (Ahead of print).
	\item Nielsen, B., \& Smeets, S. (2017). The role of the EU institutions in establishing the banking union. Collaborative leadership in the EMU reform process. Journal of European Public Policy, 1-24.
	\item Donnelly, S. (2016) ‘Expert advice and political choice in constructing European banking union’, Journal of Banking Regulation 1–15.
	\item Epstein, R.A. and Rhodes, M. (2016) ‘The political dynamics behind Europe's new banking union’, West European Politics 39(3): 415–437.
	\item Sch{\"a}fer, D. (2016). A banking union of ideas? The impact of ordoliberalism and the vicious circle on the EU banking union. JCMS: Journal of Common Market Studies, 54(4), 961-980.
	\item Schimmelfennig, F. (2014). European integration in the euro crisis: The limits of postfunctionalism. Journal of European Integration, 36(3), 321-337.
	\item Fabbrini, S. (2013) ‘Intergovernmentalism and its limits: assessing the European Union's answer to the euro crisis’, Comparative Political Studies 46(9): 1003–1029. 
	\item Hennessy, A. (2014) ‘Redesigning financial supervision in the European Union (2009–2013)’, Journal of European Public Policy 21(2): 151–168.
	\item Savage, J. D., \& Verdun, A. (2016). Strengthening the European Commission's budgetary and economic surveillance capacity since Greece and the euro area crisis: a study of five Directorates-General. Journal of European Public Policy, 23(1), 101-118.
	\item De Rynck, S. (2016). Banking on a union: the politics of changing eurozone banking supervision. Journal of European Public Policy, 23(1), 119-135.
	Howarth, D. and Quaglia, L. (2013) ‘Banking union as holy grail: rebuilding the single market in financial services, stabilizing Europe's banks and “completing” economic and monetary union’, Journal of Common Market Studies 51(S1): 103–123. 
	\item Howarth, D., \& Quaglia, L. (2014). The steep road to European Banking Union: constructing the single resolution mechanism. JCMS: Journal of Common Market Studies, 52(S1), 125-140.
	\item Schwarzer, D. (2015). Building the euro area's debt crisis management capacity with the IMF. Review of International Political Economy, 22(3), 599-625.
	\item Mény, Y. (2014). Managing the EU crises: another way of integration by stealth?. West European Politics, 37(6), 1336-1353.
	\item Ackrill, R., \& Kay, A. (2006). Historical-institutionalist perspectives on the development of the EU budget system. Journal of European Public Policy, 13(1), 113-133.
	\item Blavoukos, S., \& Pagoulatos, G. (2011). Accounting for coalition-building in the European Union: Budget negotiations and the south. European Journal of Political Research, 50(4), 559-581.
	\item D{\"u}r, A., \& Mateo, G. (2010). Bargaining power and negotiation tactics: the negotiations on the EU's financial perspective, 2007-13. JCMS: Journal of Common Market Studies, 48(3), 557-578.
	\item Kauppi, H., \& Widgrén, M. (2007). Voting rules and budget allocation in the enlarged EU. European Journal of Political Economy, 23(3), 693-706.
	\item Saurugger, S. (2014). Europeanisation in times of Crisis. Political Studies Review, 12(2), 181-192.
	\item Ladi, S., \& Tsarouhas, D. (2014). The politics of austerity and public policy reform in the EU. Political Studies Review, 12(2), 171-180.
	\item Papanikolaou, N. I. (2015). The road towards the establishment of the European Banking Union. Munich Personal RePEc Archive Paper, (62463).
	\item Howarth, D., \& Quaglia, L. (2015). The political economy of the euro area's sovereign debt crisis: introduction to the special issue of the Review of International Political Economy. Review of International Political Economy, 22(3), 457-484.
	\item Gandrud, C., \& Hallerberg, M. (2015). Does banking union worsen the EU's democratic deficit? The need for greater supervisory data transparency. JCMS: Journal of Common Market Studies, 53(4), 769-785.
	\item Gren, J., Howarth, D., \& Quaglia, L. (2015). Supranational Banking Supervision in Europe: The Construction of a Credible Watchdog. JCMS: Journal of Common Market Studies, 53(S1), 181-199.
	\item Crum, B. (2013). Saving the Euro at the Cost of Democracy?. JCMS: Journal of Common Market Studies, 51(4), 614-630.
	\item Schwarzer, D. (2012). The Euro Area Crises, Shifting Power Relations and Institutional Change in the European Union. Global Policy, 3(s1), 28-41.
	\item Yiangou, J., O’keeffe, M., \& Gl{\"o}ckler, G. (2013). ‘Tough love’: how the ECB’s monetary financing prohibition pushes deeper euro area integration. Journal of European Integration, 35(3), 223-237.
	\item Tosun, J., Wetzel, A., \& Zapryanova, G. (2014). The EU in crisis: advancing the debate. Journal of European Integration, 36(3), 195-211.
	\item Krampf, A. (2014). From the Maastricht Treaty to Post-crisis EMU: The ECB and Germany as Drivers of Change. Journal of Contemporary European Studies, 22(3), 303-317.
\end{itemize}


\subsection*{PART IV: EXITING THE UNION?}


\subsubsection*{Week 11: Questions of Democracy - Transparency, legitimacy and accountability}

Study questions

\begin{itemize}
	\item Are the EU decision-making processes we have examined in this course democratic? How might we make them more democratic?
	\item How might we assess the democratic quality of EU decision making?
	\item Has the EU become more or less democratic and accountable over time?
\end{itemize}

\noindent Required Readings

\begin{itemize}
	\item Lelieveldt, H., \& Princen, S. (2015). The Politics of the European Union, 2nd Edition. Cambridge University Press. CH 12
	\item Follesdal, A., \& Hix, S. (2006). Why there is a democratic deficit in the EU: A response to Majone and Moravcsik. JCMS: Journal of Common Market Studies, 44(3), 533-562.
	\item Cross, J. P. (2013). Striking a pose: Transparency and position taking in the Council of the European Union. European Journal of Political Research, 52(3), 291-315.
	\item Dawson, M. (2015). The Legal and Political Accountability Structure of `Post-Crisis' EU Economic Governance. Journal of Common Market Studies, 53(5), 976-993.
\end{itemize}

\noindent Further reading

\begin{itemize}
	\item Hix, S. (2013). What's Wrong with the Europe Union and How to Fix it. John Wiley \& Sons.
	\item Maier, M., Maier, J., Baumert, A., Jahn, N., Krause, S., \& Adam, S. (2015). Measuring citizens’ implicit and explicit attitudes towards the European Union. European Union Politics.
	\item Chalmers, A. W., \& Dellmuth, L. M. (2015). Fiscal redistribution and public support for European integration. European Union Politics.
	\item V{\"o}ssing, K. (2015). Transforming public opinion about European integration: Elite influence and its limits. European Union Politics, 16(2), 157-175.
	\item Williams, C., \& Spoon, J. J. (2015). Differentiated party response: The effect of Euroskeptic public opinion on party positions. European Union Politics, 16(2).
	\item van Elsas, E., \& van der Brug, W. (2015). The changing relationship between left-right ideology and euroscepticism, 1973–2010. European Union Politics, 16(2), 194-215.
	\item B{\o}lstad, J. (2015). Dynamics of European integration: Public opinion in the core and periphery. European Union Politics, 16(1), 23-44.
	\item Fishkin, J. S., Luskin, R. C., \& Siu, A. (2014). Europolis and the European public sphere: Empirical explorations of a counterfactual ideal. European Union Politics, 15(3).
	\item Corbett, R. (2014). `European Elections are Second-Order Elections': Is Received Wisdom Changing?. JCMS: Journal of Common Market Studies, 52(6), 1194-1198.
	\item Cross, J. P. (2014). The seen and the unseen in legislative politics: explaining censorship in the Council of Ministers of the European Union. Journal of European Public Policy, 21(2), 268-285.
	\item Cross, J. P., \& B{\o}lstad, J. (2014). Openness and censorship in the European Union: An interrupted time series analysis. European Union Politics, 16(2).
	\item Hillebrandt, M. Z., Curtin, D., \& Meijer, A. (2014). Transparency in the EU Council of Ministers: An Institutional Analysis. European Law Journal, 20(1), 1-20.
\end{itemize}


\subsubsection*{Week 12: Negotiating Brexit}

Study questions

\begin{itemize}
	\item What factors explain the Brexit vote. Base your answer on empirical evidence from the academic literature.
	\item How has the decision to leave the EU affected the UK's ability to influence other countries in international negotiations. Provide examples and references to the literature to illustrate your arguments.
\end{itemize}

\noindent Required Readings

\begin{itemize}
	\item Hobolt, S. B. (2016). The Brexit vote: a divided nation, a divided continent. Journal of European Public Policy, 23(9), 1259-1277.
	\item Inglehart, R., \& Norris, P. (2016). Trump, Brexit, and the rise of populism: Economic have-nots and cultural backlash. HKS Working Paper No. RWP16-026. Available at SSRN: \url{https://ssrn.com/abstract=2818659}.
	\item Jensen, M. D., \& Snaith, H. (2016). When politics prevails: the political economy of a Brexit. Journal of European Public Policy, 23(9), 1302-1310.
	\item Oliver, T., \& Williams, M. J. (2016). Special relationships in flux: Brexit and the future of the US—EU and US—UK relationships. International Affairs, 92(3), 547-567.
	\item Dhingra, S., \& Sampson, T. (2016). Life after BREXIT: What are the UK’s options outside the European Union?. CEPBREXIT01. London School of Economics and Political Science, CEP, London, UK.
\end{itemize}

\noindent Further reading

\begin{itemize}
	\item Goodwin, M. J., \& Heath, O. (2016). The 2016 Referendum, Brexit and the Left Behind: An Aggregate-level Analysis of the Result. The Political Quarterly, 87(3), 323-332.
	\item Kierzenkowski, R., et al.  (2016). The Economic Consequences of Brexit: A Taxing Decision, OECD Economic Policy Papers, No. 16, OECD Publishing, Paris.
	\url{http://dx.doi.org/10.1787/5jm0lsvdkf6k-en}.
	\item Kaufmann, E. (2016). It’s NOT the economy, stupid: Brexit as a story of personal values. British Politics and Policy at LSE.
	\item Whitman, R. G. (2016). Brexit or Bremain: what future for the UK's European diplomatic strategy?. International Affairs, 92(3), 509-529.
	\item Ottaviano, G. I. P., Pessoa, J. P., Sampson, T., \& Van Reenen, J. (2014). Brexit or Fixit? The trade and welfare effects of leaving the European Union.
	\item Oliver, T. (2016). European and international views of Brexit. Journal of European Public Policy, 23(9), 1321-1328.
	\item Menon, A., \& Salter, J. P. (2016). Brexit: initial reflections. International Affairs, 92(6), 1297-1318.
	
\end{itemize}
