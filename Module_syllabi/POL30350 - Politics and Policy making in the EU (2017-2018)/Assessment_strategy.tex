%!TEX root = POL30350_syllabus.

\section*{Assessment strategy}

\subsection*{End of year exam}

	There will be a formal two-hour examination at the end of the semester as scheduled by the examination office. This examination will constitute 80\% of your final mark. Example exams demonstrating the structure and requirements of the end of year exam will be distributed in due course.

\subsection*{Blog-post assignment}

	In addition to the end of year exam, you will be required to submit a blog-post assignment of \textit{strictly} under 1000 words. This 1000 words does not include the bibliography. This blog-post assignment will constitute 20\% of your final mark in this course. Your blog-post assignment should be submitted no later than \textbf{3 p.m. on 27th October 2017}. You will be required to submit a copy of your assignment electronically via SafeAssign on Blackboard. Please read the SafeAssign guidelines in order to do this correctly.

	The blog-post assignments are intended to assess your substantive knowledge of European policy making and apply the insights you have gained from the course to current issues facing the EU and the policy makers active within this organisation. Good blog posts should show consistency in argument, clear structure, simple and direct writing, good punctuation and evidence of wide reading. The relatively short length of the post means that students must prioritise what they include in their text. They need to think carefully about what elements of their argument need to be developed and what elements can be given less attention. You should endeavour to include graphs, data, and existing research findings in the academic literature, with accurate referencing where appropriate. Your answers should demonstrate your ability to:

\begin{itemize}
	\item Identify important, relevant and recent developments in European policy making
	\item Identify the debates and academic authors in the discipline that address these developments
	\item Understand and apply the main theoretical approaches covered in the course to analyse EU politics
	\item Critically assess these approaches by drawing on the secondary literature on European policy making as well as empirical evidence and data
\end{itemize}

The blog post is also intended to assess your study skills. Your answers should demonstrate your ability to:

\begin{itemize}
	\item Draw \textit{selectively} on a range of relevant material, including existing literature on EU policy making, official documents, contemporary news sources, data, and the internet
	\item Understand, analyse and critically assess that material
	\item Use the material to make and sustain a well-structured line of argument
	\item Write in a concise and cogent style - The 1000 word limit forces you to prioritise information and think carefully about how to make a point in a relatively short amount of text.
\end{itemize}

	A selection of the best blog posts shall be published (with the author's consent) on the Dublin European Institute's blog found at \url{http://www.europedebate.ie}. This widely read and highly visible platform provides the opportunity for students to contribute to the public debate about European integration in a concrete manner to a national and international audience. It also allows students to publicly demonstrate their mastery of their chosen topic in a professional context, and as such represents a great opportunity to build a profile for future employers be they in academia, the public sector or the private sector. 

	A large selection of excellent example blog posts to inspire you can be found at \url{http://blogs.lse.ac.uk/europpblog/} and \url{http://www.europedebate.ie}. What you will notice about the blog posts on these sites is that they take insights gained from the political science literature, and use these insights to make a concise argument about current issues in EU integration and policy making. Please consult and explore this webpage at length to get an idea of what is expected of you.

	One important difference between the above mentioned blog posts and your assignment, is that you are required to include more detailed and complete referencing in your blog assignment. You want to demonstrate that you have read the relevant literature and can reference it where necessary. The best blog-post assignments will combine clarity of argument with well researched points and demonstrate a clear mastery of the relevant literature.

\subsection*{Grading Criteria}

In essence, markers assess four crucial elements in any answer:

\begin{itemize}
	\item Analysis/understanding
	\item Extent and use of reading
	\item Organisation/structure
	\item Writing proficiency
\end{itemize}

	The various grades/classifications reflect the extent to which an answer displays essential features of each of these elements (and their relative weighting). At its simplest: the better the analysis, the wider the range of appropriate sources consulted, the greater the understanding of the materials read, the clearer the writing style, and the more structured the argument, the higher will be the mark. 
