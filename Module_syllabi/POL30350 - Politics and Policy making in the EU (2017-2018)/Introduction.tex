%!TEX root = POL30350_syllabus.tex

\section*{Introduction}

	The European Union (EU) currently consists of 28 European countries, and over its history has developed significant policy-making powers across a whole range of policy areas. This module surveys the development of European policy making competencies and will develop students understanding of the functioning of the EU as a political system. The module approaches the EU as a decision-making body that brings together a multitude of actors with varying roles, powers, and preferences, with the aim of explaining how these actors get what they want in EU politics. We will focus on the negotiations between actors when updating treaties, engaging in policy making, dealing with crisis, and deciding on budget allocations. The manner in which institutions and actor behaviour impact upon these negotiations will the examined. The module is research led, in that it focuses upon the political science literature that seeks to explain various aspects of European politics and policy making. This is reflected in the extensive reading list provided with the course. The module is aimed at those wanting a full and detailed understanding of (1) the political institutions of the EU, (2) its policy-making processes, and (3) current and future challenges facing the EU as a political system. The module does not assume any prior knowledge of the EU or EU politics.