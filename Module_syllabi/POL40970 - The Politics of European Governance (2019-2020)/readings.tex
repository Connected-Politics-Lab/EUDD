%!TEX root = POL40970.tex

\section*{Module readings}

\subsection*{Required Readings:}

	The following text shall be used extensively throughout the module, so it is recommended that it is purchased:

\begin{itemize}
	\item Cini, M. \& P\'{e}rez-Sol\'{o}rzano Borrag\'{a}n, N. (2019). \textit{European Union Politics}. 6th Edition. Oxford: Oxford University Press.
\end{itemize}

\subsection*{Further reading:}

	In addition to the readings required for each topic, a series of recommended readings are also included in the syllabus. These readings will generally not be discussed in class, but are useful for those who wish to explore a particular topic in more detail. They will also be very useful when approaching the essay assignment, for which you are required to demonstrate a more in-depth understanding of your chosen topic. 

The following are good general texts on the EU and EU politics.

\begin{itemize}
	\item Nugent, N. (2006). \textit{Government and Politics of the European Union}. 6th ed. Duke University Press.
	\item Peterson, J. \& Bomberg, E. (1999). \textit{Decision-Making in the European Union}. New York: St. Martin’s Press.
	\item Peterson, J. \& Shackleton, M. (2006). \textit{The Institutions of the European Union}. 2nd ed. New York: Oxford University Press.
	\item Richardson, J. (2005). \textit{European Union: Power and Policy-Making}. 3rd ed. London: Routledge.
	\item Bache, I. \& George, S. (2006). \textit{Politics in the European Union}. 2nd ed. New York: Oxford University Press.
	\item Moravcsik, A. (1998). \textit{The Choice for Europe}. Ithaca: Cornell University Press.
	\item Rosamond, B. (2000). \textit{Theories of European Integration}. New York: St. Martin’s Press.
	\item Cini, M. \& Bourne, A.K. (2006). \textit{Palgrave Advances in European Union Studies}. New York: Palgrave.
	\item Jorgensen, K.E., Pollack, M.A., \& Rosamond, B. (2007). \textit{The Handbook of European Union Politics}. New York: Sage Publications.
\end{itemize}

In addition to these readings, students should keep up to date on current European affairs by reading daily newspapers, or one of the many websites devoted to EU politics. These websites include the following:

\begin{itemize}
	\item \url{http://www.euobserver.com}
	\item \url{http://europa.eu.int}
	\item \url{http://www.eupolitics.com}
	\item \url{http://www.ft.com}
	\item \url{http://blogs.lse.ac.uk/europpblog/}
	\item \url{http://blogs.lse.ac.uk/brexit/}
\end{itemize}

	Finally, each week a number of study questions are provided in order to guide your thinking about the topics being discussed and the required readings. These study questions will form the basis of the seminar topics covered and so careful consideration of the issues they raise should help you get to grasp with the issues being dealt with.

\section*{WEEKLY READING LIST}

\subsection*{PART I: THE HISTORY AND THEORIES OF EUROPEAN INTEGRATION}

\subsubsection*{Week 1: The Origins of the EU, and the development of Integration Theory.}

Study question

\begin{itemize}
	\item What are the main milestones that the integration process in Europe went through? How has the European integration project developed over time and how has the policy remit of the EU evolved?
	\item Why did neofunctionalists like Ernst Haas believe that the process of European integration, having begun so modestly, would snowball into an ongoing process of political integration? Are you convinced by the idea of spill-over effects proposed by the neofunctionalists as an explanation of the integration process?
	\item  By the late 1950s, the neofunctionalist spill-over processes predicted by Ernst Haas seemed to be in full swing, only to be disrupted in the 1960s by French President Charles de Gaulle. Looking beyond the personality of de Gaulle, (1) analyze why the decades of the 1960s and 1970s appeared to have falsified Haas’ neofunctionalism, and (2) assess whether the 1960s and 1970s were a period of progress, or rollback, or a mix of the two, in the integration process.
	\item Many saw the Single European Act (SEA) to be an effort to reinvigorate the integration process after several decades of stagnation, but the explanations for the SEA found in the required readings differ. What are the differences between the respective explanations for the SEA put forward by Zysman and Sandholtz and Moravcsik. Which argument is more persuasive, and why?
	\item Brexit is the first example of a member state trying to leave the EU. Can the existing theories of European integration shed light on the Brexit process and any likely outcomes?
\end{itemize}

\noindent Required Readings

\begin{itemize}
	\item Hix, S. and H\o yland, B. 2011. The political system of the European Union. Palgrave Macmillan. Chapter 1.
	\item Schmitter, P.C. (2006). Ernst B. Haas and the legacy of neofunctionalism. \textit{Journal of European Public Policy}. 12(2): 255–272.
	\item Stone Sweet, A. \& Sandholtz, W. (1997). European Integration and Supranational Governance. \textit{Journal of European Public Policy}. 4(3): 297-317.
	\item Moravcsik, A. (1998). \textit{The Choice for Europe}. Ithaca: Cornell University Press. Chapter 1.
\end{itemize}

\noindent Further reading

\begin{itemize}
	\item Cini, M. \& P\'{e}rez-Sol\'{o}rzano Borrag\'{a}n, N. (2019). \textit{European Union Politics}. 6th Edition. Oxford: Oxford University Press. Chapters 4 and 5.
	\item Jones, E., Kelemen, R.D. and Meunier, S. 2016. Failing forward? The Euro crisis and the incomplete nature of European integration. Comparative Political Studies, 49(7), pp.1010-1034.
	\item Marks, G. (2011). Europe and Its Empires: From Rome to the European Union. \textit{Journal of Common Market Studies}. 50(1): 1-20.
	\item Haas, E.B. (1958). \textit{The Uniting of Europe}. Stanford: Stanford University Press.
	\item Hoffmann, S. (1966). Obstinate or Obsolete? The Fate of the Nation-State and the Case of Western Europe. \textit{Daedalus}. 95(3).
	\item Deutsch, K.W. et al. (1957). \textit{Political Community in the North Atlantic Area: International Organization in the Light of Historical Experience}. Princeton: Princeton University Press.
	\item Diebold, W. (1959). \textit{The Schuman Plan: A Study in Economic Cooperation, 1950-1959}. New York: Praeger.
	\item Haas, E.B. (1976). Turbulent Fields and the Theory of Regional Integration. \textit{International Organization}. 30: 173-212.
	\item Monnet, J. (1978). \textit{Memoirs}. New York: Doubleday.
	\item Moravcsik, A. \textit{The Choice for Europe}. Ithaca: Cornell University Press. Chapter 2.
	\item Rittberger, B. (2001). Which Institutions for Post-War Europe? Explaining the Institutional Design of Europe’s First Community. \textit{Journal of European Public Policy}. 8(5): 673-708.
	\item Caporaso, J. (1998). Regional integration theory: understanding our past and anticipating our future. \textit{Journal of European Public Policy}. 5(1): 1–16.
	\item Mattli, W. (1999). Explaining regional integration outcomes. \textit{Journal of European Public Policy}. 6(1): 1–27.
	\item Risse, T. (2005). Neofunctionalism, European identity, and the puzzles of European integration. \textit{Journal of European Public Policy}. 12(2): 291–309.
	\item Corbett, R. (1987). The 1985 Intergovernmental Conference and the Single European Act. in Roy Pryce (ed.), \textit{The Dynamics of European Union}. New York: Croom Helm.
	\item Tranholm-Mikkelsen, J. (1991). Neofunctionalism: Obstinate or Obsolete? \textit{Journal of International Studies}. 20(1): 1-22.
	\item Moravcsik, A. (1993). Preferences and Power in the European Community: A Liberal Intergovernmentalist Approach. \textit{Journal of Common Market Studies}. 31: 473–524.
\end{itemize}


\subsubsection*{Week 2: Institutional change and Institutionalism}

Study Questions

\begin{enumerate}
	\item While the neofunctionalist-intergovernmentalist debate enlightened our understanding of the integration process, new theories of EU politics have since emerged focusing on institutions. What are these theories and what types of questions do they attempt to address? 
	\item How has the focus of the theoretical debate changed in light of these new institutionalist theories? 
	\item What do the different `institutionalism' have in common and how do they differ?
\end{enumerate}

\noindent Required Reading

\begin{itemize}
	\item Aspinwall, M.D. \& Schneider, G. (2000). Same menu, separate tables: The institutionalist turn in political science and the study of European integration. \textit{European Journal of Political Research}. 38(1): 1–36.
	\item Tsebelis, G. \& Garrett, G. (2001) The Institutional Foundations of Intergovernmentalism and Supranationalism. \textit{International Organization}. 55(2): 357-90.
	\item Pierson, P. (1996). The Path to European Integration: A Historical Institutionalist Analysis. \textit{Comparative Political Studies}. 29(2): 123-163.
	\item Checkel, J.T. \& Moravcsik, M. (2001). A Constructivist Research Programme in EU Studies? \textit{European Union Politics}. 2(2): 219-49.
\end{itemize}

\noindent Further reading

\begin{itemize}
	\item Cini, M. \& P\'{e}rez-Sol\'{o}rzano Borrag\'{a}n, N. (2019). \textit{European Union Politics}. 6th Edition. Oxford: Oxford University Press. Chapter 6.
	\item Moravcsik, A. \& Nicolaidis, K. (1998). Explaining the Treaty of Amsterdam: Interests, Influence, Institutions. \textit{Journal of Common Market Studies}. 37: 59–85.
	\item Garrett, G, \& Tsebelis, G. (1996). An Institutional Critique of Intergovernmentalism. \textit{International Organization}. 50(02): 269–99.
	\item Hix, S. (1994). The Study of the European Community: The Challenge to Comparative Politics. \textit{West European Politics}. 17: 1-30.
	\item Garrett, G. \& Tsebelis, G. (1996). An Institutional Critique of Intergovernmentalism. \textit{International Organization}. Vol. 50, No. 2, pp. 269-99.
	\item Christiansen, T., Jorgensen, K.E. \& Wiener, A. (1999). The Social Construction of Europe. \textit{Journal of European Public Policy}. 6: 528-44.
	\item Moravcsik, A. (1999). Is Something Rotten in the State of Denmark? Constructivism and European Integration. \textit{Journal of European Public Policy}. 6(4).
	\item Checkel, J.T. \& Moravcsik, A. (2001). A Constructivist Research Programme in EU Studies? \textit{European Union Politics}. 2(2): 219-49.
	\item Jupille, J. \& Caporaso, J.A. (1999). Institutionalism and the European Union: Beyond International Relations and Comparative Politics. \textit{Annual Review of Political Science}. 2: 429–44.
	\item Aspinwall, M. \& Schneider, G. eds. (2000). \textit{The Rules of Integration: Institutionalist Approaches to the Study of Europe}. Manchester: Manchester University Press.
\end{itemize}


\subsection*{PART II: THE POLITICAL INSTITUTIONS OF THE EU AND ITS POLICY-MAKING PROCESS}

\subsubsection*{Week 3: Negotiating the treaties}

Study questions

\begin{itemize}
	\item What are the sources of states’ bargaining power during treaty negotiations? How do domestic constraints condition states’ bargaining strength at IGCs?  
    \item There is conflicting evidence regarding which players and what factors are most likely to affect the outcome of a treaty. What is the role of supranational actors in explaining IGC bargains?
    \item How has the Treaty of Lisbon shifted the EU’s institutional balance of power between the major institutions of the EU?
\end{itemize}

\noindent Required Readings

\begin{itemize}
	\item Cini, M. \& P\'{e}rez-Sol\'{o}rzano Borrag\'{a}n, N. (2019). \textit{European Union Politics}. Oxford: Oxford University Press. Chapter 3.
	\item D{\"o}rfler, T., Holzinger, K., \& Biesenbender, J. (2017). Constitutional Dynamics in the European Union: Success, Failure, and Stability of Institutional Treaty Revisions. International Journal of Public Administration, 1-13.
	\item Slapin, J. B. (2008). Bargaining power at Europe's intergovernmental conferences: testing institutional and intergovernmental theories. \textit{International Organization}. 62(01): 131-162.
	\item Finke, D. (2009). Challenges to intergovernmentalism: an empirical analysis of EU treaty negotiations since Maastricht. \textit{West European Politics}. 32(3): 466-495.
\end{itemize}

\noindent Further reading

\begin{itemize}
	\item Slapin, J. B. (2006). Who is powerful? Examining preferences and testing sources of bargaining strength at European intergovernmental conferences. \textit{European Union Politics}. 7(1): 51-76.
	\item Schneider, C. J. (2013). Globalizing electoral politics: Political competence and distributional bargaining in the European Union. World Politics, 65(3), 452-490.
	\item Johansson, K. M. (2016). The role of Europarties in EU treaty reform: Theory and practice. Acta Politica, 1-20.
	\item Panke, D. (2017). Speech is silver, silence is golden? Examining state activity in international negotiations. The Review of International Organizations, 12(1), 121-146.
	\item Finke, D., K{\"o}nig, T., Proksch, S.O., \& Tsebelis, G. (2012). Reforming the European Union: realizing the impossible. New Jersey: Princeton University Press.
	\item Slapin, J. B. (2009). Exit, voice, and cooperation: Bargaining power in international organizations and federal systems. \textit{Journal of Theoretical Politics}. 21(2): 187-211.
	\item Finke, D. (2010). \textit{European integration and its limits: intergovernmental conflicts and their domestic origins}. ECPR Press.
	\item Franchino, F. (2013). Challenges to liberal intergovernmentalism. \textit{European Union Politics}. 14(2): 324-337.
	\item Finke, D. (2009). Domestic Politics and European Treaty Reform Understanding the Dynamics of Governmental Position-Taking. \textit{European Union Politics}. 10(4): 482-506.
	\item Reinhard, J., Biesenbender, J., \& Holzinger, K. (2014). Do arguments matter? Argumentation and negotiation success at the 1997 Amsterdam Intergovernmental Conference. \textit{European Political Science Review}. 6(02): 283-307.
	\item Finke, D. (2013). Reforming International Institutions: The Domestic Origins and Conditional Logic of Governmental Reform Preferences. \textit{International Studies Quarterly}. 57(2): 288-302.
	\item Reinhard, J. (2012). Because we are all Europeans!’When do EU Member States use normative arguments?. \textit{Journal of European Public Policy}. 19(9): 1336-1356.
\end{itemize}

\subsubsection*{Week 4: The EU as a Political System; the Commission and Executive Politics} 

Study Questions

\begin{enumerate}
	\item Why do Member States choose to delegate policy-making powers to the supranational level? What types of problems does such delegation solve and what types of problems arise from the delegation decision?
	\item Some recent studies find congruence between the Commission and member states’ policy positions suggesting that the Commission might not be “completely independent in the performance of their duties” as specified in the treaty and not quite as supranational as it has been often portrayed. To what can we attribute this policy congruence? i.e., what are some of the key factors that constrain the Commission’s independence and condition its policy positions, as identified in the readings?
	
\end{enumerate}

\noindent Required Reading

\begin{itemize}
	\item Cini, M. \& P\'{e}rez-Sol\'{o}rzano Borrag\'{a}n, N. (2019). \textit{European Union Politics}. Oxford: Oxford University Press. Chapter 10-11.
	\item Pollack, M.A., (1997). Delegation, agency, and agenda setting in the European Community. \textit{International Organization}. 51(1): 99–134.
	\item Egeberg, M. (2014). The European Commission: From agent to political institution. Public Administration, 92(1), 240-246.
	\item Wonka, A. (2007). Technocratic and Independent? The Appointment of European Commissioners and its Policy Implications. \textit{Journal of European Public Policy}. 14(2): 169-89.
	\item Kl{\"u}ver, H., Mahoney, C. and Opper, M. (2015). Framing in context: how interest groups employ framing to lobby the European Commission. Journal of European Public Policy, 22(4), 481-498.
\end{itemize}

\noindent Further reading

\begin{itemize}
	\item Ringe, N. (2005). Policy Preference Formation in Legislative Politics: Structures, Actors, and Focal Points. \textit{American Journal of Political Science}. 49(4): 731–45.
	\item Thomson, R. (2008). National Actors in International Organizations: The Case of the European Commission. \textit{Comparative Political Studies}. 41: 169-92.
	\item Hooghe, L. (2002). \textit{The European Commission and the Integration of Europe}. New York: Cambridge University Press.
	\item Pollack, M.A. (2003). \textit{The Engines of European Integration: Delegation, Agency and Agenda Setting in the European Union}. New York: Oxford University Press.
	\item Egeberg, M. (2006). Executive Politics as Usual: Role Behaviour and Conflict Dimensions in the College of European Commissioners. \textit{Journal of European Public Policy}. 13(1): 1-15.
	\item Franchino, F. (2009). Experience and the distribution of portfolio payoffs in the European Commission. \textit{European Journal of Political Research}. 48(1): 1-30.
	\item Hooghe, L. (1999). Images of Europe: Orientations to European Integration Among Senior Officials of the Commission. \textit{British Journal of Political Science}. 29: 345–367.
	\item Hooghe, L. (2002). \textit{The European Commission and the Integration of Europe}. New York: Cambridge University Press. 
	\item Hooghe, L. (2005). Several Roads Lead to International Norms, But Few via International Socialization: A Case Study of the European Commission. \textit{International Organization}. 59: 861-898.
	\item Rasmussen, A. (2007). Challenging the Commission’s Right of Initiative? Conditions for Institutional Change and Stability. \textit{West European Politics}. 30(2): 244-64.
	\item Egeberg, M., Gornitzka, Å., \& Trondal, J. (2014). A not so technocratic executive? Everyday interaction between the European Parliament and the Commission. West European Politics, 37(1), 1-18.
	\item Christensen, J. (2015). Recruitment and Expertise in the European Commission. West European Politics, 38(3), 649-678.
	\item Egeberg, M., Trondal, J., \& Vestlund, N. M. (2015). The quest for order: unravelling the relationship between the European Commission and European Union agencies. Journal of European Public Policy, 22(5), 609-629.
	\item Wonka, A. (2008). ‘Decision-Making Dynamics in the European Commission: Partisan, National or Sectoral?’, Journal of European Public Policy 15(8), pp.1145–63.
	\item Hartlapp, M., Metz, J., \& Rauh, C. (2014). Which policy for Europe?: power and conflict inside the European Commission. Oxford University Press.
	\item Egeberg, M., Trondal, J., \& Vestlund, N. (2014). Situating EU agencies in the political-administrative space. ARENA Working Papers 6.
    \item Kassim, H. (2008). ‘“Mission Impossible”, but Mission Accomplished: the Kinnock Reforms and the European Commission’, Journal of European Public Policy 15(5), 648–68
    \item Smith, A., (2014). How the European Commission’s Policies Are Made: Problematization, Instrumentation and Legitimation, Journal of European Integration, 36(1), 55-72.
    \item Schmidt, S.K. (2000). ‘Only an Agenda Setter? The European Commission’s Power Over the Council of Ministers’, European Union Politics 1(1), pp.37–61
    \item Peterson, J. (2012). “The College of Commissioners.” In: John Peterson \& Michael Shackleton (eds.). The Institutions of the European Union. 3rd edition. Oxford: Oxford University Press: 96-123.
    \item Quittkat, C. (2011). The European Commission's Online Consultations: A Success Story? Journal of Common Market Studies 49(3): 653-674.
	\item Eising, R., Rasch, D., \& Rozbicka, P. (2015). Institutions, policies, and arguments: context and strategy in EU policy framing. Journal of European Public Policy, 22(4), 516-533.
\end{itemize}

\subsubsection*{Week 5: Legislative Politics \& Policy making (The Council of Ministers and the European Parliament)}

Study Questions

\begin{enumerate}
	\item Is the European Parliament a ``normal" Parliament? In what sense is it ``normal" and it what sense (if any) does it remain unique or sui generis? How has this changed over time as the EU has been reformed?
	\item What drives policy outcomes in the Council of Ministers?
	\item How is the impact of the Council, Parliament and Commission mediated through the institutional structure and rules of the decision-making process? What factors impact upon a legislative actors ability to influence legislative outcomes? 
	\item How successful have the attempts to formally model the legislative process been at predicting legislative outcomes? Is this approach suitable for the task at hand? What aspects of the decision-making process might such models miss?
	\item How might informal political arrangements reduce the difficulty associated with reaching agreements under more formal legislative procedures?
\end{enumerate}

\noindent Required Reading

\begin{itemize}
	\item Cini, M. \& P\'{e}rez-Sol\'{o}rzano Borrag\'{a}n, N. (2019). \textit{European Union Politics}. Oxford: Oxford University Press. Chapter 12.
	\item Hix, S., Noury, A., \& Roland, G. (2002). A `Normal' Parliament? Party Cohesion and Competition in the European Parliament, 1979-2001. \textit{EPRG Working Paper}. No 9, available on-line at: \url{http://www.lse.ac.uk/Depts/eprg/working-papers.htm}.
	\item Thomson, R., Stokman, F.N., Achen, C.H., \& K\"{o}nig, T. (2006). \textit{The European Union Decides}. New York: Cambridge University Press. Chapters 1,2,11.
	\item Bailer, S., Mattila, M., \& Schneider, G. (2015). Money makes the EU go round: The objective foundations of conflict in the Council of Ministers. JCMS: Journal of Common Market Studies, 53(3), 437-456.
	\item Cross, J. P., \& Hermansson, H. (2017). Legislative amendments and informal politics in the European Union: A text reuse approach. European Union Politics, 18(4), 581-602.
\end{itemize}

\noindent Further reading

\begin{itemize}
	\item Thomson, R. \& Hosli, M. (2006). Who Has Power in the EU? \textit{Journal of Common Market Studies}. 44(2): 391-417.
	\item Slapin, J.B. (2014). Measurement, Model Testing, and Legislative Influence in the European Union. \textit{European Union Politics}. 15(1): 24–42.
	\item Rasmussen, A., \& Reh, C. (2013). The consequences of concluding codecision early: trilogues and intra-institutional bargaining success. Journal of European Public Policy, 20(7), 1006-1024.
	\item Reh, C., Héritier, A., Bressanelli, E., \& Koop, C. (2013). The informal politics of legislation explaining secluded decision making in the European Union. Comparative Political Studies, 46(9), 1112-1142.
	\item Cross, J. P., \& Hermansson, H. (2017). Legislative amendments and informal politics in the European Union: A text reuse approach. European Union Politics, (Ahead of print).
	\item Hix, S. \& H\o yland, B. (2011). \textit{The Political System of the European Union}. Chapter 3.
	\item Leinaweaver, J. \& Thomson, R. (2014). Testing Models of Legislative Decision-Making with Measurement Error: the Robust Predictive Power of Bargaining Models Over Procedural Models. \textit{European Union Politics}. 15(1): 43–58.
	\item Tsebelis, G. \& Garrett, G. (2000). Legislative Politics in the European Union. \textit{European Union Politics}. 1: 9-36.
	\item Tsebelis, G. \& Garrett, G. (2001). The Institutional Foundations of Intergovernmentalism and Supranationalism in the European Union. \textit{International Organization}. 55: 357-90.
	\item Corbett, R. (2000). Academic Modelling of the Codecision Procedure: A Practitioner’s Puzzled Reaction. \textit{European Union Politics}. 1: 373–78.
	\item Kreppel, A. (2001). T\textit{he European Parliament and Supranational Party System: A Study in Institutional Development}. New York: Cambridge University Press.
	\item Cross, J.P. (2013). Everyone’s a winner (almost): Bargaining success in the Council of Ministers of the European Union. \textit{European Union Politics}. 14(1): 70-94.
	\item Cross, J.P. (2012). Interventions and negotiation in the Council of Ministers of the European Union. \textit{European Union Politics}. 13(1): 47-69.
	\item Hix, S., Raunio, T. \& Scully, R. eds., (2003). Fifty Years on: Research on the European Parliament. \textit{Journal of Common Market Studies}. 43(2).
	\item Carubba, C. et al. (2003). Selection Bias in the Use of Roll-Call Votes to Study Legislative Behavior. \textit{EPRG Working Paper}. No. 11, available on-line at: \url{http://www.lse.ac.uk/Depts/eprg/workingpapers.
htm}.
	\item Hix, S., Noury, A., \& Roland, G. (2005). Power to the Parties: Cohesion and Competition in the European Parliament, 1979-2001. \textit{British Journal of Political Science}. 35(2): 209-234
	\item \textit{European Parliament Research Group Working Papers Series}. Available on-line at: \url{http://www.lse.ac.uk/Depts/eprg/working-papers.htm}.
	\item Hayes-Renshaw, F. \& Wallace, F. (2006). \textit{The Council of Ministers}. 2nd revised and updated edition (London: Palgrave).
	\item Thomson, R. (2011). \textit{Resolving controversy in the European Union: legislative decision-making before and after enlargement}. Cambridge: Cambridge University Press.
\end{itemize}


\subsubsection*{Week 6: Judicial Politics}

\textit{The ECJ, Judicial Politics and Compliance}

Study Question

\begin{enumerate}
	\item Describe the powers of the ECJ. Do these powers make the ECJ an independent court? What limits (both formal and informal) exist on the powers of the ECJ?
	\item What do we mean by the term ``judicial activism"? Does the ECJ engage in judicial activism often? How have member states reacted to examples of judicial activism by the ECJ?
	\item Carruba et al. and Stone Sweet \& Brunell disagree on the extent to which member states can act collectively to constrain the ECJ. Which account of ECJ power and influence is more convincing and why?
	\item We observe significant variation in the levels of compliance with EU law both between Member States and over time. What can explain this variation?
\end{enumerate}

\noindent Required Readings

\begin{itemize}
	\item Cini, M. \& P\'{e}rez-Sol\'{o}rzano Borrag\'{a}n, N. (2019). \textit{European Union Politics}. Oxford: Oxford University Press. Chapter 13.
	\item Stone Sweet, A. (2010). The European Court of Justice and the Judicialization of EU Governance. \textit{Living Reviews in European Governance}. 5(2).
	\item Carrubba, C.J., Gabel, M. \& Hankla, C. (2008). Judicial behavior under political constraints: Evidence from the European Court of Justice. \textit{American Political Science Review}. 102(4): 435–452.
	\item Stone Sweet, A. \& Brunell, T. (2012). The European Court of Justice, State Noncompliance, and the Politics of Override. \textit{American Political Science Association}. 106(01): 204–213.
	\item Zhelyazkova, A. (2013). Complying with EU directives' requirements: the link between EU decision-making and the correct transposition of EU provisions. Journal of European Public Policy, 20(5), 702-721.
\end{itemize}

\noindent Further reading

\begin{itemize}
	\item Hix, S. \& H\o yland, B. (2011). \textit{The Political System of the European Union}. Chapter 12.
	\item Angelova, M., Dannwolf, T. \& K\"{o}nig, T. (2012). How Robust are Compliance Findings. \textit{Journal of European Public Policy}. 19(8): 1269-91.
	\item Garrett, G. (1992). International Cooperation and Institutional Choice: The European Community’s Internal Market. \textit{International Organization}. 46: 533–60.
	\item Burley, A-M. \& Mattli, W. (1993). Europe Before the Court: A Political Theory of Legal Integration. \textit{International Organization}. 47: 41–76.
	\item Garrett, G. (1995). The Politics of Legal Integration in the European Union. \textit{International Organization}. 49: 171–81.
	\item Mattli, W. \& Slaughter, A-M. (1995). Law and Politics in the European Union: A Reply to Garrett. \textit{International Organization}. 49: 183–90.
	\item Garrett, G., Kelemen, D. \& Schulz, H. (1998). The European Court of Justice, National Governments, and Legal Integration in the European Union. \textit{International Organization}. 52: 149–76.
	\item Stone Sweet, A. \& Brunell, T.L. (1998). Constructing a Supranational Constitution: Dispute Resolution and Governance in the European Community. \textit{American Political Science Review}. 92: 63–81.
	\item Weiler, J.H.H. (2000). \textit{The Constitution of Europe: ``Do the New Clothes Have an Emperor?" and Other Essays on European Integration}. New York: Cambridge University Press.
	\item Tallberg, J. (2004). European governance and supranational institutions: making states comply. Routledge.
	\item Alter, K. (2010). The European Court's political power: selected essays. Oxford University Press.
	\item Hartlapp, M. (2007). On Enforcement, Management and Persuasion: Different Logics of Implementation Policy in the EU and the ILO*. JCMS: Journal of Common Market Studies, 45(3), 653-674.
	\item Falkner, G., Treib, O., \& Holzleithner, E. (2008). Compliance in the enlarged European Union: living rights or dead letters?. Ashgate Publishing, Ltd..
	\item Dimitrova, A. L., \& Toshkov, D. (2009). Post-accession compliance between administrative co-ordination and political bargaining. European Integration online Papers (EIoP), (2).
	\item Falkner, G. (2007). Time to Discuss: Data to Crunch or Problems to Solve? A Rejoinder to Robert Thomson. \textit{West European Politics}. 30(5): 1009–1021.
	\item Hartlapp, M. \& Falkner, G. (2009). Problems of Operationalization and Data in EU Compliance Research. \textit{European Union Politics}. 10(2): 281–304.
	\item Thomson, R. (2010). Opposition through the back door in the transposition of EU directives. \textit{European Union Politics}. 11(4): 577–596.
	\item Thomson, R. (2009). Same effects in different worlds: the transposition of EU directives. \textit{Journal of European Public Policy}. 16(1): 1–18.
	\item Thomson, R. (2007). Time to Comply: National Responses to Six EU Labour Market Directives Revisited. \textit{West European Politics}. 30(5): 987–1008.
	\item Thomson, R., Torenvlied, R. \& Arregui, J. (2007). The Paradox of Compliance: Infringements and Delays in Transposing European Union Directives. \textit{British Journal of Political Science}. 37(04).
	\item Toshkov, D. (2008). Embracing European law compliance with EU directives in Central and Eastern Europe. European Union Politics, 9(3), 379-402.
	\item Toshkov, D. (2010). Taking stock: a review of quantitative studies of transposition and implementation of EU law. Institute for European Integration Research.
	\item Angelova, M., Dannwolf, T., \& K{\"o}nig, T. (2012). How robust are compliance findings? A research synthesis. Journal of European Public Policy, 19(8), 1269-1291.
	\item Zhelyazkova, A. \& Torenvlied, R. (2009). The Time-Dependent Effect of Conflict in the Council on Delays in the Transposition of EU Directives. \textit{European Union Politics}. 10(1): 35–62.
	\item Thomann, E., \& Sager, F. (2017). Moving beyond legal compliance: innovative approaches to EU multilevel implementation. Journal of European Public Policy, 24(9), 1252-1268.
	\item Thomann, E., \& Zhelyazkova, A. (2017). Moving beyond (non-) compliance: the customization of European Union policies in 27 countries. Journal of European Public Policy, 24(9), 1269-1288.
	\item Mastenbroek, E. (2017). Guardians of EU law? Analysing roles and behaviour of Dutch legislative drafters involved in EU compliance. Journal of European Public Policy, 24(9), 1289-1307.
	\item Gollata, J. A., \& Newig, J. (2017). Policy implementation through multi-level governance: analysing practical implementation of EU air quality directives in Germany. Journal of European Public Policy, 24(9), 1308-1327.
	\item Scholten, M. (2017). Mind the trend! Enforcement of EU law has been moving to ‘Brussels’. Journal of European Public Policy,  24(9), 1348-1366.
	\item Heidbreder, E. G. (2017). Strategies in multilevel policy implementation: moving beyond the limited focus on compliance. Journal of European Public Policy,  24(9), 1367-1384.
	\item Thomann, E., \& Sager, F. (2017). Toward a better understanding of implementation performance in the EU multilevel system. Journal of European Public Policy, 24(9), 1385-1407.
	\item Finke, D., \& Dannwolf, T. (2015). Who let the dogs out? The effect of parliamentary scrutiny on compliance with EU law. Journal of European Public Policy, 22(8), 1127-1147.
\end{itemize}


\subsubsection*{Week 7: Interest representation: Lobbying in Brussels}

Study Question

\begin{itemize}
	\item Describe the different access point through which lobbyists attempt to influence EU policy making. Do the institutional structures of the Commission, EP, and Council aid or hinder lobbyists in their lobbying attempts?
	\item What different types of factors have been identified as influencing lobbying success?
	\item What attempts have been made to regulate EU lobbying? Have these attempts been successful?
\end{itemize}

\noindent Key reading

\begin{itemize}
	\item Cini, M. \& P\'{e}rez-Sol\'{o}rzano Borrag\'{a}n, N. (2019). \textit{European Union Politics}. Oxford: Oxford University Press. Chapter 14.
	\item European Commission for Democracy Through Law (Venice Commission). (2013). \textit{Report on the Role of Extra-Institutional Actors in the Democratic System}. Available here: \url{http://www.venice.coe.int/webforms/documents/default.aspx?pdffile=CDL-AD(2013)011-e}
	\item Fl{\"o}the, L. and Rasmussen, A., 2018. Public voices in the heavenly chorus? Group type bias and opinion representation. Journal of European Public Policy, pp.1-19.
	\item Binderkrantz, A.S. and Rasmussen, A., 2015. Comparing the domestic and the EU lobbying context: perceived agenda-setting influence in the multi-level system of the European Union. Journal of European Public Policy, 22(4), pp.552-569.
	\item Bouwen, P. (2004): Exchanging access goods for access: A comparative study of business lobbying in the European Union Institutions. \textit{European Journal of Political Research}. 43: 337–369.

\end{itemize}

\noindent Further reading

\begin{itemize}
	\item Carroll, B.J. and Rasmussen, A., 2017. Cultural capital and the density of organised interests lobbying the European Parliament. West European Politics, 40(5), pp.1132-1152.
	\item Marshall, D. (2010) Who to Lobby and When: Institutional Determinants of Interest Group Strategies in European Parliament Committees. \textit{European Union Politics}. 11(4): 553-575.
	\item Mahoney, C. \& Baumgartner, F., (2008). Converging perspectives on interest group research in Europe and America. \textit{West European Politics}. 31(6): 1253–1273.
	\item Mahoney, C. (2007). Lobbying Success in the United States and the European Union. \textit{Journal of Public Policy}. 27(1): 35-56.
	\item Beyers, J. (2008) Policy Issues, Organizational Format and the Political Strategies of Interest Organizations. \textit{West European Politics}. 31(6): 1188-1211.
	\item Beyers, J. \& Kerremans, B. (2004). Bureaucrats, Politicians, and Societal Interests: How Is European Policy Making Politicized?” \textit{Comparative Political Studies}. 37(10): 1119-1150.
	\item B\"{o}rzel, T., Heard-Laur\'{e}ote, K. (2009). Networks in EU Multi-level governance: Concepts and Contributions. \textit{Journal of Public Policy}. 29(2): 135-152.
	\item Coen, D. (2007). Empirical and theoretical studies in EU lobbying. \textit{Journal of European Public Policy}. 14(3): 333-345.
	\item Coen, D \& Richardson, J (eds.) (2009). \textit{Lobbying the European Union}. Oxford: Oxford University Press.
	\item D\"{u}r, A. (2008). Interest groups in the European Union: How Powerful are They? \textit{West European Politics}. 31(6): 1212-1230.
	\item Eising, R. (2007). The access of business interests to EU institutions: towards elite pluralism? \textit{Journal of European Public Policy}. 14(3): 384-403.
	\item Eising, R. (2008). Interest Group in EU policymaking. \textit{Living Reviews in European Governance}. Vol. 3 (\url{http://europeangovernance.livingreviews.org/Articles/lreg-2008-4/} ).
	\item Greenwood, J (2007a). \textit{Interest Representation in the European Union}. Hampshire: Palgrave Macmillan.
	\item Greenwood, J. (2007b). Review Article: Organized Civil Society and Democratic Legitimacy in the EU. \textit{British Journal of Political Science}. 37: 333-35. 
	\item Kl\"{u}ver, H. (2010). Measuring Interest Group Influence using Quantitative Text Analysis. \textit{European Union Politics}. 10(4): 535- 549.
	\item Mahoney, C. (2008). \textit{Brussels versus the Beltway. Advocacy in the United States and the European Union}. Washington DC: Georgetown University Press.
	\item Quittkat, C. (2011). The European Commission’s Online Consultations: a Success Story? \textit{Journal of Common Market Studies}. 49(3): 653-674.
	\item Skodvin, T., Gullberg, A.T., \& Aakre, S. (2010). Target-group influence and political feasibility: the case of climate policy design in Europe. \textit{Journal of European Public Policy}. 17(6): 854-873.
\end{itemize}


\subsubsection*{Week 8: Peer-review workshop}

Exercise

\begin{enumerate}
	\item Write a 2-page review of the draft paper distributed by the lecturer. Utilise the guidelines provided to do so. These reviews will be considered in class and used as the basis for the seminar.
\end{enumerate}

\noindent Required Readings

\begin{itemize}
	\item Distributed draft paper
	\item Guidelines for workshop
\end{itemize}


\subsubsection*{Week 9: Brexit, public opinon, and the future of the EU} 

Study Questions

\begin{enumerate}
	\item What drives public opinion about the EU? Can traditional explanations of public opinion help us understand the Brexit vote?
	\item The balance of power in the Brexit negotiations is heavily in favour of the EU, why is this? What (if anything) might the UK be able to do to mitigate its disadvantaged bargaining position?
	\item What are the implications of Brexit for the future of the EU?
\end{enumerate}

\noindent Required Readings

\begin{itemize}
	\item Cini, M. \& P\'{e}rez-Sol\'{o}rzano Borrag\'{a}n, N. (2019). \textit{European Union Politics}. Oxford: Oxford University Press. Chapter 27 \& 28.
	\item Jennings, W. and Lodge, M. 2018. Brexit, the tides and Canute: the fracturing politics of the British state, Journal of European Public Policy.
	\item Manners, I. 2018. Political Psychology of European Integration: The (Re)production of Identity and Difference in the Brexit Debate. Political Psychology, 39, pp.1213-1232.
	\item Hix, S. 2018. Brexit: Where is the EU–UK Relationship Heading?. JCMS: Journal of Common Market Studies, 56, pp.11–27.
	\item Richards, L. , Heath, A. and Carl, N. 2018. Red Lines and Compromises: Mapping Underlying Complexities of Brexit Preferences. The Political Quarterly, 89, pp.280-290.
	\item Laffan, B. 2018. Brexit: Re-opening Ireland's English Question. The Political Quarterly, 89, pp.568-575. 
	\item Springford, J. 2018. Theresa May's Irish trilemma. \url{https://www.cer.eu/insights/theresa-mays-irish-trilemma}.
\end{itemize}

\noindent Further reading

\begin{itemize}
	\item Richardson, J. 2018. Brexit: The EU Policy-Making State Hits the Populist Buffers. The Political Quarterly, 89, pp.118-126.
	\item Hobolt, S. B. (2016). The Brexit vote: a divided nation, a divided continent. Journal of European Public Policy, 23(9), 1259-1277.
	\item Hix, S. 2018. Brexit: Where is the EU–UK Relationship Heading?. JCMS: Journal of Common Market Studies, 56, pp.11–27.
	\item Lavery, S., McDaniel, S. and Schmid, D., 2018. Finance fragmented? Frankfurt and Paris as European financial centres after Brexit. Journal of European Public Policy, pp.1-19.
	\item Jennings, W. and Lodge, M. 2018. Brexit, the tides and Canute: the fracturing politics of the British state, Journal of European Public Policy
	\item Qvortrup, M. 2016. Referendums on Membership and European Integration 1972–2015. The Political Quarterly, 87, pp.61-68.
	\item Henderson, A. , Jeffery, C. , Liñeira, R. , Scully, R. , Wincott, D. and Wyn Jones, R. 2016. England, Englishness and Brexit. The Political Quarterly, 87, pp.187-199.
	\item Renwick, A. , Allan, S. , Jennings, W. , Mckee, R. , Russell, M. and Smith, G. 2018. What Kind of Brexit do Voters want? Lessons from the Citizens’ Assembly on Brexit. The Political Quarterly, 89, pp.649-658.
	\item Laffan, B. 2018. Brexit: Re-opening Ireland's English Question. The Political Quarterly, 89, pp.568-575. 
	\item Chen W, Los B, McCann P, Ortega-Argilés R, Thissen M, van Oort F. 2018. The continental divide? Economic exposure to Brexit in regions and countries on both sides of The Channel. Papers in Regional Science. 97, pp.25–54.
	\item Richardson, J. 2018. Brexit: The EU Policy-Making State Hits the Populist Buffers. The Political Quarterly, 89, pp.118-126.
	\item Richards, L. , Heath, A. and Carl, N. 2018. Red Lines and Compromises: Mapping Underlying Complexities of Brexit Preferences. The Political Quarterly, 89, pp.280-290.
	\item Hantzsche, A. , Kara, A. and Young, G. 2018. The Economic Effects of the UK Government's Proposed Brexit Deal. The World Economy, Accepted Author Manuscript.
	\item Gasiorek, M. , Serwicka, I. and Smith, A. 2019. Which Manufacturing Industries and Sectors Are Most Vulnerable to Brexit?. The World Economy, Accepted Author Manuscript.
	\item Manners, I. 2018. Political Psychology of European Integration: The (Re)production of Identity and Difference in the Brexit Debate. Political Psychology, 39, pp.1213-1232.
	\item Warlouzet, L. 2018. Britain at the Centre of European Co-operation (1948–2016). JCMS: Journal of Common Market Studies, 56, pp.955–970.
	\item Dhingra, S., \& Sampson, T. 2016. Life after BREXIT: What are the UK’s options outside the European Union?. CEPBREXIT01. London School of Economics and Political Science, CEP, London, UK.
	\item Inglehart, R., \& Norris, P. (2016). Trump, Brexit, and the rise of populism: Economic have-nots and cultural backlash. HKS Working Paper No. RWP16-026. Available at SSRN: \url{https://ssrn.com/abstract=2818659}.
	\item Goodwin, M. J., \& Heath, O. (2016). The 2016 Referendum, Brexit and the Left Behind: An Aggregate-level Analysis of the Result. The Political Quarterly, 87(3), 323-332.
	\item Jensen, M. D., \& Snaith, H. (2016). When politics prevails: the political economy of a Brexit. Journal of European Public Policy, 23(9), 1302-1310.
	\item Oliver, T., \& Williams, M. J. (2016). Special relationships in flux: Brexit and the future of the US—EU and US—UK relationships. International Affairs, 92(3), 547-567.
	\item Kierzenkowski, R., et al.  (2016). The Economic Consequences of Brexit: A Taxing Decision, OECD Economic Policy Papers, No. 16, OECD Publishing, Paris.
	\url{http://dx.doi.org/10.1787/5jm0lsvdkf6k-en}.
	\item Kaufmann, E. (2016). It’s NOT the economy, stupid: Brexit as a story of personal values. British Politics and Policy at LSE.
	\item Whitman, R. G. (2016). Brexit or Bremain: what future for the UK's European diplomatic strategy?. International Affairs, 92(3), 509-529.
	\item Ottaviano, G. I. P., Pessoa, J. P., Sampson, T., \& Van Reenen, J. (2014). Brexit or Fixit? The trade and welfare effects of leaving the European Union.
	\item Oliver, T. (2016). European and international views of Brexit. Journal of European Public Policy, 23(9), 1321-1328.
	\item Menon, A., \& Salter, J. P. (2016). Brexit: initial reflections. International Affairs, 92(6), 1297-1318.
\end{itemize}


\subsubsection*{Week 10: The EU budget}

Study questions

\begin{itemize}
	\item What actors and institutions are involved in negotiating the EU budget?
	\item How are the powers of these actors to shape the EU budget influenced by the institutional rules structuring the budget-making process?
	\item How has the budget-making process in the EU evolved over time? 
\end{itemize}

\noindent Required Readings

\begin{itemize}
	\item Citi, M. (2015). European Union budget politics: Explaining stability and change in spending allocations. \textit{European Union Politics}. 16(2). 260-280.
	\item Crombez, C. \& Høyland, B. (2015). The budgetary procedure in the European Union and the implications of the Treaty of Lisbon. \textit{European Union Politics}. 16(1): 67–89.
	\item Benedetto, G. (2017). Power, money and reversion points: the European Union's annual budgets since 2010. Journal of European Public Policy, 24(5), 633-652.
	\item Schneider, C. (2013). Globalizing Electoral Politics: Political Competence and Distributional Bargaining in the European Union. World Politics, 65(3), 452-490. 
\end{itemize}

\noindent Further reading

\begin{itemize}
	\item Gehring, K. and Schneider, S.A. 2018. Towards the Greater Good? EU Commissioners' Nationality and Budget Allocation in the European Union. American Economic Journal: Economic Policy, 10(1), pp.214-39.
	\item Dellmuth, L. M. \& Stoffel, M. F. (2012). Distributive politics and intergovernmental transfers: The local allocation of European Union structural funds. \textit{European Union Politics}. 13(3): 413–433.
	\item Patz, R., \& Goetz, K. H. (2015). From Politicised Budgeting to Political Budgets in the EU?. Available here: \url{http://www.icpublicpolicy.org/conference/file/reponse/1434615065.pdf}.
	\item Citi, M. (2013). EU budgetary dynamics: incremental or punctuated equilibrium?. \textit{Journal of European Public Policy}. 20(8): 1157-1173.
	\item Citi, M. (2014). Reforming the EU budget: A Time Series Analysis of Institutional and Partisan Effects. Available here: \url{http://papers.ssrn.com/sol3/papers.cfm?abstract_id=2432087}.
	\item Alt, J. E. \& Lassen, D. D. (2006). Transparency, political polarization, and political budget cycles in OECD countries. \textit{American Journal of Political Science}. 50(3): 530–550.
	\item Bailer, S., Mattila, M., \& Schneider, G. (2015). Money makes the EU go round: The objective foundations of conflict in the council of ministers. \textit{Journal of Common Market Studies}. 53(3): 437-456.
	\item Blavoukos, S. \& Pagoulatos, G. (2011). Accounting for coalition-building in the European Union: Budget negotiations and the south. \textit{European Journal of Political Research}. 50(4): 559–581.
	\item Bojar, A. (2014). Intra-governmental bargaining and political budget cycles in the European Union. \textit{European Union Politics}. 15(1): 132–151.
	\item Breunig, C. (2006). The more things change, the more things stay the same: A comparative analysis of budget punctuations. \textit{Journal of European Public Policy}. 13(7): 1069–1085.
	\item Goldbach, R. \& Fahrholz, C. (2011). The euro area’s common default risk: Evidence on the Commission’s impact on European fiscal affairs. \textit{European Union Politics}. 12(4): 507–528.
	\item Jones, B. D., Baumgartner, F. R., Breunig, C., et al. (2009). A general empirical law of public budgets: A comparative analysis. \textit{American Journal of Political Science}. 53(4): 855–873.
	\item K{\"o}nig, T. \& Troeger, V. E. (2005). Budgetary politics and veto players. \textit{Swiss Political Science Review}. 11(4): 47–75.
	\item Laffan, B. (2000). The big budgetary bargains: From negotiation to authority. \textit{Journal of European Public Policy}. 7(5): 725–743.
	\item Mink, M. \& de Haan, J. (2006). Are there political budget cycles in the euro area? \textit{European Union Politics}. 7(2): 191–211.
\end{itemize}


\subsubsection*{Week 11: ``Europeanisation" of National Politics; Enlargement and Differentiated Integration}

Study Questions

\begin{enumerate}
	\item What do we mean by ``Europeanisation," according to B\"{o}rzel \& Risse? How might the EU create pressures for change inside the member states of the EU? Is there evidence that the EU really is resulting in such changes in the EU’s various member states? If so, what do these changes look like?

	\item What kinds of questions does the 2004 enlargement of the European Union pose for students of European integration, according to Schimmelfennig \& Sedelmeier? What evidence do we see of ``Europeanisation" in the new member states of the EU, and what mechanisms seem to be driving the changes we see? How has the EU itself adapted to this increased diversity in membership?
\end{enumerate}

\noindent Required Readings

\begin{itemize}
	\item Cini, M. \& P\'{e}rez-Sol\'{o}rzano Borrag\'{a}n, N. (2019). \textit{European Union Politics}. Oxford: Oxford University Press. Chapter 8,18.
	\item Schimmelfennig, F. \& Sedelmeier, U. (2002). Theorising EU Enlargement: Research Focus, Hypotheses, and the State of Research. \textit{Journal of European Public Policy}. 9(4): 500-528.
	\item Schimmelfennig, F. (2001). The Community Trap: Liberal Norms, Rhetorical Action, and the Eastern Enlargement of the European Union. \textit{International Organization}. 55(1): 47-80.
	\item Holzinger, K. \& Schimmelfennig, F. (2012). Differentiated Integration in the European Union: Many Concepts, Sparse Theory, Few Data. \textit{Journal of European Public Policy}. 19(2): 292–305.
\end{itemize}

\noindent Further reading

\textit{On Europeanization}

\begin{itemize}
	\item B\"{o}rzel, T.A. \& Risse, T. (2000). When Europe hits home: Europeanization and domestic change. \textit{European integration online papers (EIoP)}. 4(15).
	\item Green Cowles, M. Caporaso, J.A. \& Risse, T. (2001). \textit{Transforming Europe: Europeanization and Domestic Change}. Ithaca: Cornell University Press.
	\item Featherstone, K. \& Radaelli, C. eds. (2003). \textit{The Politics of Europeanization}. New York: Oxford University Press.
\end{itemize}

\textit{On Enlargement}

\begin{itemize}
	\item Slapin, J. B. (2015). How European Union membership can undermine the rule of law in emerging democracies. West European Politics, 38(3), 627-648.
	\item Moravcsik, A. \& Vachudova, M. (2002) Bargaining Among Unequals: Enlargement and the Future of European Integration. \textit{EUSA Review}. 15(4): 1-3, available on-line at:
	\url{http://www.eustudies.org/MoravcsikVachudovaEssay.pdf}.
	\item Jacoby, W. (2004). \textit{The Enlargement of the European Union: Ordering from the Menu in Central Europe}. New York: Cambridge University Press.
	\item Vachudova, M.A. (2005). \textit{Europe Undivided: Democracy, Leverage, and Integration after Communism}. New York: Oxford University Press.
	\item Schimmelfennig, F. \& Sedelmeier, U. (eds.) (2005). \textit{The Europeanization of Central and Eastern Europe}. Ithaca: Cornell University Press.
	\item Leuffen, D., Rittberger, B., \& Schimmelfennig, F. (2013). \textit{Differentiated Integration: Explaining Variation in the European Union}. Basingstoke: Palgrave Macmillan. Ch. 1.	
\end{itemize}

\subsubsection*{Week 12: The Democratic Deficit, Democratic Legitimacy and Legislative Transparency}  

Study Questions

\begin{enumerate}

	\item What is the EU's ``democratic deficit"?  What are the fundamental causes of this deficit, and why is it unlikely to be fixed in the near future? Why does Moravcsik argue that the EU is not particularly ``broken" and so should bot be ``fixed"? Why, finally, do Follesdal \& Hix believe that the democratic deficit is more serious than Moravcsik (and another analyst, Giandomenico Majone) thinks, and what do they recommend as to fix the problem? Which view do you find most convincing, and why?

	\item The Laeken declaration (2001) states that: ``the European project […] derives its legitimacy from democratic, transparent and efficient institutions". At the same time, transparency is seen by Naurin as a necessary but not sufficient condition for democratic accountability in the European Union. How transparent is the EU in general and the legislative process in particular. Has this changed over time? What are the implications of the current levels of transparency in EU politics for the aforementioned democratic deficit? Finally, can the EU be simultaneously democratic transparent and efficient?

\end{enumerate}

\noindent Required Readings

\begin{itemize} 
	\item Cini, M. \& P\'{e}rez-Sol\'{o}rzano Borrag\'{a}n, N. (2019). \textit{European Union Politics}. Oxford: Oxford University Press. Chapter 9.
	\item The European Council. (2001). \textit{The Laeken declaration}. Available here: \url{http://european-convention.eu.int/pdf/lknen.pdf}
	\item Moravcsik, A. (2002). In Defense of the Democratic Deficit: Reassessing Legitimacy in the European Union. \textit{Journal of Common Market Studies}. 40(4): 603-624.
	\item Follesdal, A. \& Hix, S. (2006). Why there is a democratic deficit in the EU: A response to Majone and Moravcsik. \textit{Journal of common market studies}. 44(3): 533-562.
	\item Naurin, D. (2007). \textit{Deliberation behind closed doors: transparency and lobbying in the European Union}. ECPR press. Chapter 1-2.
	\item Cross, J.P. (2013a). Striking a pose: transparency and position taking in the Council of the European Union. \textit{European Journal of Political Research}. 52(3): 291–315.
\end{itemize}

\noindent Further reading

\textit{The Democratic Deficit and Legislative Transparency}

\begin{itemize}
	\item Scharpf, F.W. (1999). \textit{Governing in Europe: Effective and Democratic?}. New York: Oxford University Press.
	\item Olsen, J.P. Sbragia, A. \& Scharpf, F.W. (2002). Symposium: Governing in Europe: Effective and Democratic? \textit{Journal of European Public Policy}. 7(2): 310-24.
	\item Cross, J.P. (2014). The seen and the unseen in legislative politics: explaining censorship in the Council of Ministers of the European Union. \textit{Journal of European Public Policy}. 21(2): 268-285.
	\item Cross, J.P. \& B\o lstad, J. (2015). Openness and Censorship in the EU: an Interrupted Time-Series Analysis. \textit{European Union Politics}. 16(2): 216-240.
	\item Crombez, C. (2003). The Democratic Deficit in the European Union: Much Ado about Nothing? \textit{European Union Politics}. 4(1): 101–120.
	\item Hobolt, S.B. (2012). Citizen Satisfaction with Democracy in the European Union*. \textit{Journal of Common Market Studies}. 50(s1): 88–105.
	\item Majone, G. (1998). Europe’s ``democratic deficit": The question of standards. \textit{European Law Journal}. 4(1): 5–28.
	\item Stasavage, D. (2004). Open-door or closed-door? Transparency in domestic and international bargaining. \textit{International Organization}. 58(04): 667–703.
	\item Stasavage, D. (2007). Polarization and publicity: rethinking the benefits of deliberative democracy. \textit{Journal of Politics}. 69(1): 59–72. 
	\item Zweifel, T.D. (2002). ... Who is without sin cast the first stone: the EU's democratic deficit in comparison. \textit{Journal of European Public Policy}. 9(5): 812–840.
\end{itemize}


