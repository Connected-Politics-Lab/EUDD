%!TEX root = POL40970_2016.tex

\subsection*{Grading Criteria}

In essence, markers assess four crucial elements in any answer:

\begin{itemize}
	\item Analysis/understanding
	\item Extent and use of reading
	\item Organisation/structure
	\item Writing proficiency
\end{itemize}

The various grades/classifications reflect the extent to which an answer displays essential features of each of these elements (and their relative weighting). At its simplest: the better the analysis, the wider the range of appropriate sources consulted, the greater the understanding of the materials read, the clearer the writing style, and the more structured the argument, the higher will be the mark. 

The following provides an indicative outline of the criteria used by markers to award a particular grade/classification. If you are in any confusion about how to correctly approach referencing and bibliography issues, there are many good guides available online (Here for instance: \url{http://www.ucd.ie/t4cms/Guide69.pdf}). Proper referencing is ESSENTIAL in a good assignment.

\subsection*{Grade explanation}

\subsubsection*{Grade: A}

\textit{Excellent Performance}

A deep and systematic engagement with the assessment task, with consistently impressive demonstration of a comprehensive mastery of the subject matter, reflecting:

\begin{itemize}
	\item A deep and broad knowledge and critical insight as well as extensive reading;
	\item A critical and comprehensive appreciation
	of the relevant literature or theoretical, technical or professional framework
	\item An exceptional ability to organise, analyse and present arguments fluently and lucidly with a high level of critical analysis, amply supported by evidence, citation or quotation;
	\item A highly-developed capacity for original,
	creative and logical thinking
	\item An extensive and detailed knowledge of the subject matter
	\item A highly-developed ability to apply this knowledge to the task set
	\item Evidence of extensive background reading
	\item Clear, fluent, stimulating and original expression
	\item Excellent presentation (spelling, grammar, graphical) with minimal or no presentation errors
	\item Referencing style consistently executed in recognised style
\end{itemize}

\subsubsection*{Grade: B}

\textit{Very Good Performance}

A thorough and well organised response to the assessment task, demonstrating:

\begin{itemize}
	\item A thorough familiarity with the relevant literature or theoretical, technical or professional framework
	\item Well-developed capacity to analyse issues, organise material, present arguments clearly and cogently well supported by evidence, citation or quotation;
	\item Some original insights and capacity for creative and logical thinking
	\item A broad knowledge of the subject matter
	\item Considerable strength in applying that knowledge to the task set
	\item Evidence of substantial background reading
	\item Clear and fluent expression
	\item Quality presentation with few presentation errors
	\item Referencing style for the most part consistently executed in recognised style
\end{itemize}

\subsubsection*{Grade: C}

\textit{Good Performance}

An intellectually competent and factually sound answer with, marked by:

\begin{itemize}
	\item Evidence of a reasonable familiarity with the relevant literature or theoretical, technical or professional framework
	\item Good developed arguments, but more statements of ideas
	\item Arguments or statements adequately but not well supported by evidence, citation or quotation
	\item Some critical awareness and analytical qualities
	\item Some evidence of capacity for original and logical thinking
	\item Adequate but not complete
	knowledge of the subject matter
	\item Omission of some important subject matter or the appearance of several minor errors
	\item Capacity to apply knowledge appropriately to the task albeit with some errors
	\item Evidence of some background reading
	\item Clear expression with few areas of confusion
	\item Writing of sufficient quality to convey meaning but some lack of fluency and command of suitable vocabulary
	\item Good presentation with some presentation errors
	\item Referencing style executed in recognised style, but with some errors
\end{itemize}

\subsubsection*{Grade: D}

\textit{Satisfactory Performance}

An acceptable level of intellectual engagement with the assessment task showing:

\begin{itemize}
	\item Some familiarity with the relevant literature or theoretical, technical or professional framework
	\item Mostly statements of ideas, with limited development of argument
	\item Limited use of evidence, citation or quotation
	\item Limited critical awareness displayed
	\item Limited evidence of capacity for original and logical thinking
	\item Basic grasp of subject matter, but somewhat lacking in focus and structure
	\item Main points covered but insufficient detail
	\item Some effort to apply knowledge to the task but only a basic capacity or understanding displayed
	\item Little or no evidence of background reading
	\item Several minor errors or one major error
	\item Satisfactory presentation with an acceptable level of presentation errors
	\item Referencing style inconsistent
\end{itemize}

\subsubsection*{Grade: D-}

\textit{Acceptable}

The minimum acceptable of intellectual engagement with the assessment task which:

\begin{itemize}
	\item The minimum acceptable appreciation of the relevant literature or theoretical, technical or professional framework
	\item Ideas largely expressed as statements, with little or no developed or structured argument
	\item Minimum acceptable use of evidence, citation or quotation
	\item Little or no analysis or critical awareness displayed or is only partially successful
	\item Little or no demonstrated capacity for original and logical thinking
	\item Shows a basic grasp of subject matter but may be poorly focussed or badly structured or contain irrelevant material
	\item Has one major error and some minor errors
	\item Demonstrates the capacity to
	complete only moderately difficult tasks related to the subject material
	\item No evidence of background reading
	\item Displays the minimum acceptable standard of presentation (spelling, grammar, graphical)
	\item Referencing inconsistent with major errors
\end{itemize}


\subsubsection*{Grade: E}

\textit{Fail (marginal)}

A factually sound answer with a partially successful, but not entirely acceptable, attempt to:

\begin{itemize}
	\item Integrate factual knowledge into a broader literature or theoretical, technical or professional framework develop arguments
	\item Support ideas or arguments with evidence, citation or quotation
	\item Engages with the subject matter or
	problem set, despite major
	deficiencies in structure, relevance or focus
	\item Has two major error and some minor
	errors
	\item Demonstrates the capacity to
	complete only part of, or the simpler
	elements of, the task
	\item An incomplete or rushed answer e.g.
	the use of bullet points through part /
	all of answer
	\item Little or no referencing style evident
\end{itemize}

\subsubsection*{Grade: F}

\textit{Fail (unacceptable)}

An unacceptable level of intellectual engagement with
the assessment task, with:

\begin{itemize}
	\item No appreciation of the relevant literature or theoretical, technical or professional framework
	\item No developed or structured argument
	\item No use of evidence, citation or quotation
	\item No analysis or critical awareness displayed or is only partially successful
	\item No demonstrated capacity for original and logical thinking
	\item A failure to address the question resulting in a largely irrelevant answer or material of marginal relevance predominating
	\item A display of some knowledge of material relative to the question posed, but with very serious omissions / errors and/or major inaccuracies included in answer
	\item Solutions offered to a very limited portion of the problem set
	\item An answer unacceptably incomplete (e.g. for lack of time)
	\item A random and undisciplined development, layout or presentation
	\item Unacceptable standards of presentation, such as grammar, spelling or graphical presentation
	\item Evidence of substantial plagiarism
	\item No referencing style evident
\end{itemize}

\subsubsection*{Grade: G}

\textit{Fail (wholly unacceptable)}

No intellectual engagement with the assessment task

\begin{itemize}
	\item Complete failure to address the question resulting in an entirely irrelevant answer
	\item Little or no knowledge displayed relative to the question posed
	\item Little or no solution offered for the problem set
	\item Evidence of extensive plagiarism
	\item No referencing style evident
\end{itemize}

\subsubsection*{Grade: NG}

\textit{No grade (no work was submitted by the student or student was absent from the assessment, or work submitted did not merit a grade).}

\subsection*{Extenuating circumstances}

In the case that a student will not be able to meet an assessment deadline or will be absent from the module for an extended period of time, and this is known \textbf{IN ADVANCE}, they should consult the UCD policies on extenuating circumstances found here:
\url{http://www.ucd.ie/registry/academicsecretariat/extc.htm}. It is important that in such cases you make the issue known to the lecturer as son as possible. The sooner that the lecturer is made aware of the situation, the more likely it is that you can be accommodated.

\subsection*{Late assignment submissions}

If a student submits an assignment late, the following penalties will be applied:

\begin{itemize}
	\item Coursework received at any time within two weeks of the due date will be graded, but a penalty will apply.
	\begin{itemize}
		\item Coursework submitted at any time up to one week after the due date will have the grade awarded reduced by two grade points (for example, from B- to C).
		\item Coursework submitted more than one week but up to two weeks after the due date will have the grade reduced by four grade points (for example, from B- to D+).
Where a student finds they have missed a deadline for submission, they should be advised that they may use the remainder of the week to improve their submission without additional penalty.
	\end{itemize}
	\item Coursework received more than two weeks after the due date will not be accepted.
\end{itemize}
