%!TEX root = POL40970_2016.tex

\section*{Module structure}

	The module follows the seminar format. This means that in general students will be responsible for directing individual class sessions. It also means that the module requires you to read a considerable amount of material, think about what you have read, and regularly talk about your understanding of the readings. Each week different student groups will lead the discussion by preparing a presentation to introduce the topic to their peers and run the seminar for the 1st hour. A peer-review exercise will help students develop the drafts of their research papers, and a final research paper project will constitute the end of module assessment. More details on assessment requirements are provided below.

\section*{Learning outcomes}

At the end of this module students will be able to:

\begin{itemize}
	\item Explain the historical development of the European integration project from its beginnings as the European Coal and Steel Community to the institution we currently see today.
	\item Explain how the three main legislative institutions of the EU function together and produce policy.
	\item Analyse European integration using different theoretical approaches.
	\item Discuss different EU policy areas and the manner in which the EU functions in these areas.
	\item Explain key concepts about integration to others both verbally and in writing.
	\item Discuss the current and future challenges facing the EU in light of the theories discussed throughout the module.
\end{itemize}