%!TEX root = POL40970_2016.tex

\section*{Module assessment}

\subsection*{Module grading scheme}

\begin{itemize}
	\item Attendance \& participation: 10\% (A doctors note or equivalent is required to be excused from class. Each unexcused absence = -2\% of the possible 10\% available for attendance/participation) 
	\item Running a seminar: 20\% (includes outline \& materials) 
	\item Peer-review work: 20\%
	\item Final research paper: 50\%
\end{itemize}

\subsection*{In-class participation and reading}

	This module will require both extensive and attentive reading, and regular in-class participation. As a consequence, all students are required to attend all lectures/seminars unless prior notification and a certified excuse is presented to the module lecturers. Students also need to demonstrate that they have completed the readings and have thought about the issues involved. The success of the module depends on the commitment of its students. As such, full attendance and participation will form 10\% of the final grade.

\subsection*{Student-led seminar and presentation}

	In addition, students will be required to lead discussion as part of a group once during the semester (preparation of handouts and/or overheads recommended). Presentations will relate to the seminar topics for a particular week, and topics will be assigned at the beginning of the term. %Students’ presentations will be assessed according to the criteria below.

%\subsubsection*{Presentation grading scheme:}

%\begin{itemize}
%	\item Content: Clarity of the puzzle or problem being addressed; Quality of argument that is presented; Structure of presentation; Coherent conclusion (25 \%).
%	\item Research skills: Evidence of research and use of sources; Interpretation of evidence and analysis (25 \%)
%	\item Discussion Skills: Responsiveness to class comments/questions; Listening and engaging class in discussion; class activities (25 \%)
%	\item Presentation skills	Fluency; Familiarity with own material; Use of audio-visual aids or handouts (25 \%) 
%\end{itemize}

\subsection*{Peer review}

%http://teachingcenter.wustl.edu/strategies/Pages/peer-review.aspx
%http://teachingcenter.wustl.edu/strategies/Pages/peer-review-how-to.aspx

	The peer review process is central to publishing work in academia. Part of your assessment for this module will therefore involve peer reviewing one each others' work. Students will be required to submit a first draft of their research paper on \textbf{1 November 2019}. The draft essays will then be distributed for peer review. Each student is required to write a review of two papers. These reviews are due on \textbf{15 November 2019} and are  graded by the professor and then returned to the authors. The advantage of this approach is that students can use these reviews to improve their work when redrafting their final research paper for submission. A full set of peer-review guidelines will be distributed before this exercise, and a peer-review workshop during \textbf{Week 8} of the semester will prepare students for the task.

Through this process you will:

\begin{itemize}
	\item Learn how to carefully read a piece of writing, with attention to the details of the piece in terms of structure and content (whether the piece is your own or another writer's);
	\item Learn how to strengthen your writing by taking into account the responses of actual and anticipated readers;
	\item Make the transition from writing primarily for yourselves or for an instructor to writing for a broader audience
	\item Learn how to formulate and communicate constructive feedback on a peer's work;
	\item Learn how to gather and respond to feedback on your own work.
\end{itemize}

\subsection*{Research paper}

	Students will submit a single well developed research paper on a topic relevant to European integration. This paper should be no longer than \textbf{5,000 words} including bibliography, and should analyse an issue of importance relating to the EU or European integration that  merits academic research. The analysis should be original and should include an evaluation of approaches to understanding, resolving or further investigating the question. This paper is not a literature review and marks will be awarded for applying original ideas or approaches to established thoughts on the issue. The lecturer will provide plenty of advice and help in choosing a topic and an appropriate research methodology. It is important that students begin to think about the topic they wish to write about immediately, as this is not a trivial undertaking. Identifying a good research question from the start will save much work down the line. The deadline for the first draft of the research paper for peer review is \textbf{1 November 2019}. It is essential that a first draft is submitted for this date so that you can receive peer reviews from your peers. The deadline for the final paper is \textbf{5pm on 6 December 2019}. The research paper will be submitted through Brightspace.

\subsection*{Important dates}

\begin{itemize}
	\item \textbf{1 November 2019}: First draft of research paper due
	\item \textbf{4 November 2019}: Peer-review workshop
	\item \textbf{15 November 2019}: Peer-review assignment due
	\item \textbf{6 December 2019}: Final research paper due
\end{itemize}
