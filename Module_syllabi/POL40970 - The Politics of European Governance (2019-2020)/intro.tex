%!TEX root = POL40970_2016.tex

\section*{Introduction}

	The European Union (EU) currently consists of 28 European countries. It has evolved into a political community that deeply affects the daily lives of its citizens. How can we explain the process of European Integration? What does the political structure of the EU look like? The module surveys the development of European integration and introduces students to the EU’s political system. We will discuss theories of European integration and examine major challenges, that the EU faces at the beginning of the 21st century. The module is research lead, in that it focuses upon the political science literature that seeks to explain various aspects of European integration. This is reflected in the extensive reading list provided with the module.

	The module is aimed at those wanting a full and detailed introduction to (1) the history and theories of European integration; (2) the political institutions of the EU and its policy-making process; and (3) current and future challenges facing the EU as a political system. The module does not assume any prior knowledge of the EU or EU politics.